\documentclass[a4paper,12pt]{article}
\usepackage{amsmath}
\usepackage{amssymb}
\usepackage{amsthm}
\usepackage{marvosym}
\usepackage[utf8]{inputenc}
\usepackage[T1]{fontenc}
\usepackage[ngerman]{babel}
\usepackage[left=2.5cm,right=2cm,top=3cm,bottom=3cm]{geometry}
\usepackage{fancyhdr}
\usepackage{tabularx}
\usepackage{booktabs}
\usepackage{amsfonts}
\usepackage{tcolorbox}
\usepackage{adjustbox}
\usepackage{eurosym}
\usepackage[inline]{enumitem}
\usepackage{lastpage}
\usepackage{mathtools}
\usepackage{microtype}
\usepackage[ruled, linesnumbered, noline]{algorithm2e}
\usepackage{listings}
\usepackage{xcolor}

\setlength{\parindent}{0pt}

\usepackage{tikz}
\usetikzlibrary{decorations.pathmorphing,calc,shadows.blur,shadings,er,positioning}

\definecolor{codegreen}{rgb}{0,0.6,0}
\definecolor{codegray}{rgb}{0.5,0.5,0.5}
\definecolor{codepurple}{rgb}{0.58,0,0.82}
\definecolor{backcolour}{rgb}{0.95,0.95,0.92}

\lstdefinestyle{mystyle}{
	backgroundcolor=\color{backcolour},   
	commentstyle=\color{codegreen},
	keywordstyle=\color{magenta},
	numberstyle=\tiny\color{codegray},
	stringstyle=\color{codepurple},
	basicstyle=\ttfamily\footnotesize,
	breakatwhitespace=false,         
	breaklines=true,                 
	captionpos=b,                    
	keepspaces=true,                 
	numbers=left,                    
	numbersep=5pt,                  
	showspaces=false,                
	showstringspaces=false,
	showtabs=false,                  
	tabsize=2
}

\lstset{style=mystyle}
%\usepackage{pgfplots}
%\usepackage[leqno]{amsmath}

%\newcounter{mathseed}
%\setcounter{mathseed}{1}
%\pgfmathsetseed{\arabic{mathseed}} 
%\pgfdeclarelayer{background}
%\pgfsetlayers{background,main}

\newcommand{\RN}[1]{%
	\textup{\uppercase\expandafter{\romannumeral#1}}%
}

\newcommand{\obda}{o.B.d.A. }

\newenvironment{enum}{\begin{enumerate*}[label=\alph*),itemjoin=\hspace{2em}]}{\end{enumerate*}\\[2ex]}

\DeclareMathOperator{\N}{\mathbb N}
\DeclareMathOperator{\Q}{\mathbb Q}
\DeclareMathOperator{\R}{\mathbb R}
\DeclareMathOperator{\Z}{\mathbb Z}
\DeclareMathOperator{\C}{\mathbb C}
\DeclareMathOperator{\F}{\mathcal{F}}
\DeclareMathOperator{\E}{\mathcal{E}}
\DeclareMathOperator{\BigO}{\mathcal O}
\DeclareMathOperator{\mL}{\mathcal{L}}
\DeclareMathOperator{\mU}{\mathcal{U}}
\DeclareMathOperator{\ggt}{\text{ggT}}
\DeclareMathOperator{\kgv}{\text{kgV}}


%------ Defining multiple types of theorems ------%
\newtheorem{axiom}{Axiom}[section]
\newtheorem{theorem}[axiom]{Theorem}
\newtheorem{lemma}[axiom]{Lemma}
\newtheorem{satz}[axiom]{Satz}
\theoremstyle{definition}
\newtheorem*{example}{Beispiel}
\newtheorem*{bemerkung}{Bemerkung}
\newtheorem*{frage}{Frage}
\newtheorem*{loesung}{Lösung}
\newtheorem*{behauptung}{Behauptung}
\newtheorem*{beobachtung}{Beobachtung}
\newtheorem{definition}[axiom]{Definition}
\newtheorem{folg}[axiom]{Folgerung}
\newtheorem{notation}[axiom]{Notation}

%---- Changing forall and exists ----%
\let\oldforall\forall
\renewcommand{\forall}{\:\oldforall \, }
\let\oldexist\exists
\renewcommand{\exists}{\:\oldexist \: }
\newcommand\existu{\oldexist! \: }
\let\oldepsilon\epsilon
\renewcommand{\epsilon}{\varepsilon}
\newcommand{\corresponds}{\;\widehat{=}\;}

%------ Giving circled star ---------%
\makeatletter
\newcommand{\ostar}{\mathbin{\mathpalette\make@circled\star}}
\newcommand{\make@circled}[2]{%
	\ooalign{$\m@th#1\smallbigcirc{#1}$\cr\hidewidth$\m@th#1#2$\hidewidth\cr}%
}
\newcommand{\smallbigcirc}[1]{%
	\vcenter{\hbox{\scalebox{0.77778}{$\m@th#1\bigcirc$}}}%
}
\makeatother
%-----------------------------------%

%-------------- Vars ---------------%
\newcommand{\vldate}{19. Dezember 2024}
\newcommand{\vlname}{Computerorientierte Mathematik I}
\newcommand{\vltopic}{Lineare Sortieralgorithmen}

\pagestyle{fancy}
\setlength{\headheight}{34pt}
\fancyhf{}
\fancyhead[L]{\includegraphics[width=1cm]{C:/Users/chole/Documents/tub/tublogo.png}}
\fancyhead[C]{{\bfseries\vlname}\\%
	\vltopic}
\fancyhead[R]{\vldate}
\fancyfoot[L]{}
\fancyfoot[C]{- \thepage\ of \pageref{LastPage} -}
\fancyfoot[R]{}

\renewcommand{\headrulewidth}{1pt}
\renewcommand{\footrulewidth}{1pt}
\renewcommand{\labelitemii}{-}

\begin{document}
	\setcounter{section}{8}
	\section{Sortieren in Linearzeit}
	\subsection{Counting Sort}
	\begin{algorithm}[h]
		\caption{CountingSort}
		\KwIn{Array der Länge $n$ mit Einträgen $A[i] \in \{1, 2, \ldots, k\}$}
		\KwOut{Sortiertes Array $B$}
		Let $C$ be Array of length $k$\\
		\For{$i \gets 1$ \KwTo $k$}{
			$C[i] \gets 0$
		}
		\For{$j \gets 1$ \KwTo $n$}{
			$C[A[j]] \gets C[A[j]] + 1$
		}
		\For{$i \gets 1$ \KwTo $k - 1$}{
			$C[i + 1] \gets C[i + 1] + C[i]$
		}
		Let $B$ be Array of length $n$
		\For{$j \gets n$ \KwTo $1$}{
			$B[C[A[j]]] \gets A[j]$\\
			$C[A[j]] \gets C[A[j]] - 1$
		}
	\end{algorithm}
	\begin{satz}
		CountingSort arbeitet korrekt und besitzt Laufzeitfunktion in $\Theta(n + k)$.
	\end{satz}
	\subsection{RadixSort}
	\begin{algorithm}[h]
		\caption{RadixSort}
		\KwIn{Array der Länge $n$. Jeder Eintrag ist $d$-stellige Zahl, wobei die erste Stelle die niederste und die letzte Stelle die höchste ist.}
		\KwOut{Sortiertes Array.}
		\For{$i \gets 1$ \KwTo $d$}{sortiere $A$ nach $i$-ter Stelle mit stabilem Sortierverfahren.}
	\end{algorithm}
	\begin{satz}
		Für $n$ Zahlen mit je $d$ Stellen, bei denen jede Stelle bis zu $k$ mögliche Werte annehmen kann, sortiert RadixSort diese Zahlen korrekt in Zeit $\Theta\big(d(n + k)\big)$, falls das stabile Sortierverfahren in Zeile $2$ Laufzeit $\Theta(n + k)$ besitzt (bspw. CountingSort).
	\end{satz}
	\subsection{BucketSort}
	\begin{algorithm}[H]
		\caption{BucketSort}
		\KwIn{Array $A$ der Länge $n$ mit Einträgen $A[i] \in (0, 1]$}
		\KwOut{Sortierte Liste}
		Let $B$ be Array of length $n$
		\For{$i \gets 1$ \KwTo $n$}{
			$B[i] \gets [~]$
		}
		\For{$i \gets 1$ \KwTo $n$}{
			insert $A[i]$ into list $B\big[\lceil n \cdot A[i]\rceil\big]$
		}
		\For{$i = 1$ \KwTo $n$}{
			sort list $B[i]$ with insertionSort
		}
		concatenate lists $B[1], B[2], \ldots, B[n]$
	\end{algorithm}
	\begin{satz}
		Werden die Einträge unabhängig und gleichverteilt aus $(0, 1]$ gezogen, so ist die erwartete Laufzeit von BucketSort in $\BigO(n)$.
	\end{satz}
	\begin{bemerkung}
		\begin{itemize}
			\item Die erwartete Laufzeit von BucketSort ist linear in $n$, da die erwartete Summe der Quadrate der Bucketgrößen linear in $n$ ist.
			\item BucketSort kann durch Skalierung für beliebige Zahlenbereiche angewandt werden.
		\end{itemize}
	\end{bemerkung}
	\begin{proof}[Laufzeitbeweis.]
		Es sei $n$ die Anzahl der Elemente in Bucket $B[i], i = 1, \ldots, n$. Dann gilt für die Laufzeitfunktion $T(n)$ von BucketSort:
		\[
			T(n) \leq c \cdot n + \sum_{i = 1}^{n} d \cdot n_i^2
		\]
		für Konstanten $c, d > 0$. Wegen Linearität des Erwartungswertes gilt für die erwartete Laufzeit:
		\[
			E[T(n)] \leq c \cdot n + \sum_{i = 1}^{n} d \cdot E[n_i^2]
		\]
		Definiere Zufallsvariablen
		\[
			x_{ij} \leq \begin{dcases*}
				1, \text{ falls $A[j]$ in Bucket $B[i]$ landet.}\\
				0, \text{ sonst.}
			\end{dcases*}
		\]
		Dann ist 
		\[
			n_i = \sum_{j = 1}^{n}x_{ij}
		\]
		und
		\begin{align*}
			E[n_i^2] &= E\Big[\Big(\sum_{j = 1}^{n}x_{ij}\Big)\Big]\\
			&= E\Big[\sum_{j = 1}^{n}\sum_{k = 1}^{n}x_{ij}x_{ik}\Big]\\
			&= E\Big[\sum_{j = 1}^{n}x_{ij}^2 + \sum_{j = 1}^{n}\sum_{k \neq j}^{}x_{ij}x_{ik}\Big]\\
			&= \sum_{j = 1}^{n}E[x_{ij}^2] + \sum_{j = 1}^{n}\sum_{k \neq j}^{}E[x_{ij}x_{ik}]\\
			&= \sum_{j = 1}^{k}\frac{1}{n} + \sum_{j = 1}^{n}\sum_{k \neq j}^{}E[x_{ij}]E[x_{ik}]\\
			&= 1 + \frac{n(n - 1)}{n^2} = 2 - \frac{1}{n}
		\end{align*}
	\end{proof}
\end{document}