\documentclass[a4paper,12pt]{article}
\usepackage{amsmath}
\usepackage{amssymb}
\usepackage{amsthm}
\usepackage{marvosym}
\usepackage[utf8]{inputenc}
\usepackage[T1]{fontenc}
\usepackage[ngerman]{babel}
\usepackage[left=2.5cm,right=2cm,top=3cm,bottom=3cm]{geometry}
\usepackage{fancyhdr}
\usepackage{tabularx}
\usepackage{booktabs}
\usepackage{amsfonts}
\usepackage{tcolorbox}
\usepackage{adjustbox}
\usepackage{eurosym}
\usepackage[inline]{enumitem}
\usepackage{lastpage}
\usepackage{mathtools}
\usepackage{microtype}
\usepackage{algorithm}
\usepackage{algpseudocode}
\usepackage{listings}
\usepackage{xcolor}
%\usepackage{minted}
\setlength{\parindent}{0pt}

\usepackage{tikz}
\usetikzlibrary{decorations.pathmorphing,calc,shadows.blur,shadings,er,positioning}

\definecolor{codegreen}{rgb}{0,0.6,0}
\definecolor{codegray}{rgb}{0.5,0.5,0.5}
\definecolor{codepurple}{rgb}{0.58,0,0.82}
\definecolor{backcolour}{rgb}{0.95,0.95,0.92}

\lstdefinestyle{mystyle}{
	backgroundcolor=\color{backcolour},   
	commentstyle=\color{codegreen},
	keywordstyle=\color{magenta},
	numberstyle=\tiny\color{codegray},
	stringstyle=\color{codepurple},
	basicstyle=\ttfamily\footnotesize,
	breakatwhitespace=false,         
	breaklines=true,                 
	captionpos=b,                    
	keepspaces=true,                 
	numbers=left,                    
	numbersep=5pt,                  
	showspaces=false,                
	showstringspaces=false,
	showtabs=false,                  
	tabsize=2
}

\lstset{style=mystyle}
%\usepackage{pgfplots}
%\usepackage[leqno]{amsmath}

%\newcounter{mathseed}
%\setcounter{mathseed}{1}
%\pgfmathsetseed{\arabic{mathseed}} 
%\pgfdeclarelayer{background}
%\pgfsetlayers{background,main}

\newcommand{\RN}[1]{%
	\textup{\uppercase\expandafter{\romannumeral#1}}%
}

\newcommand{\obda}{o.B.d.A. }

\newenvironment{enum}{\begin{enumerate*}[label=\alph*),itemjoin=\hspace{2em}]}{\end{enumerate*}\\[2ex]}

\DeclareMathOperator{\N}{\mathbb N}
\DeclareMathOperator{\Q}{\mathbb Q}
\DeclareMathOperator{\R}{\mathbb R}
\DeclareMathOperator{\Z}{\mathbb Z}
\DeclareMathOperator{\BigO}{\mathcal O}
\DeclareMathOperator{\ggt}{\text{ggT}}
\DeclareMathOperator{\kgv}{\text{kgV}}


%------ Defining multiple types of theorems ------%
\newtheorem{axiom}{Axiom}[section]
\newtheorem{theorem}[axiom]{Theorem}
\newtheorem{lemma}[axiom]{Lemma}
\newtheorem{satz}[axiom]{Satz}
\theoremstyle{definition}
\newtheorem*{example}{Beispiel}
\newtheorem*{bemerkung}{Bemerkung}
\newtheorem*{frage}{Frage}
\newtheorem*{loesung}{Lösung}
\newtheorem{definition}[axiom]{Definition}
\newtheorem{folg}[axiom]{Folgerung}
\newtheorem{notation}[axiom]{Notation}

%---- Changing forall and exists ----%
\let\oldforall\forall
\renewcommand{\forall}{\:\oldforall \, }
\let\oldexist\exists
\renewcommand{\exists}{\:\oldexist \: }
\newcommand\existu{\oldexist! \: }
\let\oldepsilon\epsilon
\renewcommand{\epsilon}{\varepsilon}
\newcommand{\corresponds}{\;\widehat{=}\;}

%------ Giving circled star ---------%
\makeatletter
\newcommand{\ostar}{\mathbin{\mathpalette\make@circled\star}}
\newcommand{\make@circled}[2]{%
	\ooalign{$\m@th#1\smallbigcirc{#1}$\cr\hidewidth$\m@th#1#2$\hidewidth\cr}%
}
\newcommand{\smallbigcirc}[1]{%
	\vcenter{\hbox{\scalebox{0.77778}{$\m@th#1\bigcirc$}}}%
}
\makeatother
%-----------------------------------%

%-------------- Vars ---------------%
\newcommand{\vldate}{05. November 2024}
\newcommand{\vlname}{Computerorientierte Mathematik I}
\newcommand{\vltopic}{Karazuas Multiplikation -- Laufzeit}

\pagestyle{fancy}
\setlength{\headheight}{34pt}
\fancyhf{}
\fancyhead[L]{\includegraphics[width=1cm]{C:/Users/chole/Documents/tub/tublogo.png}}
\fancyhead[C]{{\bfseries\vlname}\\%
	\vltopic}
\fancyhead[R]{\vldate}
\fancyfoot[L]{}
\fancyfoot[C]{- \thepage\ of \pageref{LastPage} -}
\fancyfoot[R]{}

\renewcommand{\headrulewidth}{1pt}
\renewcommand{\footrulewidth}{1pt}
\renewcommand{\labelitemii}{-}

\begin{document}
	Für $a,b \in \N$ gilt:
	\[
		\text{ggT}(a,b)\cdot \text{kgV}(a,b) = a\cdot b
	\]
	\begin{proof}
		Falls $a = 0$ oder $b = 0$, ist die Aussage klar. Seien also $a,b > 0$. Es sei $d \coloneq \text{ggT}(a,b), m \coloneq \text{kgV}(a,b)$. Da $b \mid (a\cdot b)$, gibt es $n \in \N$ mit $d\cdot n = a\cdot b$.
		
		Da $d \mid a$ und $d \mid b$, gibt es $u,v$, sodass $d\cdot u = a, d\cdot v = b$.
		\begin{align*}
			\Longrightarrow d \cdot u \cdot b &= d \cdot n = d \cdot v \cdot a\\
			\Longleftrightarrow u \cdot b &= n = v \cdot a
		\end{align*}
		Damit ist $n$ Vielfaches von $a$ und $b$, also $n \geq \kgv(a,b) \eqcolon m$. Da $m$ Vielfaches von $a$ und $b$ ist, gibt es $r,s \in \N$ mit $m = a\cdot r = b \cdot s$. Nach dem Lemma von Bézout gibt es $x,y \in \Z$ mit $d = a \cdot x + b \cdot y$.
		\begin{align*}
			m \cdot d &= m \cdot a \cdot a + m \cdot b \cdot y\\
			&= b \cdot s \cdot a \cdot x + a \cdot r \cdot b \cdot y\\
			&= a \cdot b \cdot (s \cdot x + r \cdot y)\\
			&= d \cdot n \cdot (s \cdot x + r \cdot y)\\
			m &= n \cdot (s \cdot x + r \cdot y)
		\end{align*}
		das heißt, $m$ ist ein Vielfaches von $n$ und daher $m \geq n$.
	\end{proof}
	\setcounter{section}{3}
	\section{Approximative Darstellung reeller Zahlen}
	\subsection{Normalisierte $b$-adische Darstellung reeller Zahlen}
	\begin{satz}
		Es sei $b \in \N_{\geq 2}.$ Für alle $x \in \R \backslash\{0\}$ existieren eindeutige Zahlen $\mathcal{E} \in \Z, \sigma \in \{\pm 1\}$ und $z_i \in \{0,1,\ldots, b-1\}$ für $i \in \N$, sodass gilt:
		\[
			x = \sigma \cdot \Big(\sum_{i = 0}^{\infty}z_i \cdot b^{-i}\Big)
		\]
		wobei $\{i \in \N: z_i \neq (b-1)\}$ unendlich groß ist und $z_0 \neq 0$.
	\end{satz}
	Diese Darstellung heißt dann (normalisierte) $b$-adische Darstellung von $x$.
	\begin{example}
		\begin{align*}
			\pi &= (+1) \cdot (3\cdot 10^0 + 1\cdot 10^{-1} + 4 \cdot 10^{-2} + 1 \cdot 10^{-3} + \dots) \cdot 10^0\\
			-\frac{2}{5} &= (-1) \cdot (4 \cdot 10^0 + 0 \cdot 10^{-1} + 0 \cdot 10^{-2} + \dots) \cdot 10^{-1}\\
			\frac{1}{3} &= (+1) \cdot \Big(\sum_{i = 0}^{\infty}3 \cdot 10^{-i}\Big) \cdot 10^{-1} = 0.\overline{3}\\
			&= (+1) \cdot \Big(\sum_{i = 0}^{\infty}2^{-2i}\Big) \cdot 2^{-2} = (0.\overline{01})_2
		\end{align*}
	\end{example}
	\subsection{Normalisierte $b$-adische Darstellung: Existenz}
	Für $x \in \R \backslash\{0\}$ sei $\sigma \coloneq \frac{x}{\lvert x \rvert}$ und $\mathcal{E} = \lfloor \log_b \lvert x \rvert \rfloor$. Setze $a_0 \coloneq \lvert x\rvert \cdot b^{-\mathcal{E}}$ also $1 \leq a_0 < b$ und definiere rekursiv für $i \in \N$:
	\[
		z_i \coloneq \lfloor a_i \rfloor \text{ und } a_{i+1} \coloneq b \cdot (a_i - z_i)
	\]
	Dann gilt auch $0 \leq a_i < b$ und $z_i \in \{0,\ldots, b-1\}$ für alle $i \in \N$ und $z_0 \neq 0$.
	\begin{satz}
		Für alle $n \in \N$ gilt:
		\[
			a_0 = \sum_{i = 0}^{n} z_i b^{-i} + a_{n+1} \cdot b^{-(n+1)}
		\]
	\end{satz}
	\begin{proof}[Beweis durch Induktion.]
		Induktionsanfang $(n = 0)$:
		\[
			z_0 \cdot b^0 + a_1 \cdot b^{-1} = z_0 \cdot b^0 + b(a_0 - z_0) \cdot b^{-1} = a_0
		\]
		Induktionsschluss $(n \to n + 1)$:
		\begin{align*}
			a_0 &= \sum_{i = 0}^{n}z_i \cdot b^{-i} + a_{n+1} \cdot b^{-(n+1)}\\
			&= \sum_{i = 0}^{n+1}z_i \cdot b^{-i} + \underbrace{(a_{n+1} - z_{n+1}) \cdot b^{-(n+1)}}_{=a_{n+2} \cdot b^{-(n+2)}}
		\end{align*}
	\end{proof}
	Daraus folgt 
	\begin{align*}
		x = \sigma \cdot b^{\mathcal{E}} \cdot a_0 &= \sigma \cdot b^{\mathcal{E}} \cdot \lim_{n\to \infty}\Big(\sum_{i = 0}^{n}z_i \cdot b^{-i}\Big)\\
		&= \sigma \cdot b^{\mathcal{E}} \cdot \sum_{i = 0}^{\infty}z_i \cdot b^{-i}
	\end{align*}
	Wäre $\lvert \{i \in \N \mid z_i \neq b-1\}\rvert < \infty$, so gäbe es $n_0 \in \N$ mit $z_i = b-1$ für alle $i > n_0$.
\end{document}