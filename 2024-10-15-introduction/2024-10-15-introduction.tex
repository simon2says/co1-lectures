\documentclass[a4paper,12pt]{article}
\usepackage{amsmath}
\usepackage{amssymb}
\usepackage{amsthm}
\usepackage{marvosym}
\usepackage[utf8]{inputenc}
\usepackage[T1]{fontenc}
\usepackage[ngerman]{babel}
\usepackage[left=2.5cm,right=2cm,top=3cm,bottom=3cm]{geometry}
\usepackage{fancyhdr}
\usepackage{tabularx}
\usepackage{booktabs}
\usepackage{amsfonts}
\usepackage{tcolorbox}
\usepackage{adjustbox}
\usepackage{eurosym}
\usepackage[inline]{enumitem}
\usepackage{lastpage}
\usepackage{mathtools}
\usepackage{microtype}
\usepackage{algorithm}
\usepackage{algpseudocode}
\usepackage{listings}
\usepackage{xcolor}
%\usepackage{minted}

\usepackage{tikz}
\usetikzlibrary{decorations.pathmorphing,calc,shadows.blur,shadings,er,positioning}

\definecolor{codegreen}{rgb}{0,0.6,0}
\definecolor{codegray}{rgb}{0.5,0.5,0.5}
\definecolor{codepurple}{rgb}{0.58,0,0.82}
\definecolor{backcolour}{rgb}{0.95,0.95,0.92}

\lstdefinestyle{mystyle}{
	backgroundcolor=\color{backcolour},   
	commentstyle=\color{codegreen},
	keywordstyle=\color{magenta},
	numberstyle=\tiny\color{codegray},
	stringstyle=\color{codepurple},
	basicstyle=\ttfamily\footnotesize,
	breakatwhitespace=false,         
	breaklines=true,                 
	captionpos=b,                    
	keepspaces=true,                 
	numbers=left,                    
	numbersep=5pt,                  
	showspaces=false,                
	showstringspaces=false,
	showtabs=false,                  
	tabsize=2
}

\lstset{style=mystyle}
%\usepackage{pgfplots}
%\usepackage[leqno]{amsmath}

%\newcounter{mathseed}
%\setcounter{mathseed}{1}
%\pgfmathsetseed{\arabic{mathseed}} 
%\pgfdeclarelayer{background}
%\pgfsetlayers{background,main}
\newlength{\diebox}
\newcommand{\luecke}[1]{\settowidth{\diebox}{#1}\raisebox{-1.0ex}{\parbox{2\diebox}{\dotfill}}}
\newcommand{\textgrid}[1]
{\begin{tikzpicture}
		\draw[step=0.5cm,color=gray] (0,0) grid (\textwidth ,#1);
\end{tikzpicture}}

\newcommand{\RN}[1]{%
	\textup{\uppercase\expandafter{\romannumeral#1}}%
}

\newcommand{\obda}{o.B.d.A. }

\newenvironment{enum}{\begin{enumerate*}[label=\alph*),itemjoin=\hspace{2em}]}{\end{enumerate*}\\[2ex]}

%------ Defining multiple types of theorems ------%
\newtheorem{axiom}{Axiom}[section]
\newtheorem{theorem}[axiom]{Theorem}
\newtheorem{lemma}[axiom]{Lemma}
\newtheorem{satz}[axiom]{Satz}
\theoremstyle{definition}
\newtheorem*{example}{Beispiel}
\newtheorem*{bemerkung}{Bemerkung}
\newtheorem{definition}[axiom]{Definition}
\newtheorem{folg}[axiom]{Folgerung}
\newtheorem{notation}[axiom]{Notation}

%---- Changing forall and exists ----%
\let\oldforall\forall
\renewcommand{\forall}{\:\oldforall \, }
\let\oldexist\exists
\renewcommand{\exists}{\:\oldexist \: }
\newcommand\existu{\oldexist! \: }
\let\oldepsilon\epsilon
\renewcommand{\epsilon}{\varepsilon}

%------ Giving circled star ---------%
\makeatletter
\newcommand{\ostar}{\mathbin{\mathpalette\make@circled\star}}
\newcommand{\make@circled}[2]{%
	\ooalign{$\m@th#1\smallbigcirc{#1}$\cr\hidewidth$\m@th#1#2$\hidewidth\cr}%
}
\newcommand{\smallbigcirc}[1]{%
	\vcenter{\hbox{\scalebox{0.77778}{$\m@th#1\bigcirc$}}}%
}
\makeatother
%-----------------------------------%

%-------------- Vars ---------------%
\newcommand{\vldate}{15. Oktober 2024}
\newcommand{\vlname}{Computerorientierte Mathematik I}
\newcommand{\vltopic}{Einführung}

\pagestyle{fancy}
\setlength{\headheight}{34pt}
\fancyhf{}
\fancyhead[L]{\includegraphics[width=1cm]{C:/Users/chole/Documents/tub/tublogo.png}}
\fancyhead[C]{{\bfseries\vlname}\\%
	\vltopic}
\fancyhead[R]{\vldate}
\fancyfoot[L]{}
\fancyfoot[C]{- \thepage\ of \pageref{LastPage} -}
\fancyfoot[R]{}

\renewcommand{\thesection}{\arabic{section}}
\renewcommand{\thesubsection}{\arabic{section}.\arabic{subsection}.}
\renewcommand{\headrulewidth}{1pt}
\renewcommand{\footrulewidth}{1pt}
\renewcommand{\labelitemii}{-}

\begin{document}
	\section{Literatur}
	Hougardy, Vygen -- Algorithmische Mathematik\\[2ex]
	Theobald, Iliman -- Einführung in die computerorientierte Mathematik mit Sage\\[2ex]
	Cormen et. al -- Introduction to Algorithms\\[2ex]
	Knuth -- The Art of Computer Programming\\[2ex]
	Wegener -- theoretische Informatik, eine algorithmenorientierte Einführung\\[2ex]
	Downey -- Programmieren lernen mit Python
	
	\section{Einleitung}
	\begin{definition}
		Ein Algorithmus ist eine Rechenvorschrift, die
		\begin{itemize}
			\item die erwartete Eingabe genau spezifiziert
			\item die auf der Eingabe durchzuführenden Rechenschritte angibt
		\end{itemize}
	\end{definition}
	
	\subsection{Naiver Primzahltest}
	\begin{definition}
		Eine natürliche Zahl $n \geq 2$ heißt prim und wird Primzahl genannt, falls sie keine positiven Teiler außer $1$ und sich selbst besitzt.
	\end{definition}
	\begin{algorithm}
		\begin{algorithmic}
			\If{$n \leq 1$}
			
			antwort $\gets$ False
			\Else
			
			antwort $\gets$ True
			\EndIf
			
			\For{$i\gets 2$ to $n-1$}
			
			\If{($i$ ist Teiler von $n$)}
			
			\Return false
			\EndIf
			\EndFor
		\end{algorithmic}
	\end{algorithm}
	
	\newpage
	\subsection{Das Collatz-Problem}
	\begin{definition}
		Eine Collatz-Zahlenfolge
		\begin{itemize}[label=-]
			\item beginnt mit natürlicher Zahl $n>0$
			\item ist $n$ gerade, nimm als nächstes $n/2$
			\item ist $n$ ungerade, dann nimm $3n+1$
			\item wiederholt dies mit der erhaltenen Zahl
			\item stopp, falls die erhaltene Zahl $1$ ist.
		\end{itemize}
	\end{definition}
	Vermutung: Für jedes $n>0$ terminiert das Programm nach endlich vielen Schritten.
	
	\begin{lstlisting}[language=Python,captionpos=b,caption={Code für die Collatz-Conjecture}]
		while n > 1:
			print("%s, " n)
			if n % 2 == 0
				n = n/2
			else
				n = n*3+1
		print("1") \end{lstlisting}
	
	
	
	\subsection{Mathematische Notation}
	\subsubsection{Zahlenmengen}
	$\mathbb{N} = \{0,1,2,3,\dots\}$\\
	$\mathbb{Z} = \{\dots,-2,-1,0,1,2,\dots\}$\\
	$\mathbb{Q} = \{\frac{p}{q} | p \in \mathbb{Z}, q \in \mathbb{N}\backslash0\}$\\
	$\mathbb{R} = \{\text{reelle Zahlen}\}$\\
	$\rightarrow$ Darstellung z.B als Kommazahlen mit endlich oder unendlich vielen Nachkommastellen
	{\bfseries Beispiele:} $-24,83, 0.333\dots = \frac{1}{3}, \sqrt{2}, \pi = 3.141592, 0.9999\dots = 0.\overline{9} = 1$
\end{document}