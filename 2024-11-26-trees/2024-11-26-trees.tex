\documentclass[a4paper,12pt]{article}
\usepackage{amsmath}
\usepackage{amssymb}
\usepackage{amsthm}
\usepackage{marvosym}
\usepackage[utf8]{inputenc}
\usepackage[T1]{fontenc}
\usepackage[ngerman]{babel}
\usepackage[left=2.5cm,right=2cm,top=3cm,bottom=3cm]{geometry}
\usepackage{fancyhdr}
\usepackage{tabularx}
\usepackage{booktabs}
\usepackage{amsfonts}
\usepackage{tcolorbox}
\usepackage{adjustbox}
\usepackage{eurosym}
\usepackage[inline]{enumitem}
\usepackage{lastpage}
\usepackage{mathtools}
\usepackage{microtype}
\usepackage[ruled, linesnumbered, noline]{algorithm2e}
\usepackage{listings}
\usepackage{xcolor}

\setlength{\parindent}{0pt}

\usepackage{tikz}
\usetikzlibrary{decorations.pathmorphing,calc,shadows.blur,shadings,er,positioning}

\definecolor{codegreen}{rgb}{0,0.6,0}
\definecolor{codegray}{rgb}{0.5,0.5,0.5}
\definecolor{codepurple}{rgb}{0.58,0,0.82}
\definecolor{backcolour}{rgb}{0.95,0.95,0.92}

\lstdefinestyle{mystyle}{
	backgroundcolor=\color{backcolour},   
	commentstyle=\color{codegreen},
	keywordstyle=\color{magenta},
	numberstyle=\tiny\color{codegray},
	stringstyle=\color{codepurple},
	basicstyle=\ttfamily\footnotesize,
	breakatwhitespace=false,         
	breaklines=true,                 
	captionpos=b,                    
	keepspaces=true,                 
	numbers=left,                    
	numbersep=5pt,                  
	showspaces=false,                
	showstringspaces=false,
	showtabs=false,                  
	tabsize=2
}

\lstset{style=mystyle}
%\usepackage{pgfplots}
%\usepackage[leqno]{amsmath}

%\newcounter{mathseed}
%\setcounter{mathseed}{1}
%\pgfmathsetseed{\arabic{mathseed}} 
%\pgfdeclarelayer{background}
%\pgfsetlayers{background,main}

\newcommand{\RN}[1]{%
	\textup{\uppercase\expandafter{\romannumeral#1}}%
}

\newcommand{\obda}{o.B.d.A. }

\newenvironment{enum}{\begin{enumerate*}[label=\alph*),itemjoin=\hspace{2em}]}{\end{enumerate*}\\[2ex]}

\DeclareMathOperator{\N}{\mathbb N}
\DeclareMathOperator{\Q}{\mathbb Q}
\DeclareMathOperator{\R}{\mathbb R}
\DeclareMathOperator{\Z}{\mathbb Z}
\DeclareMathOperator{\C}{\mathbb C}
\DeclareMathOperator{\F}{\mathcal{F}}
\DeclareMathOperator{\E}{\mathcal{E}}
\DeclareMathOperator{\BigO}{\mathcal O}
\DeclareMathOperator{\mL}{\mathcal{L}}
\DeclareMathOperator{\mU}{\mathcal{U}}
\DeclareMathOperator{\ggt}{\text{ggT}}
\DeclareMathOperator{\kgv}{\text{kgV}}


%------ Defining multiple types of theorems ------%
\newtheorem{axiom}{Axiom}[section]
\newtheorem{theorem}[axiom]{Theorem}
\newtheorem{lemma}[axiom]{Lemma}
\newtheorem{satz}[axiom]{Satz}
\theoremstyle{definition}
\newtheorem*{example}{Beispiel}
\newtheorem*{bemerkung}{Bemerkung}
\newtheorem*{frage}{Frage}
\newtheorem*{loesung}{Lösung}
\newtheorem{definition}[axiom]{Definition}
\newtheorem{folg}[axiom]{Folgerung}
\newtheorem{notation}[axiom]{Notation}

%---- Changing forall and exists ----%
\let\oldforall\forall
\renewcommand{\forall}{\:\oldforall \, }
\let\oldexist\exists
\renewcommand{\exists}{\:\oldexist \: }
\newcommand\existu{\oldexist! \: }
\let\oldepsilon\epsilon
\renewcommand{\epsilon}{\varepsilon}
\newcommand{\corresponds}{\;\widehat{=}\;}

%------ Giving circled star ---------%
\makeatletter
\newcommand{\ostar}{\mathbin{\mathpalette\make@circled\star}}
\newcommand{\make@circled}[2]{%
	\ooalign{$\m@th#1\smallbigcirc{#1}$\cr\hidewidth$\m@th#1#2$\hidewidth\cr}%
}
\newcommand{\smallbigcirc}[1]{%
	\vcenter{\hbox{\scalebox{0.77778}{$\m@th#1\bigcirc$}}}%
}
\makeatother
%-----------------------------------%

%-------------- Vars ---------------%
\newcommand{\vldate}{26. November 2024}
\newcommand{\vlname}{Computerorientierte Mathematik I}
\newcommand{\vltopic}{Bäume, Wälder, Blätter}

\pagestyle{fancy}
\setlength{\headheight}{34pt}
\fancyhf{}
\fancyhead[L]{\includegraphics[width=1cm]{C:/Users/chole/Documents/tub/tublogo.png}}
\fancyhead[C]{{\bfseries\vlname}\\%
	\vltopic}
\fancyhead[R]{\vldate}
\fancyfoot[L]{}
\fancyfoot[C]{- \thepage\ of \pageref{LastPage} -}
\fancyfoot[R]{}

\renewcommand{\headrulewidth}{1pt}
\renewcommand{\footrulewidth}{1pt}
\renewcommand{\labelitemii}{-}

\begin{document}
	\setcounter{section}{6}
	\setcounter{subsection}{6}
	\subsection{Wälder, Bäume, Blätter}
	\begin{definition}
		Ein ungerichteter Graph ohne Kreise heißt Wald. Ein zusammenhängender Wald heißt Baum. Ein Knoten von Grad 1 in einem Baum heißt Blatt.
	\end{definition}
	\begin{satz}
		Sei $G$ ein Wald mit $n$ Knoten und $m$ Kanten, der aus $p$ Bäumen besteht. Dann ist $n = m + p$.
	\end{satz}
	\begin{proof}[Beweis durch vollständige Induktion über $m$.]
		Induktionsanfang $(m = 0)$: Klar.
		
		Induktionsschluss $(m - 1 \to m)$:
		Die Behauptung gelte für ein beliebiges, aber festes $m - 1 \in \N$. Betrachte einen Wald $G$ mit $m$ Kanten, $n$ Knoten und $p$ Bäumen. $G$ enthält einen Baum mit mindestens einer Kante und zwei Knoten. Betrachte ein Blatt $v$ und lösche die zu $v$ inzidente Kante $e$. Der Wald $G\setminus e$ hat $m - 1$ Kanten, $n$ Knoten und $p + 1$ Bäume. Nach Induktion gilt: $n = m - 1 + p + 1 = m + p$.
	\end{proof}
	\subsection{Charakterisierung von Bäumen}
	\begin{satz}
		Für ungerichtete Graphen $G$ sind folgende Aussagen äquivalent:
		\begin{enumerate}[label=(\roman*)]
			\item $G$ ist ein Baum.
			\item $G$ ist kreisfrei und hat $\lvert V(G) - 1\rvert$ Kanten.
			\item $G$ ist zusammenhängend und hat $\lvert V(G) - 1\rvert$ Kanten.
			\item $G$ ist minimal zusammenhängend, d. h. durch Löschen einer beliebigen Kante wird $G$ unzusammenhängend.
			\item $G$ ist maximal kreisfrei, d. h. bei Hinzufügen einer beliebigen Kante entsteht ein Kreis.
			\item Zwischen je zwei Knoten gibt es immer einen eindeutigen Weg.
		\end{enumerate}
	\end{satz}
	\begin{satz}
		Ein ungerichteter Graph ist genau dann zusammenhängend, wenn $G$ einen Baum als aufgespannten Teilgraphen enthält.
	\end{satz}
	\begin{proof}
		Wir zeigen: $(i) \implies (vi) \implies (iv) \implies (v) \implies (ii) \implies (iii) \implies (i)$.
		
		$(i) \implies (vi)$: Seien $v, w \in V$. Da $G$ zusammenhängend ist, gibt es einen Weg. 
		
		Annahme: Es gibt zwei $v$-$w$-Wege $P$ und $Q$ mit $E(P) \neq E(Q)$. Dann definiert $E(H) \coloneq E(P) \Delta E(Q)$ einen Eulerschen Teilgraphen $H$ von $G$. Also kann $E(H) \neq \emptyset$ in kantendisjunkte Kreise zerlegt werden (Lemma).
		
		Insbesondere gibt es Kreis in $H$ und damit auch $G$. Widerspruch!\\[2ex]
		$(vi) \implies (iv)$: $G$ ist zusammenhängend. Lösche $e \in E(G)$ mit $\Psi(e) = \{v, w\}$. Da $v, e, w$ eindeutiger $v$-$w$-Weg in $G$ war, gibt es keinen $v$-$w$-Weg mehr. Folglich ist $G \setminus e$ unzusammenhängend.\\[2ex]
		$(iv) \implies (v)$: $G$ ist kreisfrei: Löschen einer Kreiskante kann den Zusammenhang nicht zerstören, da die Endknoten der Kante durch den Weg verbunden bleiben. Da $G$ zusammenhängend ist, gibt es einen $v$-$w$-Weg für jedes Knotenpaar von $v$ nach $w$. Neue Kante $\Psi(e) = \{v, w\}$ bildet mit $v$-$w$-Weg einen Kreis.\\[2ex]
		$(v) \implies (ii)$: $G$ ist kreisfrei. Zeige: $G$ ist zusammenhängend. Seien $v, w$ zwei Knoten. Zeige: Es gibt einen $v$-$w$-Weg in $G$. Fügt man eine neue Kante $e$ zwischen $v$ und $w$ ein, so entsteht ein Kreis. Lösche man $e$ aus diesem Kreis, bleibt ein $v$-$w$-Weg in $G$ übrig. Aus dem letzten Satz folgt $\lvert V(G) \rvert = \lvert E(G)\rvert + 1$.\\[2ex]
		$(ii) \implies (iii)$: Folgt direkt aus dem letzten Satz.\\[2ex]
		$(iii) \implies (i)$: Sukzessives Löschen von $k$ Kreiskanten erhält den Zusammenhang und liefert einen Baum mit $\lvert V(G)\rvert - 1 - k$ Kanten. Wegen des letzten Satzes ist $k = 0$.
	\end{proof}
	\subsection{Charakterisierung zusammenhängender Graphen}
	\begin{definition}
		Sei $G = (V, E, \Psi)$ ein Graph und $X \subseteq V$.
		\begin{enumerate}[label=(\alph*)]
			\item Ist $G$ ungerichtet, so sei 
			\[
				\delta(X) \coloneq \{e \in E \mid \lvert \Psi(e) \cap X \rvert 1\}
			\]
			die Menge aller Kanten zwischen Knoten in $X$ und Kanten in $V \setminus X$.
			\item Ist $G$ gerichtet, so sei
			\begin{align*}
				\delta^+(X) &\coloneq \{e \in E \mid \Psi(e) \in X \times V \setminus X\}\\
				\delta^-(X) &\coloneq \{e \in E \mid \Psi(e) \in V \setminus X \times V\}\\
				\delta(X) &\coloneq \delta^+(X) \cup \delta^-(X)
			\end{align*}
		\end{enumerate}
	\end{definition}
	\begin{satz}
		Ein Graph $G$ ist genau dann zusammenhängend, wenn $\delta(X) \neq \emptyset$ für alle $\emptyset \subsetneq X \subsetneq V$. Ist $G$ gerichtet, ist $G$ genau dann stark zusammenhängend, wenn $\delta^+(X) \neq \emptyset$ für alle $\emptyset \subsetneq X \subsetneq V$.
	\end{satz}
	\begin{proof}
		\[
			G \text{ zusammenhängend } \Longleftrightarrow \delta(X) \neq \emptyset \text{ für alle } \emptyset \subsetneq X \subsetneq V
		\]
		$\implies:$ Es sei $\emptyset \subsetneq X \subsetneq V$ und $v \in X, w \in V \setminus X$.\\[2ex]
		Da $G$ zusammenhängend ist, gibt es einen $v$-$w$-Weg $P$ in $G$. Da $v \in X, w \notin X$, enthält $P$ Kante $e$ mit $\Psi(e) = \{x, y\}, x \in X, y \notin X$. Folglich ist $e \in \delta(X) \neq \emptyset$.\\[2ex]
		$\Longleftarrow:$ Widerspruchsannahme: $G$ sei unzusammenhängend. Dann gibt es Knoten $u, v$, die nicht durch einen Weg in $G$ verbunden sind. Es sei $X \coloneq \{w \in V \mid \exists u$-$v$-Weg in $G\}$, dann ist $u \in X$ und $v \notin X$. Behauptung: $\delta(X) = \emptyset$\\[2ex]
		$\implies$ Widerspruch! Denn gäbe es Kante $e \in \delta(X) und \Psi(e) = \{w, x\}, w \in X, x \notin X$, so gäbe es Weg von $u$ über $w$ und $e$ nach $x$. Widerspruch zu $x \notin X.$
	\end{proof}
	\subsection{Branchings und Arboreszenzen}
	\begin{definition}
		Ein gerichteter Graph $G$ ist ein Branching, falls der zu Grunde liegende ungerichtete Graph ein Wald ist und $\lvert \delta^-(v)\rvert \leq 1$ für alle Knoten $v$. Ein zusammenhängendes Branching heißt Arboreszenz.
	\end{definition}
	\begin{bemerkung}
		\begin{itemize}
			\item Eine Arboreszenz mit $n$ Knoten hat genau $n-1$ Kanten.
			\item Folglich existiert genau ein Knoten $r$ mit $\delta^-(r) = \emptyset$.
			\item Für eine Kante $(v, w)$ eines Branchings heißt $w$ Nachfolger von $v$ und $v$ Vorgänger von $w$.
			\item Ein Knoten ohne Nachfolger heißt Blatt.
		\end{itemize}
	\end{bemerkung}
	\begin{satz}
		Für gerichtete Graphen $G, r \in V(G)$ sind folgende Aussagen äquivalent:
		\begin{enumerate}[label=(\roman*)]
			\item $G$ ist eine Arboreszenz mit Wurzel $r$.
			\item $G$ ist ein Branching mit $\lvert V(G)\rvert - 1$ Kanten und $\delta^-(r) = \emptyset$.
			\item $G$ hat $\lvert V(G)\rvert - 1$ Kanten und jeder Knoten ist von $r$ aus erreichbar.
			\item Jeder Knoten ist von $r$ aus erreichbar. Entfernen einer beliebigen Kante zerstört diese Eigenschaft.
			\item $G$ erfüllt $\delta^+(X) \neq \emptyset$ für alle $X \subsetneq V$ mit $r \in X$. Entfernen einer beliebigen Kante zerstört diese Eigenschaft.
			\item $\delta^-(x) = \emptyset$ und $\forall v \in V(G)$ gibt es einen eindeutigen $r$-$v$-Kantenzug.
			\item $\delta^-(r) = \emptyset, \lvert \delta^-(v) = 1\rvert$ für jedes $v \in V \setminus \{r\}$ und $G$ ist kreisfrei.
		\end{enumerate}
	\end{satz}
	\begin{satz}
		Ein gerichteter Graph $G$ ist genau dann stark zusammenhängend, wenn er für jedes $r \in V(G)$ aufspannende Arboreszenz mit Wurzel $r$ enthält.
	\end{satz}
\end{document}