\documentclass[a4paper,12pt]{article}
\usepackage{amsmath}
\usepackage{amssymb}
\usepackage{amsthm}
\usepackage{marvosym}
\usepackage[utf8]{inputenc}
\usepackage[T1]{fontenc}
\usepackage[ngerman]{babel}
\usepackage[left=2.5cm,right=2cm,top=3cm,bottom=3cm]{geometry}
\usepackage{fancyhdr}
\usepackage{tabularx}
\usepackage{booktabs}
\usepackage{amsfonts}
\usepackage{tcolorbox}
\usepackage{adjustbox}
\usepackage{eurosym}
\usepackage[inline]{enumitem}
\usepackage{lastpage}
\usepackage{mathtools}
\usepackage{microtype}
\usepackage{algorithm}
\usepackage{algpseudocode}
\usepackage{listings}
\usepackage{xcolor}
%\usepackage{minted}
\setlength{\parindent}{0pt}

\usepackage{tikz}
\usetikzlibrary{decorations.pathmorphing,calc,shadows.blur,shadings,er,positioning}

\definecolor{codegreen}{rgb}{0,0.6,0}
\definecolor{codegray}{rgb}{0.5,0.5,0.5}
\definecolor{codepurple}{rgb}{0.58,0,0.82}
\definecolor{backcolour}{rgb}{0.95,0.95,0.92}

\lstdefinestyle{mystyle}{
	backgroundcolor=\color{backcolour},   
	commentstyle=\color{codegreen},
	keywordstyle=\color{magenta},
	numberstyle=\tiny\color{codegray},
	stringstyle=\color{codepurple},
	basicstyle=\ttfamily\footnotesize,
	breakatwhitespace=false,         
	breaklines=true,                 
	captionpos=b,                    
	keepspaces=true,                 
	numbers=left,                    
	numbersep=5pt,                  
	showspaces=false,                
	showstringspaces=false,
	showtabs=false,                  
	tabsize=2
}

\lstset{style=mystyle}
%\usepackage{pgfplots}
%\usepackage[leqno]{amsmath}

%\newcounter{mathseed}
%\setcounter{mathseed}{1}
%\pgfmathsetseed{\arabic{mathseed}} 
%\pgfdeclarelayer{background}
%\pgfsetlayers{background,main}

\newcommand{\RN}[1]{%
	\textup{\uppercase\expandafter{\romannumeral#1}}%
}

\newcommand{\obda}{o.B.d.A. }

\newenvironment{enum}{\begin{enumerate*}[label=\alph*),itemjoin=\hspace{2em}]}{\end{enumerate*}\\[2ex]}

\DeclareMathOperator{\N}{\mathbb N}
\DeclareMathOperator{\Q}{\mathbb Q}
\DeclareMathOperator{\R}{\mathbb R}
\DeclareMathOperator{\Z}{\mathbb Z}
\DeclareMathOperator{\BigO}{\mathcal O}


%------ Defining multiple types of theorems ------%
\newtheorem{axiom}{Axiom}[section]
\newtheorem{theorem}[axiom]{Theorem}
\newtheorem{lemma}[axiom]{Lemma}
\newtheorem{satz}[axiom]{Satz}
\theoremstyle{definition}
\newtheorem*{example}{Beispiel}
\newtheorem*{bemerkung}{Bemerkung}
\newtheorem*{frage}{Frage}
\newtheorem*{loesung}{Lösung}
\newtheorem{definition}[axiom]{Definition}
\newtheorem{folg}[axiom]{Folgerung}
\newtheorem{notation}[axiom]{Notation}

%---- Changing forall and exists ----%
\let\oldforall\forall
\renewcommand{\forall}{\:\oldforall \, }
\let\oldexist\exists
\renewcommand{\exists}{\:\oldexist \: }
\newcommand\existu{\oldexist! \: }
\let\oldepsilon\epsilon
\renewcommand{\epsilon}{\varepsilon}
\newcommand{\corresponds}{\;\widehat{=}\;}

%------ Giving circled star ---------%
\makeatletter
\newcommand{\ostar}{\mathbin{\mathpalette\make@circled\star}}
\newcommand{\make@circled}[2]{%
	\ooalign{$\m@th#1\smallbigcirc{#1}$\cr\hidewidth$\m@th#1#2$\hidewidth\cr}%
}
\newcommand{\smallbigcirc}[1]{%
	\vcenter{\hbox{\scalebox{0.77778}{$\m@th#1\bigcirc$}}}%
}
\makeatother
%-----------------------------------%

%-------------- Vars ---------------%
\newcommand{\vldate}{31. Oktober 2024}
\newcommand{\vlname}{Computerorientierte Mathematik I}
\newcommand{\vltopic}{Zahldarstellungen}

\pagestyle{fancy}
\setlength{\headheight}{34pt}
\fancyhf{}
\fancyhead[L]{\includegraphics[width=1cm]{C:/Users/chole/Documents/tub/tublogo.png}}
\fancyhead[C]{{\bfseries\vlname}\\%
	\vltopic}
\fancyhead[R]{\vldate}
\fancyfoot[L]{}
\fancyfoot[C]{- \thepage\ of \pageref{LastPage} -}
\fancyfoot[R]{}

\renewcommand{\headrulewidth}{1pt}
\renewcommand{\footrulewidth}{1pt}
\renewcommand{\labelitemii}{-}

\begin{document}
	\setcounter{section}{2}
	\section{Rechnen mit ganzen Zahlen}
	\subsection{Addition von Zahlen in $b$-adischer Darstellung}
	\begin{itemize}
		\item Wir verstehen laut Definition unter der Laufzeit eines Algorithmus die Zahl elementarer Rechenoperationen.
		\item Diese Definition liegt der Annahme zugrunde, dass jede dieser elementaren Operationen in konstant vielen Schritten ausgeführt wird.
		\item Diese Annahme ist nur dann realistisch, wenn diese Operationen auf Zahlen beschränkter Größe angewandt werden.
	\end{itemize}
	
	\begin{algorithm}
	\caption{Addition $b$-adischer Zahlen}
		\begin{algorithmic}
			\Require $x = (x_{k-1}\dots x_1x_0)_b, y = (y_{k-1}\dots y_1y_0)_b, b\geq 2$
			\State $c \gets 0$
			\For{$j \gets 0$ To $k-1$}
				\State $s_j \gets (x_j + y_j + c) \mod b$
				\State $c \gets \lfloor \frac{x_j + y_j + c}{b}\rfloor$
			\EndFor
			\State $s_k \gets c$
			\State \Return $(s_ks_{k-1}\dots s_1s_0)$
		\end{algorithmic}
	\end{algorithm}
	\begin{satz}
		Der Algorithmus ist korrekt und hat eine Laufzeitfunktion in $\BigO(k)$.
	\end{satz}
	\begin{proof}[Beweis per Induktion über $k$]
		Für jedes $k\geq 1$ arbeitet der Algorithmus korrekt; es gilt immer $c\in \{0,1\}$.\\[2ex]
		Induktionsanfang $(k = 1)$: $s_0 = (x_0 + y_0 + c) \mod b, s_1 = \lfloor \frac{x_0 + y_0 + c}{b} \rfloor = c$
		\[
			(s_1s_0)_b = \big\lfloor \frac{x_0 + y_0}{b}\big\rfloor \cdot b + (x_0 + y_0) \mod b = x_0 + y_0
		\]
		Außerdem gilt $0 \leq x + y_0 \leq 2b - 2$ und daher $c \in \{0,1\}$.\\[2ex]
		Induktionsschritt $(k \to k + 1)$:  Die Behauptung gelte für ein beliebiges, aber festes $k \geq 1$. Dann ist 
		\[
			(x_{k-1}\dots x_0)_b + (y_{k-1}\dots y_0)_b = (cs_{k-1}\dots s_0)_b
		\]
		also gilt für $x = (x_kx_{k-1}\dots x_0)$ und $y = (y_ky_{k-1}\dots y_0)$:
		\begin{align*}
			x + y &= x_kb^k + y_kb^k + (x_{k-1}\dots x_0)_b + (y_{k-1}\dots y_0)\\
			&= (x_k + y_k + c)b^k + (s_{k-1}\dots s_0)_b\\
			&= (s_{k+1}s_k)b^k + (s_{k-1}\dots s_0)\\
			&= (s_{k+1}s_k\dots s_0)_b
		\end{align*}
		und $c' = \lfloor \frac{x_k + y_k + c}{b} \rfloor \in \{0,1\}$, da $0 \leq x_k + y_k + c \leq 2b-2 + 1 < 2b$
	\end{proof}
	\begin{bemerkung}
		\begin{itemize}
			\item Der Algorithmus kann mittels der allgemeinen Komplementdarstellung auch zur Addition ganzer Zahlen verwendet werden.
			\item Insbesondere kann man den Algorithmus auch zur Subtraktion nutzen, indem man $f(-x) = K_b^\ell(x)$ verwendet.
			\item Auch $f(-x)$ kann in Laufzeit von $\BigO(k)$ berechnet werden, sodass die Subtraktion ebenfalls Laufzeit von $\BigO(k)$ hat.
		\end{itemize}
	\end{bemerkung}
	\subsection{Multiplikation ganzer Zahlen in $b$-adischer Darstellung}
	Die naive schriftliche Multiplikation zweier $k$-stelliger Zahlen hat Laufzeit $\Theta(k^2)$.
	\[
		\bigg(\sum_{j=0}^{k-1}x_jb^j\bigg)\bigg(\sum_{j=0}^{k-1}y_jb^j\bigg)
	\]
	\begin{example}
		Betrachte $x = (x_1x_0)_b$ und $y = (y_1y_0)_b$ für $b \geq 2$ und berechne
		\[
			x \cdot y = \underbrace{(x_1y_1)}_{:=p}b^2 + \underbrace{(x_1y_0 + y_1x_0)}_{r-p-q}b + \underbrace{(x_0y_0)}_{:=q}
		\]
		Rechenschritte: $4$ Multiplikationen und $3$ Additionen.\\[2ex]
		Alternative Idee: Berechne $p := x_1y_1, q = x_0y_0, r = (x_1 + x_0) \cdot (y_1 + y_0)$.
		\[
			x \cdot y = pb^2 + (r-p-q)b + q
		\]
		Rechenschritte: $3$ Multiplikationen, $6$ Additionen
		
		Es seien $x = (x_3x_2x_1x_0), y = (y_3y_2y_1y_0)$. Berechne
		\begin{align*}
			p &:= (x_3x_2)_b \cdot (y_3y_2)_b\\
			q &:= (x_1x_0)_b \cdot (y_1y_0)_b\\
			r &:= \big((x_3x_2)_b + (x_1x_0)_b\big) \cdot \big((y_3y_2)_b + (y_1y_0)_b\big)
		\end{align*}
		und verwende dabei die alternative Idee. Rechenschritte: $9$ Multiplikationen und $3\cdot 6 + 2$ Additionen.\\[2ex]
		Dann ist 
		\[
			x\cdot y = pb^4 + (r-p-q)b^2 + q
		\]
		Rechenschritte gesamt: $9$ Multiplikationen, $24$ Additionen.
	\end{example}
	\subsection{Karazubas Multiplikation}
	\begin{algorithm}[H]
		\caption{Karazubas Multiplikation}
		\begin{algorithmic}
			\Require $x, y \in \N$ in Binärdarstellung
			\If{$x < 8$ and $y < 8$}
				\Return $x \cdot y$
			\Else
				\State $\ell \gets 1 + \lfloor \log_2(\max\{x,y\})\rfloor$
				\State $k \gets \lfloor \frac{\ell}{2}\rfloor, b \gets 2^k$
				\State $x' \gets \lfloor \frac{x}{b} \rfloor, x'' \gets x \mod b$
				\State $y' \gets \lfloor \frac{y}{b} \rfloor, y'' \gets y \mod b$
				\State $p \gets x' \cdot y'$
				\State $q \gets x'' \cdot y''$
				\State $r \gets (x' + x'') \cdot (y' + y'')$
				\Return $pb^2 + (r-p-q)b + q$
			\EndIf
		\end{algorithmic}
	\end{algorithm}
	\begin{example}
		$x = (1100011)_2, y = (0010110)_2$\\[2ex]
		$\ell := 1 + \lfloor \log_2(x)\rfloor = 7, k = \lfloor \frac{\ell}{2} \rfloor, b = 2^k = 8$
		\begin{align*}
			p & \coloneq x' \cdot y' = (1100)_2 \cdot (0010)_2 = (11000)_2\\
			q & \coloneq x'' \cdot y'' = (011)_2 \cdot (110)_2 = (10010)_2\\
			r & \coloneq (x' + x'') \cdot (y' + y'') = (1111)_2 + (1000)_2 = (1111000)_2\\
			pb^2 + (r-p-q)b + q &= (11000)_22^6 + (1001110)_22^3 + (10010)_2\\
			&= (11000000000)_2 + (1001110000)_2 + (10010)_2\\
			&= (100010000010)_2
		\end{align*}
	\end{example}
	\subsection{Karazubas Multiplikation -- Korrektheit und Laufzeit}
	\begin{satz}
		Der Algorithmus ist korrekt und hat eine Laufzeitfunktion in $\BigO(\ell^{\log_2(3)})$ mit $\ell \coloneq 1 + \lfloor \log_2(\max\{x,y\})\rfloor$
	\end{satz}
	\begin{bemerkung}
		\begin{itemize}
			\item $\log_2(3) < 1.59; \ell = 1~000 \Longrightarrow \ell^2 = 1~000~000, \ell^{\log_2(3)} < 60~000$
			\item Es gibt schnellere Verfahren (z. B. Schönhage-Strassen-Algorithmus)
			\item Ganzzahlige Division kann auf Multiplikation zurückgeführt werden.
		\end{itemize}
	\end{bemerkung}
	\begin{proof}[Laufzeitanalyse]
		Es sei $T(\ell)$ maximale Laufzeit für Imputzahlen mit $\leq \ell$ Ziffern.
		\[T(\ell) \leq 
		\begin{dcases}
			c & \text{ für } \ell \leq 3,\\
			c \cdot \ell + 3T\Big(\Big\lceil \frac{\ell}{2}\Big\rceil\Big) + 1 & \text{ für } \ell \geq 4
		\end{dcases}
		\]
	\end{proof}
\end{document}