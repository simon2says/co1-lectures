\documentclass[a4paper,12pt]{article}
\usepackage{amsmath}
\usepackage{amssymb}
\usepackage{amsthm}
\usepackage{marvosym}
\usepackage[utf8]{inputenc}
\usepackage[T1]{fontenc}
\usepackage[ngerman]{babel}
\usepackage[left=2.5cm,right=2cm,top=3cm,bottom=3cm]{geometry}
\usepackage{fancyhdr}
\usepackage{tabularx}
\usepackage{booktabs}
\usepackage{amsfonts}
\usepackage{tcolorbox}
\usepackage{adjustbox}
\usepackage{eurosym}
\usepackage[inline]{enumitem}
\usepackage{lastpage}
\usepackage{mathtools}
\usepackage{microtype}
\usepackage{algorithm}
\usepackage{algpseudocode}
\usepackage{listings}
\usepackage{xcolor}
%\usepackage{minted}
\setlength{\parindent}{0pt}

\usepackage{tikz}
\usetikzlibrary{decorations.pathmorphing,calc,shadows.blur,shadings,er,positioning}

\definecolor{codegreen}{rgb}{0,0.6,0}
\definecolor{codegray}{rgb}{0.5,0.5,0.5}
\definecolor{codepurple}{rgb}{0.58,0,0.82}
\definecolor{backcolour}{rgb}{0.95,0.95,0.92}

\lstdefinestyle{mystyle}{
	backgroundcolor=\color{backcolour},   
	commentstyle=\color{codegreen},
	keywordstyle=\color{magenta},
	numberstyle=\tiny\color{codegray},
	stringstyle=\color{codepurple},
	basicstyle=\ttfamily\footnotesize,
	breakatwhitespace=false,         
	breaklines=true,                 
	captionpos=b,                    
	keepspaces=true,                 
	numbers=left,                    
	numbersep=5pt,                  
	showspaces=false,                
	showstringspaces=false,
	showtabs=false,                  
	tabsize=2
}

\lstset{style=mystyle}
%\usepackage{pgfplots}
%\usepackage[leqno]{amsmath}

%\newcounter{mathseed}
%\setcounter{mathseed}{1}
%\pgfmathsetseed{\arabic{mathseed}} 
%\pgfdeclarelayer{background}
%\pgfsetlayers{background,main}

\newcommand{\RN}[1]{%
	\textup{\uppercase\expandafter{\romannumeral#1}}%
}

\newcommand{\obda}{o.B.d.A. }

\newenvironment{enum}{\begin{enumerate*}[label=\alph*),itemjoin=\hspace{2em}]}{\end{enumerate*}\\[2ex]}

\DeclareMathOperator{\N}{\mathbb N}
\DeclareMathOperator{\Q}{\mathbb Q}
\DeclareMathOperator{\R}{\mathbb R}
\DeclareMathOperator{\Z}{\mathbb Z}
\DeclareMathOperator{\BigO}{\mathcal O}


%------ Defining multiple types of theorems ------%
\newtheorem{axiom}{Axiom}[section]
\newtheorem{theorem}[axiom]{Theorem}
\newtheorem{lemma}[axiom]{Lemma}
\newtheorem{satz}[axiom]{Satz}
\theoremstyle{definition}
\newtheorem*{example}{Beispiel}
\newtheorem*{bemerkung}{Bemerkung}
\newtheorem*{frage}{Frage}
\newtheorem*{loesung}{Lösung}
\newtheorem{definition}[axiom]{Definition}
\newtheorem{folg}[axiom]{Folgerung}
\newtheorem{notation}[axiom]{Notation}

%---- Changing forall and exists ----%
\let\oldforall\forall
\renewcommand{\forall}{\:\oldforall \, }
\let\oldexist\exists
\renewcommand{\exists}{\:\oldexist \: }
\newcommand\existu{\oldexist! \: }
\let\oldepsilon\epsilon
\renewcommand{\epsilon}{\varepsilon}
\newcommand{\corresponds}{\;\widehat{=}\;}

%------ Giving circled star ---------%
\makeatletter
\newcommand{\ostar}{\mathbin{\mathpalette\make@circled\star}}
\newcommand{\make@circled}[2]{%
	\ooalign{$\m@th#1\smallbigcirc{#1}$\cr\hidewidth$\m@th#1#2$\hidewidth\cr}%
}
\newcommand{\smallbigcirc}[1]{%
	\vcenter{\hbox{\scalebox{0.77778}{$\m@th#1\bigcirc$}}}%
}
\makeatother
%-----------------------------------%

%-------------- Vars ---------------%
\newcommand{\vldate}{29. Oktober 2024}
\newcommand{\vlname}{Computerorientierte Mathematik I}
\newcommand{\vltopic}{Zahldarstellungen}

\pagestyle{fancy}
\setlength{\headheight}{34pt}
\fancyhf{}
\fancyhead[L]{\includegraphics[width=1cm]{C:/Users/chole/Documents/tub/tublogo.png}}
\fancyhead[C]{{\bfseries\vlname}\\%
	\vltopic}
\fancyhead[R]{\vldate}
\fancyfoot[L]{}
\fancyfoot[C]{- \thepage\ of \pageref{LastPage} -}
\fancyfoot[R]{}

\renewcommand{\headrulewidth}{1pt}
\renewcommand{\footrulewidth}{1pt}
\renewcommand{\labelitemii}{-}

\begin{document}
	\setcounter{section}{2}
	\setcounter{subsection}{2}
	\subsection{Darstellung ganzer Zahlen -- Horner-Schema}
	\begin{frage}
		Wie berechnet man den Wert einer in $b$-adischen Darstellung gegebenen Zahl möglichst effizient?
	\end{frage}
	\begin{loesung}
		Horner-Schema:
		\[
			\underbrace{\sum_{i = 0}}_{(l-1)}^{\ell - 1}z_i \overbrace{\cdot}^{(l-1)} \underbrace{b^i}_{(l-2)} = z_0 + b(z_1 + b(z_2 + \ldots b (z_{\ell - 2} + b\cdot z_{\ell - 1})))
		\]
	\end{loesung}
	\begin{bemerkung}
		Zur Berechnung des Ausdrucks auf der linken Seite benötigt man
		\begin{align*}
			\ell - 1\hspace*{7ex}&\text{Additionen}\\
			2\ell - 3\hspace*{7ex}&\text{Multiplikationen}\\
			\intertext{Zur Berechnung des Ausdrucks auf der rechten Seite benötigt man}
			\ell - 1\hspace*{7ex}& \text{Additionen}\\
			\ell - 1\hspace*{7ex}& \text{Multiplikationen}
		\end{align*}
	\end{bemerkung}
	Codierung negativer ganzer Zahlen
	
	Naheliegend: Darstellung mit explizitem Vorzeichen: $(0 \corresponds  +, 1 \corresponds -)$
	\begin{center}
		\begin{tabular}{l l l l}
			$0$\hspace*{2ex} &$0000$\hspace*{5ex} &$-0$\hspace*{2ex} &$1000$\\
			$1$ &$0001$ & $-1$ & $1001$\\
			$2$ &$0010$ & $-2$ & $1010$\\
			$3$ &$0011$ & $-3$ & $1011$\\
			$4$ &$0100$ & $-4$ & $1100$\\
			$5$ &$0101$ & $-5$ & $1101$\\
			$6$ &$0110$ & $-6$ & $1110$\\
			$7$ &$0111$ & $-7$ & $1111$
		\end{tabular}
	\end{center}
	Nachteile: \begin{itemize}
		\item Null hat zwei verschiedene Darstellungen
		\item Addition kann nicht unmittelbar auf Addition von Binärzahlen ohne Vorzeichen zurückgeführt werden.
	\end{itemize}
	
	Binäre Komplementdarstellung
	\begin{itemize}
		\item zunächst wie oben: führendes Bit zeigt Vorzeichen $(0 \corresponds +, 1 \corresponds -)$item aber: negative 
		\item aber: negative Zahlen werden komplementär dargestellt.
		\item $z \in \{-2^{\ell - 1}, -2^{\ell - 1} + 1, \ldots, -1\}$ wird dargestellt als $z + 2^\ell$
		\begin{center}
			\begin{tabular}{l l l l}
				$1$ \hspace*{2ex} & $0001$ \hspace*{5ex} & $-1$ \hspace*{2ex} & $1111$\\
				$2$ & $0010$ & $-2$ & $1110$ \\
				$3$ & $0011$ & $-3$ & $1101$ \\
				$4$ & $0100$ & $-4$ & $1100$ \\
				$5$ & $0101$ & $-5$ & $1011$ \\
				$6$ & $0110$ & $-6$ & $1010$ \\
				$7$ & $0111$ & $-7$ & $1001$ \\
				$0$ & $0000$ & $-8$ & $1000$
			\end{tabular}
		\end{center}
		Beachte: $\mod 2^\ell$ sind $z$ und $z + 2^\ell$ gleich.
	\end{itemize}
	\subsection{Das $b$-Komplement}
	\begin{definition}
		Für $\ell \in \N_{\geq 1}, b \in \N_{\geq 2}$ und $n \in \{0,1,\ldots, b^\ell - 1\}$ ist
		\[
			K_b^\ell(n) := -n \mod b^\ell
		\]
		das $\ell$-stellige $b$-Komplement von $n$.
	\end{definition}
	\begin{example}
		\begin{itemize}
			\item $n = 15, b = 10, \ell = 2$
		\[
			K_{10}^2(15) = -15 \mod 10^2 = 85
		\]
		\item $b = 2, \ell = 4$
		\[
			K_2^4(15) = -15 \mod 2^4 = 1
		\]
		\end{itemize}
	\end{example}
	\begin{lemma}
		Für $\ell \geq 1, b \geq 2$ und $n = \sum_{i = 0}^{\ell - 1}z_ib^i$ mit $z_i \in \{0,1,\ldots, b-1\}$ gilt:
		\begin{enumerate}[label=(\roman*)]
			\item \[
				K_b^\ell(n + 1) = \sum_{i = 0}^{\ell - 1}(b - 1 - z_i)b^i
			\]
			für $n + b^\ell - 1$; außerdem $K_b^\ell(0) = 0$
			\item \[
				K_b^\ell(K_b^\ell(n)) = n
			\]
		\end{enumerate}
	\end{lemma}
	\begin{proof}
		\begin{enumerate}[label=(\roman*)]
			\item \[
				K_b^\ell(0) = -0 \mod b^\ell = 0.
			\]
			Für $n \in \{0,\ldots, b^\ell - 2\}$ gilt
			\begin{align*}
				K_b^\ell(n + 1) &= -(n+1) \mod b^\ell\\
				&= b^\ell - 1 - n\\
				&= \sum_{i = 0}^{\ell - 1}(b-1)b^i - \sum_{i = 0}^{\ell - 1}z_i b^i\\
				&= \sum_{i = 0}^{\ell - 1}(b - 1 - z_i)b^i
			\end{align*}
			\item Wegen Rechenregeln für die modulo-Rechnung gilt
			\[
				K_b^\ell(K_b^\ell(n)) = (-(-n \mod b^\ell)) \mod b^\ell = n \mod b^\ell = n
			\]
		\end{enumerate}
	\end{proof}
	\begin{bemerkung}
		Berechne also $b$-Komplement von $n > 0$ wie folgt: nimm $b$-adische Darstellung von $n-1$ und bilde {\itshape stellenweise} Differenz zu $b-1$.
	\end{bemerkung}
	\subsection{$b$-Komplementdarstellung ganzer Zahlen}
	\begin{definition}
		Es seien $\ell \geq 1, b \geq 2, n \in \{- \lfloor \frac{b^\ell}{2} \rfloor, \ldots, \lceil \frac{b^\ell}{2}\rceil - 1\}$. Die $\ell$-stellige Komplementdarstellung von $n$ ist die $b$-adische Darstellung von $n$ (falls $n\geq 0$) bzw. von $K_b^l(-n)$ (falls $n < 0$), vorne mit Nullen zu einer $\ell$-stelligen Zahl aufgefüllt.
	\end{definition}
	\begin{example}
		$b = 2, \ell = 4$
		\begin{itemize}
			\item Die $4$-stellige $2$-Komplementdarstellung von $n = 5$ ist $0101$.
			\item Die $4$-stellige $2$-Komplementdarstellung von $n = -5$ ist $K_2^4(5) = 1011$.
		\end{itemize}
	\end{example}
	\begin{satz}
		Für $\ell \geq 1, b \geq 2, Z = \{-\lfloor \frac{b^\ell}{2} \rfloor, \ldots, \lceil \frac{b^\ell}{2}\rceil - 1\}$ sei
		\[
			f: Z \to \{0,1,\ldots, b^\ell - 1\}, n \to \begin{dcases}
				n &\text{ falls } \ell \geq 0\\
				K_b^\ell(-n)&\text{ falls } \ell < 0
			\end{dcases}
		\]
	\end{satz}
	das heißt, $f(n) = n \mod b^\ell$. Dann ist $f$ bijektiv und für $x,y \in Z$ gilt:
	\begin{enumerate}[label=(\roman*)]
		\item ist $x + y \in Z$, so gilt $f(x + y) = \big(f(x) + f(y)\big) \mod b^\ell$
		\item ist $x \cdot y \in Z$, so gilt $f(x \cdot y) = \big(f(x) \cdot f(y)\big) \mod b^\ell$
	\end{enumerate}
	\begin{proof}
		Für $n \in Z = \{-\lfloor \frac{b^\ell}{2}\rfloor, \ldots, \lceil \frac{b^\ell}{2} - 1\rceil\}$ ist $f(n) = n \mod b^\ell$. Da $\lvert Z \rvert = b^\ell$, ist $f$ bijektiv.\\[2ex]
		(i) und (ii) folgen aus Rechenregeln für modulo-Rechnung.
	\end{proof}
	\begin{bemerkung}
		\begin{itemize}
			\item Mit der $b$-Komplementdarstellung kann man also ohne Fallunterscheidung rechnen.
			\item Ignoriert man Über- und Unterschreitung des zulässigen Bereichs $Z$, so rechnet man mit Komplementdarstellung genau wie mit $b$-adischer Darstellung.
			\item Für $b = 2$ und $x \in Z$ gilt: $x \geq 0 \Longleftrightarrow$ Führendes Bit ist $0$.
		\end{itemize}
	\end{bemerkung}
	\subsection{Darstellung ganzer Zahlen im Computer}
	Zahlen werden in Speicherblöcken zu je $8$ Bits (\glqq Bytes\grqq) gespeichert. 
	\subsection{Darstellung großer ganzer Zahlen im Computer}
	\begin{frage}
		Wie kann ich Zahlen darstellen, die nicht in einen Speicherblock passen?
	\end{frage}
	\begin{loesung}
		\begin{itemize}
			\item Nutze $k \geq 2$ Blöcke zur Darstellung größerer Zahlen $n$.
			\item Stelle $n$ bzw $f(n)$ in $b$-adischer Darstellung mit Basis $b^\ell$ ($\ell =$ Anzahl Bits eines Speicherblocks)
			\item Damit kann jede Zahl 
			\[
				n \in Z = \bigg\{-\bigg\lfloor \frac{b^\ell}{2}\bigg\rfloor, \ldots, \bigg\lceil \frac{b^\ell}{2}\bigg\rceil - 1\bigg\}
			\]
			als $f(n) = n \mod b^k$ dargestellt werden.
		\end{itemize}
	\end{loesung}
	\begin{example}
		$b = 2^8, n = -11215$. Wähle $k = 2, denn:$
		\[
			-\bigg\lfloor \frac{b^2}{2} \bigg\rfloor = -32768 \leq n \leq 32767 = \bigg\lceil \frac{b^2}{2} \bigg\rceil - 1
		\]
		Dann ist
		\begin{align*}
			f(n) &= n \mod b^2 = 54321\\
			&= (212~49)_b\\
			&= (11010100~00110001)_2
		\end{align*}
		Darstellung im Hauptspeicher ist umgekehrt:
		\[00110001~11010100\]
	\end{example}
	\subsection{Exakte Arithmetik}
	Für jede Zahl $n \in \Z$ gibt es ein geeignetes $k \geq 1$, so dass 
	\[
		n \in Z_k :=\bigg\{-\bigg\lfloor \frac{b^k}{2}\bigg\rfloor, \ldots, \bigg\lceil \frac{b^k}{2} \bigg\rceil- 1\bigg\}
	\]
	\begin{frage}
		Ist $n$ eine Variable, wie kann man $k$ dynamisch anpassen?
	\end{frage}
	\begin{loesung}
		Es seien $n \in \Z$ und $k, k' \geq 1$, so dass $n \in Z_k \cap Z_{k'}$. Weiter seien
		\[
			f_k(n) = n \mod b^k \text{ und } f_{k'}(n) = n \mod b^{k'}.
		\]
		Eine Darstellung von $n$ in $k$ Blöcken kann dann einfach umgerechnet werden in eine Darstellung in $k'$ Blöcken.\\[2ex]
		1. Fall: Für $n \geq 0$ ist $f_{k'}(n) = f_k(n) = n$, also Voranstellen (bzw. Löschen) von $k' - k$ Null-Blöcken.\\[2ex]
		2. Fall: Für $n < 0$ ist $f_{k'}(n) = f_k(n) - b^k + b^{k'}$, also Voranstellen (bzw. Löschen) von $k'-k$ Eins-Blöcken.
		\begin{example}
			$b = 2^8 = 256$
			\begin{align*}
				Z_1 &= \{-128, \ldots, 127\}\\
				Z_2 &= \{-32768, \ldots, 32767\}
			\end{align*}
			Beispiel 1: $n = 65, f_1(n) = f_2(n) = 65$.
			\[
				\underbrace{01000001}_{k = 1} \longleftrightarrow \underbrace{00000000~01000001}_{k=2}
			\]
			Beispiel 2: $n = -63, f_1(n) = 193, f_2(n) = 65473$
			\[
				\underbrace{11000001}_{k = 1} \longleftrightarrow \underbrace{11111111~11000001}_{k=2}
			\]
		\end{example}
	\end{loesung}
\end{document}