\documentclass[a4paper,12pt]{article}
\usepackage{amsmath}
\usepackage{amssymb}
\usepackage{amsthm}
\usepackage{marvosym}
\usepackage[utf8]{inputenc}
\usepackage[T1]{fontenc}
\usepackage[ngerman]{babel}
\usepackage[left=2.5cm,right=2cm,top=3cm,bottom=3cm]{geometry}
\usepackage[dvipsnames]{xcolor}
\usepackage{fancyhdr}
\usepackage{tabularx}
\usepackage{booktabs}
\usepackage{amsfonts}
\usepackage{tcolorbox}
\usepackage{adjustbox}
\usepackage{eurosym}
\usepackage[inline]{enumitem}
\usepackage{lastpage}
\usepackage{mathtools}
\usepackage{microtype}
\usepackage[ruled, linesnumbered, noline]{algorithm2e}
\usepackage{listings}


\setlength{\parindent}{0pt}

\usepackage{tikz}
\usetikzlibrary{decorations.pathmorphing,calc,shadows.blur,shadings,er,positioning}

\definecolor{codegreen}{rgb}{0,0.6,0}
\definecolor{codegray}{rgb}{0.5,0.5,0.5}
\definecolor{codepurple}{rgb}{0.58,0,0.82}
\definecolor{backcolour}{rgb}{0.95,0.95,0.92}

\lstdefinestyle{mystyle}{
	backgroundcolor=\color{backcolour},   
	commentstyle=\color{magenta},
	keywordstyle=\color{NavyBlue},
	numberstyle=\tiny\color{codegray},
	stringstyle=\color{codepurple},
	basicstyle=\ttfamily\footnotesize,
	breakatwhitespace=false,         
	breaklines=true,                 
	captionpos=b,                    
	keepspaces=true,                 
	numbers=left,                    
	numbersep=5pt,                  
	showspaces=false,                
	showstringspaces=false,
	showtabs=false,                  
	tabsize=2
}

\lstset{style=mystyle}
%\usepackage{pgfplots}
%\usepackage[leqno]{amsmath}

%\newcounter{mathseed}
%\setcounter{mathseed}{1}
%\pgfmathsetseed{\arabic{mathseed}} 
%\pgfdeclarelayer{background}
%\pgfsetlayers{background,main}

\newcommand{\RN}[1]{%
	\textup{\uppercase\expandafter{\romannumeral#1}}%
}

\newcommand{\obda}{o.B.d.A. }

\newenvironment{enum}{\begin{enumerate*}[label=\alph*),itemjoin=\hspace{2em}]}{\end{enumerate*}\\[2ex]}

\DeclareMathOperator{\N}{\mathbb N}
\DeclareMathOperator{\Q}{\mathbb Q}
\DeclareMathOperator{\R}{\mathbb R}
\DeclareMathOperator{\Z}{\mathbb Z}
\DeclareMathOperator{\C}{\mathbb C}
\DeclareMathOperator{\F}{\mathcal{F}}
\DeclareMathOperator{\E}{\mathcal{E}}
\DeclareMathOperator{\BigO}{\mathcal O}
\DeclareMathOperator{\mL}{\mathcal{L}}
\DeclareMathOperator{\mU}{\mathcal{U}}
\DeclareMathOperator{\ggt}{\text{ggT}}
\DeclareMathOperator{\kgv}{\text{kgV}}


%------ Defining multiple types of theorems ------%
\newtheorem{axiom}{Axiom}[section]
\newtheorem{theorem}[axiom]{Theorem}
\newtheorem{lemma}[axiom]{Lemma}
\newtheorem{satz}[axiom]{Satz}
\theoremstyle{definition}
\newtheorem*{example}{Beispiel}
\newtheorem*{bemerkung}{Bemerkung}
\newtheorem*{frage}{Frage}
\newtheorem*{loesung}{Lösung}
\newtheorem*{behauptung}{Behauptung}
\newtheorem*{beobachtung}{Beobachtung}
\newtheorem{definition}[axiom]{Definition}
\newtheorem{folg}[axiom]{Folgerung}
\newtheorem{notation}[axiom]{Notation}

%---- Changing forall and exists ----%
\let\oldforall\forall
\renewcommand{\forall}{\:\oldforall \, }
\let\oldexist\exists
\renewcommand{\exists}{\:\oldexist \: }
\newcommand\existu{\oldexist! \: }
\let\oldepsilon\epsilon
\renewcommand{\epsilon}{\varepsilon}
\newcommand{\corresponds}{\;\widehat{=}\;}

%------ Giving circled star ---------%
\makeatletter
\newcommand{\ostar}{\mathbin{\mathpalette\make@circled\star}}
\newcommand{\make@circled}[2]{%
	\ooalign{$\m@th#1\smallbigcirc{#1}$\cr\hidewidth$\m@th#1#2$\hidewidth\cr}%
}
\newcommand{\smallbigcirc}[1]{%
	\vcenter{\hbox{\scalebox{0.77778}{$\m@th#1\bigcirc$}}}%
}
\makeatother
%-----------------------------------%

%-------------- Vars ---------------%
\newcommand{\vldate}{12. Dezember 2024}
\newcommand{\vlname}{Computerorientierte Mathematik I}
\newcommand{\vltopic}{Laufzeitbeschleunigungssatz}

\pagestyle{fancy}
\setlength{\headheight}{34pt}
\fancyhf{}
\fancyhead[L]{\includegraphics[width=1cm]{C:/Users/chole/Documents/tub/tublogo.png}}
\fancyhead[C]{{\bfseries\vlname}\\%
	\vltopic}
\fancyhead[R]{\vldate}
\fancyfoot[L]{}
\fancyfoot[C]{- \thepage\ of \pageref{LastPage} -}
\fancyfoot[R]{}

\renewcommand{\headrulewidth}{1pt}
\renewcommand{\footrulewidth}{1pt}
\renewcommand{\labelitemii}{-}

\begin{document}
	\setcounter{section}{8}
	\setcounter{subsection}{9}
	\begin{bemerkung}
		\begin{itemize}
			\item QuickSort gehört zu den empirisch schnellsten Sortierverfahren.
			\item Es beruht -- wie MergeSort -- auf Divide \& Conquer
			\item Im Gegensatz zu MergeSort ist der Divide-Schritt aufwendiger, der Conquer-Schritt dafür einfacher.
		\end{itemize}
	\end{bemerkung}
	\subsection{Laufzeitanalyse von QuickSort}
	\begin{satz}
		QuickSort arbeitet korrekt. Die Laufzeitfunktion ist in $\Theta(n^2)$ für $n = \lvert S\rvert$.
	\end{satz}
	\begin{proof}[Laufzeitbeweis]
		Betrachte Rekursionsbaum. Knoten eines Levels repräsentieren disjunkte Teilmengen von $S$, da $S_1$ und $S_2$ in Schritt $5$ disjunkt sind. Aufspalten (Divide) einer Teilmenge $S'$ in zwei disjunkte Teilmengen in Schritt $5$ benötigt $\lvert S'\rvert - 1$ Vergleiche und Laufzeit $\BigO(\lvert S'\rvert)$. Alle auf einem Level durchgeführten Operationen benötigen Laufzeit $\BigO(n)$. Da rekursive Aufrufe für echt kleinere Teilmengen erfolgen, ist die Rekursionstiefe höchstens $n$. Daraus folgt die Laufzeitfunktion in $\BigO(n^2)$. 
	\end{proof}
	Untere Schranke: Wird als Pivot-Element immer das kleinste (oder größte) Element gewählt, so ist die gesamte Anzahl an Vergleichen
	\[
		\sum_{i = 1}^{n - 1} i = \frac{n (n - 1)}{2} \in \BigO(n^2)
	\]
	\begin{satz}
		Wird in QuickSort als Pivot-Element der Median von $S$ gewählt, so ist die Rekursionstiefe nur $\BigO(\log n)$ für $n = \lvert S\rvert$.
	\end{satz}
	\begin{proof}
		Die Größe der Teilmenge halbiert sich mit jedem Level des Rekursionsbaums. Daher gibt es nur $\BigO(\log n)$ Level.
	\end{proof}
	Problem: Wie kann der Median von $S$ effizient berechnet werden?
	\begin{itemize}
		\item Später: Berechnung des Medians in $\BigO(n)$
		\item Es gibt also theoretische QuickSort-Variante mit Laufzeit $\Theta(n \log n)$
		\item Diese Variante ist jedoch empirisch deutlich ineffizienter
	\end{itemize}
	\begin{satz}
		Wählt man das Pivot-Element zufällig und gleichverteilt aus $S$, so ist die erwartete Laufzeit von QuickSort in $\BigO(n \log n)$ für $n = \lvert S\rvert$.
	\end{satz}
	\begin{proof}
		Es sei $f(n)$ Erwartungswert der Laufzeit für $n$ Elemente, $f(0) \coloneq 0$. Dann gibt es ein $c \in \R_{>0}$ mit $f(1) \leq c$ und
		\[
			f(n) \leq c \cdot n + \frac{1}{n} \sum_{i = 1}^{n}\big(f(i - 1) + f(n - i)\big) = c \cdot n + \frac{2}{n}\sum_{i = 1}^{n - 1}f(i)
		\]
		Zeige mit vollständiger Induktion 2. Art, dass $f(n) \leq 2c \cdot n \cdot (1 + \ln n)$ für alle $n \in \N_{>0}$.
		
		Induktionsanfang $(n = 1)$: \[
			f(1) \leq c \leq 2c = 2c \cdot 1 \cdot (1 + \ln 1)
		\]
		Induktionsschluss:
		\begin{align*}
			f(n) 
			&\leq c \cdot n + \frac{2}{n}\sum_{i = 1}^{n - 1}(2c\cdot i \cdot (1 + \ln i))\\
			&\leq c \cdot n + \frac{2c}{n}\int_{1}^{n}(2x(1 + \ln x)) dx\\
			&= c \cdot n + \frac{2c}{n}\Big[x^2 + (1 + \ln x) + \frac{x^2}{2}\Big]_1^n\\
			&= 2c \cdot n \cdot (1 + \ln n) - \frac{c}{n}\\
			&\leq 2c \cdot n \cdot (1 + \ln n)
		\end{align*}
	\end{proof}
	\subsection{Nützliche Eigenschaften von Sortierverfahren}
	\begin{center}
		\begin{tabular}{llll}
			&Laufzeit&stabil&in place\\\hline
			SelectionSort & $\Theta(n^2)$& ja & ja\\
			InsertionSort & $\Theta(n^2)$& ja & ja\\
			MergeSort & $\Theta(n \log n)$ & ja &nein\\
			QuickSort & $\Theta(n^2)$ & nein & (ja)\\
			Random QuickSort & $\Theta(n \log n)$ & nein & (ja)
		\end{tabular}
	\end{center}
	Annahme: Ein Sortierverfahren bekommt als Input ein Array. Die Elemente des Arrays besitzen jeweils einen Wert (nicht notwendig verschieden). Der Output ist ein nach Werten sortiertes Array.
	\begin{definition}
		\begin{enumerate}[label=(\roman*)]
			\item Ein Sortierverfahren heißt stabil, falls Elemente mit demselben Wert im Output in derselben Reihenfolge wie im Input auftauchen.
			\item Ein Sortierverfahren arbeitet in Place, falls es zusätzlich zum Eingabearray nur konstant viel Speicher zum Zwischenspeichern der zu sortierenden Einträge benötigt.
		\end{enumerate}
	\end{definition}
	\subsection{Median und $i$-te geordnete Statisik}
	\begin{definition}
		Es sei $A$ eine Menge von $n$ (paarweise verschiedenen) Zahlen und $1 \leq i \leq n$.
		\begin{enumerate}[label=(\alph*)]
			\item Die $i$-te geordnete Statistik ist das $i$-te Element in der aufsteigenden Sortierung von $A$.
			\item Ist $n$ ungerade, so nennt man die $\frac{n + 1}{2}$-te geordnete Statistik auch Median von $A$.
			\item Ist $n$ gerade, so nennt man die $\frac{n}{2}$-te geordnete Statistik auch (unteren) Median von $A$.
		\end{enumerate}
	\end{definition}
	\begin{algorithm}[H]
		\caption{Das Auswahlproblem}
		\KwIn{Menge $A$ von paarweise verschiedenen Zahlen, $i \in \{1, 2, \ldots, n\}$}
		\KwOut{$i$-te geordnete Statistik von $A$}
	\end{algorithm}
	Naheliegender Algorithmus:
	\begin{enumerate}[label=(\arabic*)]
		\item Sortiere $A$ aufsteigend
		\item Gib das $i$-te Element der Sortierung zurück
	\end{enumerate}
	Worst-Case-Laufzeit: $\Theta(n \log n)$. Kann man den Median auch in $\Theta(n)$ finden?
	\subsection{Divide \& Conquer für das Auswahlproblem}
	\begin{algorithm}[H]
		\caption{Algorithmus in Linearzeit für das Laufzeitproblem}
		Auswahl$(A, i)$:\\
		wähle Pivot-Element $x \in A$\\
		$A_{<x} \gets \{a \in A \mid a < x\}$\\
		$A_{>x} \gets \{a \in A \mid a > x\}$\\
		\If{$\lvert A_{< x}\rvert = i - 1$}{\Return{$x$}}
		\ElseIf{$\lvert A_{< x}\rvert \geq i$}{\Return{Auswahl$(A_{<x}, i)$}}
		\Else{\Return{Auswahl$(A_{>x}, i - 1 - \lvert A_{< x}\rvert)$}}
	\end{algorithm}
	\begin{bemerkung}
		\begin{itemize}
			\item Der Algorithmus terminiert, da die rekursiven Aufrufe nur für kleinere Mengen erfolgen, die Rekursionstiefe ist durch $n = \lvert A\rvert$ beschränkt.
			\item Der Algorithmus liefert nach Konstruktion das korrekte Ergebnis.
		\end{itemize}
	\end{bemerkung}
	\begin{beobachtung}
		Wählt man das Pivot-Element $x$ in Linearzeit so, dass
		\[
			\max\{\lvert A_{<x}\rvert, \lvert A_{>x}\rvert\} \leq \lfloor d \cdot \lvert A\rvert\rfloor
		\]
	\end{beobachtung}
\end{document}