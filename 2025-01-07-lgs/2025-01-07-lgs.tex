\documentclass[a4paper,12pt]{article}
\usepackage{amsmath}
\usepackage{amssymb}
\usepackage{amsthm}
\usepackage{marvosym}
\usepackage[utf8]{inputenc}
\usepackage[T1]{fontenc}
\usepackage[ngerman]{babel}
\usepackage[left=2.5cm,right=2cm,top=3cm,bottom=3cm]{geometry}
\usepackage{fancyhdr}
\usepackage{tabularx}
\usepackage{booktabs}
\usepackage{amsfonts}
\usepackage{tcolorbox}
\usepackage{adjustbox}
\usepackage{eurosym}
\usepackage[inline]{enumitem}
\usepackage{lastpage}
\usepackage{mathtools}
\usepackage{microtype}
\usepackage[ruled, linesnumbered, noline]{algorithm2e}
\usepackage{listings}
\usepackage{xcolor}

\setlength{\parindent}{0pt}

\usepackage{tikz}
\usetikzlibrary{decorations.pathmorphing,calc,shadows.blur,shadings,er,positioning}

\definecolor{codegreen}{rgb}{0,0.6,0}
\definecolor{codegray}{rgb}{0.5,0.5,0.5}
\definecolor{codepurple}{rgb}{0.58,0,0.82}
\definecolor{backcolour}{rgb}{0.95,0.95,0.92}

\lstdefinestyle{mystyle}{
	backgroundcolor=\color{backcolour},   
	commentstyle=\color{codegreen},
	keywordstyle=\color{magenta},
	numberstyle=\tiny\color{codegray},
	stringstyle=\color{codepurple},
	basicstyle=\ttfamily\footnotesize,
	breakatwhitespace=false,         
	breaklines=true,                 
	captionpos=b,                    
	keepspaces=true,                 
	numbers=left,                    
	numbersep=5pt,                  
	showspaces=false,                
	showstringspaces=false,
	showtabs=false,                  
	tabsize=2
}

\lstset{style=mystyle}
%\usepackage{pgfplots}
%\usepackage[leqno]{amsmath}

%\newcounter{mathseed}
%\setcounter{mathseed}{1}
%\pgfmathsetseed{\arabic{mathseed}} 
%\pgfdeclarelayer{background}
%\pgfsetlayers{background,main}

\newcommand{\RN}[1]{%
	\textup{\uppercase\expandafter{\romannumeral#1}}%
}

\newcommand{\obda}{o.B.d.A. }

\newenvironment{enum}{\begin{enumerate*}[label=\alph*),itemjoin=\hspace{2em}]}{\end{enumerate*}\\[2ex]}

\DeclareMathOperator{\N}{\mathbb N}
\DeclareMathOperator{\Q}{\mathbb Q}
\DeclareMathOperator{\R}{\mathbb R}
\DeclareMathOperator{\Z}{\mathbb Z}
\DeclareMathOperator{\C}{\mathbb C}
\DeclareMathOperator{\F}{\mathcal{F}}
\DeclareMathOperator{\E}{\mathcal{E}}
\DeclareMathOperator{\BigO}{\mathcal O}
\DeclareMathOperator{\mL}{\mathcal{L}}
\DeclareMathOperator{\mU}{\mathcal{U}}
\DeclareMathOperator{\ggt}{\text{ggT}}
\DeclareMathOperator{\kgv}{\text{kgV}}


%------ Defining multiple types of theorems ------%
\newtheorem{axiom}{Axiom}[section]
\newtheorem{theorem}[axiom]{Theorem}
\newtheorem{lemma}[axiom]{Lemma}
\newtheorem{satz}[axiom]{Satz}
\theoremstyle{definition}
\newtheorem*{example}{Beispiel}
\newtheorem*{bemerkung}{Bemerkung}
\newtheorem*{frage}{Frage}
\newtheorem*{loesung}{Lösung}
\newtheorem*{behauptung}{Behauptung}
\newtheorem*{beobachtung}{Beobachtung}
\newtheorem{definition}[axiom]{Definition}
\newtheorem{folg}[axiom]{Folgerung}
\newtheorem{notation}[axiom]{Notation}

%---- Changing forall and exists ----%
\let\oldforall\forall
\renewcommand{\forall}{\:\oldforall \, }
\let\oldexist\exists
\renewcommand{\exists}{\:\oldexist \: }
\newcommand\existu{\oldexist! \: }
\let\oldepsilon\epsilon
\renewcommand{\epsilon}{\varepsilon}
\newcommand{\corresponds}{\;\widehat{=}\;}

%------ Giving circled star ---------%
\makeatletter
\newcommand{\ostar}{\mathbin{\mathpalette\make@circled\star}}
\newcommand{\make@circled}[2]{%
	\ooalign{$\m@th#1\smallbigcirc{#1}$\cr\hidewidth$\m@th#1#2$\hidewidth\cr}%
}
\newcommand{\smallbigcirc}[1]{%
	\vcenter{\hbox{\scalebox{0.77778}{$\m@th#1\bigcirc$}}}%
}
\makeatother
%-----------------------------------%

%-------------- Vars ---------------%
\newcommand{\vldate}{19. Dezember 2024}
\newcommand{\vlname}{Computerorientierte Mathematik I}
\newcommand{\vltopic}{Lineare Sortieralgorithmen}

\pagestyle{fancy}
\setlength{\headheight}{34pt}
\fancyhf{}
\fancyhead[L]{\includegraphics[width=1cm]{C:/Users/chole/Documents/tub/tublogo.png}}
\fancyhead[C]{{\bfseries\vlname}\\%
	\vltopic}
\fancyhead[R]{\vldate}
\fancyfoot[L]{}
\fancyfoot[C]{- \thepage\ of \pageref{LastPage} -}
\fancyfoot[R]{}

\renewcommand{\headrulewidth}{1pt}
\renewcommand{\footrulewidth}{1pt}
\renewcommand{\labelitemii}{-}

\begin{document}
	\setcounter{section}{9}
	\section{Lineare Gleichungssysteme}
	\begin{definition}
		Ein Lineares Gleichungssystem ist ein System von $m$ Gleichungen mit $n$ Unbestimmten $x_1, x_2, \ldots, x_n$ der Form
		\begin{alignat*}{4}
			&a_{11} \cdot x_1 + &&a_{12} \cdot x_2 + &&\ldots + a_{1n}\cdot x_n &&= b_1\\
			&a_{21} \cdot x_1 + &&a_{22} \cdot x_2 + &&\ldots + a_{2n}\cdot x_n &&= b_2\\
			&&\vdots&&\vdots
		\end{alignat*}
		Mit Koeffizienten $a_{ij}$ und rechter Seite $b_i \in \R$. Gesucht ist die Menge aller Vektoren $x \in \R^n$, die alle Gleichungen erfüllen. Diese Menge heißt {\itshape Lösungsmenge} des Linearen Gleichungssystems.
	\end{definition}
	Die Koeffizienten eines Linearen Gleichungssystems können zu einer Matrix
	\[
		A = (a_{ij})_{1\leq i\leq m, 1\leq j \leq n} \in \R^{m \times n}
	\]
	und die rechte Seite zu dem Vektor
	\[
		b = (b_i)_{1\leq i \leq m} \in \R^m
	\]
	zusammengefasst werden. Damit lässt sich das Lineare Gleichungssystem in der Matrixform $A\cdot x = b$ schreiben:
	\[
		\begin{pmatrix}
			a_{11} & a_{12} & \ldots & a_{1n}\\
			a_{21} & a_{22} & & a_{2n}\\
			\vdots & \vdots & & \vdots\\
			a_{m1} & a_{m2} & \ldots & a_{mn}
		\end{pmatrix}
		\cdot 
		\begin{pmatrix}
			x_1\\
			x_2\\
			\vdots\\
			x_n
		\end{pmatrix}
		= 
		\begin{pmatrix}
			b_1\\
			b_2\\
			\vdots\\
			b_m
		\end{pmatrix}
	\]
	\begin{definition}
		Der Rang {\ttfamily rank(a)} einer Matrix $A \in \R^{m \times n}$ ist die Dimension des von den Spaltenvektoren $a_1, \ldots, a_n$ aufgespanten linearen Untterraumes. 
		\[
			{\text{rank(A)}} \coloneq \dim \big(\text{lin}(a_{\cdot1}, \ldots, a_{\cdot n})\big).
		\]
	\end{definition}
	\begin{example}
		\begin{enumerate}[label=(\arabic*)]
			\item 
			\[
				A = \begin{pmatrix}
					1 & 0 & 0\\
					0 & 0 & 0
				\end{pmatrix}
				\Longrightarrow\text{lin}(a_{\cdot 1}, a_{\cdot 2}, a_{\cdot 3}) = \text{lin}(a_{\cdot 1})
			\]
			also rank$(A) = 1$.
			\item 
			\[
				A = \begin{pmatrix}
					1 & 0 & 2\\
					0 & 1 & 3
				\end{pmatrix}
				\Longrightarrow\text{lin}(a_{\cdot 1}, a_{\cdot 2}, a_{\cdot 3}) = \text{lin}(a_{\cdot 1})
			\]
			also rank$(A) = 2$.
		\end{enumerate}
	\end{example}
	\begin{bemerkung}
		Der Spaltenrang ist gleich dem Zeilenrang.
	\end{bemerkung}
	\subsection{Elementare Umformungen und gestaffelte Form}
	\begin{enumerate}[label=(\alph*)]
		\item Zeilenvertauschung: vertausche zwei Zeilen $k$ und $\ell$
		\[
			k \Leftrightarrow \ell
		\]
		\item Spaltenvertauschung: vertausche zwei Spalten $i$ und $j$
		\[
			x_i \Leftrightarrow x_j
		\]
		\item Skalierung: multipliziere Gleichung $k$ mit beliebiger Zahl $\lambda \neq 0$
		\[
			k \gets \lambda \cdot k
		\]
		\item Addition: addiere Vielfaches $\lambda$ von Gleichung $\ell$ zu Gleichung $k \neq \ell$
		\[
			k \gets k + \lambda \cdot \ell
		\]
	\end{enumerate}
	Wichtig: elementare Umformungen ändern nicht die Lösungsmenge des Linearen Gleichungssystems.
	\begin{example}
		\begin{alignat*}{4}
			1 \cdot x &+ 1 \cdot y &+ 1 \cdot z &= 1\\
			4 \cdot x &+ 4 \cdot y &+ 3 \cdot z &= 5\\
			2 \cdot x &+ 1 \cdot y &+ 1 \cdot z &= 2
		\end{alignat*}
		Als Matrix:
		\[
			\begin{array}{r}
				\RN{1}\\
				\RN{2}\\
				\RN{3}
			\end{array}
			\left(
			\begin{array}{ccc|c}
				1 & 1 & 1 & 1\\
				4 & 4 & 3 & 5\\
				2 & 1 & 1 & 2
			\end{array}
			\right)
		\]
		mit elementaren Umformungen werden die ersten Koeffizienten in $\RN{2}$ und $\RN{3}$ zu $0$ gemacht. Damit wird die Variable $x$ aus $\RN{2}$ und $\RN{3}$ eliminiert.
		\[
			\begin{array}{r}
				\RN{1}\\
				\RN{4}\\
				\RN{5}
			\end{array}
			\left(
			\begin{array}{ccc|c}
				1 & 1 & 1 & 1\\
				0 & 0 & -1 & 1\\
				0 & -1 & -1 & 0
			\end{array}
			\right)
		\]
		\[
			\begin{array}{r}
				\RN{1}\\
				\RN{5}\\
				\RN{4}
			\end{array}
			\left(
			\begin{array}{ccc|c}
				1 & 1 & 1 & 1\\
				0 & -1 & -1 & 0\\
				0 & 0 & -1 & 1
			\end{array}
			\right)
		\]
		Das Lineare Gleichungssystem kann nun gelöst werden und es ergibt sich als Lösung ein Vektor: \[
			\left(\begin{array}{c}
				1\\1\\-1
			\end{array}\right)
		\]
	\end{example}
	\begin{definition}
		Eine Matrix $A \in \R^{m \times n}$ besitzt gestaffelte Form, falls
		\[
			A = \left( 
				\begin{array}{cccccccc}
					\circ & \ast & \ast &\ldots & \ast & \ast &\ldots & \ast\\
					0 & \circ & \ast & \ddots & \ast & \ast & \ldots & \ast\\
					0 & 0 & \circ & \ast & \ddots & \ast & \ldots & \ast\\
					0 & 0 & 0 & \circ & \ast & \ddots & \ast & \ast\\
					& \vdots &&&&& \vdots &\\
					0 & 0 & \ldots & 0 & 0 & \ldots & 0 & 0\\
					\vdots & \vdots && \vdots &\vdots && \vdots & \vdots\\
					0 & 0 & \ldots & 0 & 0 &\ldots & 0 & 0
				\end{array}
			\right)
		\]
		wobei alle mit $\circ$ markierten Einträge $\neq 0$ sind und alle mit $\ast$ markierten Einträge beliebig sind. Dann ist rank$(A)$ gleich der Anzahl der Zeilen mit $\circ$.
	\end{definition}
	Bringe also Lineares Gleichungssystem mittels elementarer Umformungen in gestaffelte Form:
	\[
		\begin{array}{cccccccc|c}
			\tilde{x}_1 & \tilde{x}_2 & \tilde{x}_3 & \ldots & \tilde{x}_r & \tilde{x}_{r + 1} & \ldots & \tilde{x}_n & b\\\hline
			\circ & \ast & \ast & \ldots & \ast & \ast & \ldots & \ast & \ast\\
			0 & \circ & \ast & \ldots & \ast & \ast & \ldots & \ast & \ast\\
			0 & 0 & \circ & \ddots & \vdots & \vdots & & \vdots & \vdots\\
			\vdots & \vdots & \ddots & & \ast & \ast & \ldots & \ast & \ast\\
			\vdots & \vdots & & \ddots & \circ & \ast &\ldots & \ast &\ast\\
			0 & 0 & \ldots & \ldots & 0 & 0 & \ldots & 0 &\times\\
			\vdots & \vdots &&& \vdots & \vdots && \vdots & \vdots\\
			0 & 0 & \ldots & \ldots & 0 & 0 & \ldots & 0 & \times
		\end{array}
	\]
	\begin{itemize}
		\item Alle mit $\circ$ markierten Einträge sind $\neq 0$.
		\item Alle mit $\ast$ oder $\times$ markierten Einträge sind beliebig.
		\item $\tilde{x}_1, \tilde{x}_2, \ldots, \tilde{x}_n$ ist die Umordnung der Variablen $x_1, x_2, \ldots, x_n$ durch Spaltenvertauschung. 
	\end{itemize}
	Ein Lineares Gleichungssystem in gestaffelter Form besitzt
	\begin{itemize}
		\item Keine Lösungen, wenn mindestens ein mit $\times$ markierter Eintrag $\neq 0$ ist.
		\item Lösungne, wenn alle mit $\times$ markierten Einträge $0$ sind (oder keine Nullzeilen existieren). Die Lösungsmenge kann dann wie folgt beschrieben werden:
		\begin{enumerate}[label=(\roman*)]
			\item Die Werte für $\tilde{x}_{r + 1}, \ldots, \tilde{x}_n$ können beliebig vorgegeben werden, 
			\[
				\tilde{x}_{r + 1} = t_1, \ldots, \tilde{x}_{n} = t_{n - r}, t_1, \ldots, t_{n - r} \in \R
			\]
			\item Danach können die restlichen Werte von $\tilde{x}_r, \tilde{x}_{r - 1}, \tilde{x}_1$ durch Auflösen des Linearen Gleichungssystems von oben nach unten bestimmt werden.
		\end{enumerate}
	\end{itemize}
	\subsection{Gauß-Jordan-Eliminationsverfahren}
	\begin{enumerate}
		\item Bestimme ein Element $a_{ij} \neq 0$. Bringe es durch Zeilen- und Spaltenvertauschung an die erste Position der ersten Zeile.
		\item Eliminiere in der ersten Spalte die Einträge $a_{21}, a_{31}, \ldots, a_{m1}$ durch Subtraktion des $\frac{a_{i1}}{a_{11}}$-fachen der ersten Zeile von der $i$-ten Zeile:
		\[
			i \gets i - \frac{a_{i1}}{a_{11}}\cdot \RN{1}
		\]
	\end{enumerate}
	Nach Abschluss des zweiten Schrittes hat das Schema zum Linearen Gleichungssystem die Form
	\[
		\begin{array}{cccc|c}
			\tilde{x}_1 & \tilde{x}_2 & \ldots & \tilde{x}_n & b\\\hline
			\circ & \ast & \ldots & \ast & \ast\\
			0 & \ast & \ldots& \ast & \ast\\
			\vdots & \vdots & & \vdots & \vdots\\
			0 & \ast & \ldots& \ast & \ast\\
		\end{array}
	\]
	Analysiere die Komplexität des Gauß-Jordan-Algorithmus im Einheitskostenmodell, d. h. jede elementare Rechenoperation zählt unabhängig von der Größe der beteiligten Zahlen als ein Schritt.
	
	$m$ Durchläufe der äußeren Schleife (für jede Zeile $i = 1, \ldots, m$)
	\begin{itemize}
		\item $\BigO(mn)$ Schritte, um Pivotelement $\neq 0$ zu finden.
		\item $\BigO(n)$ Schritte für Zeilenvertauschung.
		\item $\BigO(m)$ Durchläufe der inneren Schleife (für Zeilen $j = i + 1, \ldots, m$).
		\begin{itemize}
			\item $\BigO(n)$ Schritte für Subtraktion.
		\end{itemize}
	\end{itemize}
	Gesamtaufwand: $\BigO(m^2n)$ Schritte.
\end{document}