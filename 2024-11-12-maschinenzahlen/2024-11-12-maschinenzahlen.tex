\documentclass[a4paper,12pt]{article}
\usepackage{amsmath}
\usepackage{amssymb}
\usepackage{amsthm}
\usepackage{marvosym}
\usepackage[utf8]{inputenc}
\usepackage[T1]{fontenc}
\usepackage[ngerman]{babel}
\usepackage[left=2.5cm,right=2cm,top=3cm,bottom=3cm]{geometry}
\usepackage{fancyhdr}
\usepackage{tabularx}
\usepackage{booktabs}
\usepackage{amsfonts}
\usepackage{tcolorbox}
\usepackage{adjustbox}
\usepackage{eurosym}
\usepackage[inline]{enumitem}
\usepackage{lastpage}
\usepackage{mathtools}
\usepackage{microtype}
\usepackage{algorithm}
\usepackage{algpseudocode}
\usepackage{listings}
\usepackage{xcolor}
%\usepackage{minted}
\setlength{\parindent}{0pt}

\usepackage{tikz}
\usetikzlibrary{decorations.pathmorphing,calc,shadows.blur,shadings,er,positioning}

\definecolor{codegreen}{rgb}{0,0.6,0}
\definecolor{codegray}{rgb}{0.5,0.5,0.5}
\definecolor{codepurple}{rgb}{0.58,0,0.82}
\definecolor{backcolour}{rgb}{0.95,0.95,0.92}

\lstdefinestyle{mystyle}{
	backgroundcolor=\color{backcolour},   
	commentstyle=\color{codegreen},
	keywordstyle=\color{magenta},
	numberstyle=\tiny\color{codegray},
	stringstyle=\color{codepurple},
	basicstyle=\ttfamily\footnotesize,
	breakatwhitespace=false,         
	breaklines=true,                 
	captionpos=b,                    
	keepspaces=true,                 
	numbers=left,                    
	numbersep=5pt,                  
	showspaces=false,                
	showstringspaces=false,
	showtabs=false,                  
	tabsize=2
}

\lstset{style=mystyle}
%\usepackage{pgfplots}
%\usepackage[leqno]{amsmath}

%\newcounter{mathseed}
%\setcounter{mathseed}{1}
%\pgfmathsetseed{\arabic{mathseed}} 
%\pgfdeclarelayer{background}
%\pgfsetlayers{background,main}

\newcommand{\RN}[1]{%
	\textup{\uppercase\expandafter{\romannumeral#1}}%
}

\newcommand{\obda}{o.B.d.A. }

\newenvironment{enum}{\begin{enumerate*}[label=\alph*),itemjoin=\hspace{2em}]}{\end{enumerate*}\\[2ex]}

\DeclareMathOperator{\N}{\mathbb N}
\DeclareMathOperator{\Q}{\mathbb Q}
\DeclareMathOperator{\R}{\mathbb R}
\DeclareMathOperator{\Z}{\mathbb Z}
\DeclareMathOperator{\F}{\mathcal{F}}
\DeclareMathOperator{\E}{\mathcal{E}}
\DeclareMathOperator{\BigO}{\mathcal O}
\DeclareMathOperator{\ggt}{\text{ggT}}
\DeclareMathOperator{\kgv}{\text{kgV}}


%------ Defining multiple types of theorems ------%
\newtheorem{axiom}{Axiom}[section]
\newtheorem{theorem}[axiom]{Theorem}
\newtheorem{lemma}[axiom]{Lemma}
\newtheorem{satz}[axiom]{Satz}
\theoremstyle{definition}
\newtheorem*{example}{Beispiel}
\newtheorem*{bemerkung}{Bemerkung}
\newtheorem*{frage}{Frage}
\newtheorem*{loesung}{Lösung}
\newtheorem{definition}[axiom]{Definition}
\newtheorem{folg}[axiom]{Folgerung}
\newtheorem{notation}[axiom]{Notation}

%---- Changing forall and exists ----%
\let\oldforall\forall
\renewcommand{\forall}{\:\oldforall \, }
\let\oldexist\exists
\renewcommand{\exists}{\:\oldexist \: }
\newcommand\existu{\oldexist! \: }
\let\oldepsilon\epsilon
\renewcommand{\epsilon}{\varepsilon}
\newcommand{\corresponds}{\;\widehat{=}\;}

%------ Giving circled star ---------%
\makeatletter
\newcommand{\ostar}{\mathbin{\mathpalette\make@circled\star}}
\newcommand{\make@circled}[2]{%
	\ooalign{$\m@th#1\smallbigcirc{#1}$\cr\hidewidth$\m@th#1#2$\hidewidth\cr}%
}
\newcommand{\smallbigcirc}[1]{%
	\vcenter{\hbox{\scalebox{0.77778}{$\m@th#1\bigcirc$}}}%
}
\makeatother
%-----------------------------------%

%-------------- Vars ---------------%
\newcommand{\vldate}{12. November 2024}
\newcommand{\vlname}{Computerorientierte Mathematik I}
\newcommand{\vltopic}{Maschinenzahlen}

\pagestyle{fancy}
\setlength{\headheight}{34pt}
\fancyhf{}
\fancyhead[L]{\includegraphics[width=1cm]{C:/Users/chole/Documents/tub/tublogo.png}}
\fancyhead[C]{{\bfseries\vlname}\\%
	\vltopic}
\fancyhead[R]{\vldate}
\fancyfoot[L]{}
\fancyfoot[C]{- \thepage\ of \pageref{LastPage} -}
\fancyfoot[R]{}

\renewcommand{\headrulewidth}{1pt}
\renewcommand{\footrulewidth}{1pt}
\renewcommand{\labelitemii}{-}

\begin{document}
	\setcounter{section}{4}
	\setcounter{subsection}{3}
	\subsection{Maschinenzahlen}
	\begin{definition}
		Es sei $b \in \N_{\geq 2}$. Für $\mathcal{E} \in \Z$ und $\sigma \in \{\pm 1\}$ ist eine Zahl $x \in \R$ der Form
		\[
			x = \sigma \cdot \Big(\sum_{i = 0}^{m - 1} z_i \cdot b^{-i}\Big) \cdot b^{\mathcal{E}}
		\]
		mit $z_i \in \{0,\ldots, b-1\}$ und $z_0 \neq 0$ eine $m$-stellige $b$-adische normalisierte Gleitkommazahl mit Mantisse $\sum_{i = 0}^{m-1} z_i b^{-i}$ und Exponent $\mathcal{E}$.
	\end{definition}
	\begin{example}
		$b = 10, m = 4$
		\begin{align*}
			x &= 3,141 = (+1)(3 \cdot 10^0 + 1 \cdot 10^{-1} + 4\cdot 10^{-2} + 1 \cdot 10^{-3}) \cdot 10^0\\
			x &= -87,3 = (-1)(8 \cdot 10^0 + 7 \cdot 10^{-1} + 3\cdot 10^{-2}) \cdot 10^1\\
		\end{align*}
		$x = \frac{1}{3}$ besitzt für $b \in \{2,10\}$ und beliebiges $m$ keine Darstellung als $m$-stellige normalisierte Gleitkommazahl.
	\end{example}
	\begin{definition}
		Für $b \in \N_{\geq 2}, m \in \N_{\geq 1}$ bildet die Menge der $b$-adischen $m$-stelligen normalisierten Gleitkommazahlen mit Exponenten $\mathcal{E} \in \{\mathcal{E}_{min}, \ldots, \mathcal{E}_{max}\}$ zuzüglich der Zahl $0$ den Maschinenzahlbereich $\mathcal{F}(b,m,\mathcal{E}_{min}, \mathcal{E}_{max})$
	\end{definition}
	\begin{example}
		IEEE-Standard $754$
		\begin{itemize}
			\item $b = 2$
			\item $m = 53$
			\item $\mathcal{E} \in \{-1022, -1021, \ldots, 1023\}$
			\item Exponentenwerte $E + 1023 \in \{0, 2047\}$ reserviert für $\pm 0$ und $\pm \infty$.
		\end{itemize}
	\end{example}
	\subsection{Maschinengenauigkeit}
	Es sei $\mathcal{F} = F(b,m, \mathcal{E}_{min}, \mathcal{E}_{max})$ ein Maschinenzahlbereich. Eine Abbildung $rd: \R \to \mathcal{F}$ heißt Rundung zu $F$, wenn für alle $x \in \R$ gilt: $\lvert x - rd(x)\rvert = \min_{a \in \mathcal{F}}\lvert x - a\rvert$.
	\begin{example}
		\begin{itemize}
			\item Kaufmännische Rundung
			\item IEEE 754: Runde im Zweifel so, dass letzte Stelle gerade wird.
		\end{itemize}
	\end{example}
	
	\begin{definition}
		Es sei $\tilde{x}$ eine Näherung von $x \in \R$.
		\begin{enumerate}[label=(\roman*)]
			\item $\lvert x - \tilde{x}\rvert$ wird absoluter Fehler genannt.
			\item $\frac{\lvert x - \tilde{x}\rvert}{x}$ wird relativer Fehler genannt.
		\end{enumerate}
	\end{definition}
	\begin{definition}
		Es sei $\mathcal{F}$ Maschinenzahlbereich mit Rundung $rd$. Die Maschinengenauigkeit von $\mathcal{F}$ ist
		\[
			eps(F) \coloneq \sup\Big\{\Big\lvert \frac{x - rd(x)}{x}\Big\rvert \mid x \in \R \text{ und } \lvert x \rvert \in \text{ range}(\mathcal{F})\Big\},
		\]
		mit range$(\mathcal{F} \coloneq [\mathcal{F}_{min}, \mathcal{F}_{max}])$ und $\mathcal{F}_{min} (\mathcal{F}_{max})$ kleinste (größte) darstellbare positive Zahl in $\mathcal{F}$
	\end{definition}
	
	\begin{satz}
		Für jeden Maschinenzahlbereich $\mathcal{F}(b,m,\mathcal{E}_{min},\mathcal{E}_{max})$ mit $\mathcal{E}_{min} < \mathcal{E}_{max}$ gilt:
		\[
			eps(\mathcal{F}) = \frac{1}{1 + 2b^{m-1}}
		\]
	\end{satz}
	\begin{definition}
		Es sei $\F$ Maschinenzahlbereich und $s \in \N$. Dann hat $f \in \F$ (mindestens) $s$ sogmofolamte Stellen in der $b$-adischen Gleitkommadarstellung, falls $f \neq 0$ und für jede Rundung $rd$ und jede Zahl 
		$x \in \R$ mit $rd(x) = f$ gilt:
		\[
			\lvert x - f\rvert \leq \frac{1}{2} \cdot b^{\lfloor \log_b \lvert f\rvert \rfloor + 1 - s}
		\]
	\end{definition}
	\subsection{Maschinenbauarithmetik}
	Es sei $\F$ ein Maschinenzahlbereich und $\circ \in \{+,-,\cdot,/\}$ sei eine Operation. Problem:
	
	Für $x,y \in \F$ gilt im Allgemeinen nicht $x \circ y \in \F$.
	
	Pragmatische Lösung: Ersatzoperation $\circledcirc \in \{\oplus, \ominus, \odot, \oslash\}$ mit $x \circledcirc y = rd(x \circ y)$ für Rundung $rd$ zu $\F$.
	\begin{example}
		Es sei $\F = (10,2,-5,5), x = 4,5 \cdot 10^1 = 45$ und $y = 1,1-10^0 = 1,1$
		\[
			x \oplus y = rd(x + y) = rd(46,1) = rd(4,61 \cdot 10^1) = 4,6 \cdot 10^1
		\]
	\end{example}
	\begin{bemerkung}
		\begin{itemize}
			\item Zur Berechnung von $x \circledcirc y$ muss man $x \circ y$ nicht berechnen.
			\item Grundrechenarten für natürliche Zahlen reichen.
			\item Ist $\lvert x \circ y \rvert \in \text{ range}(\F)$, so gilt für den relativen Fehler
			\[
			\Big\lvert \frac{x \circ y - x \circledcirc y}{x \circ y}\Big\rvert = \Big\lvert \frac{x \circ y - rd(x \circ y)}{x \circ y}\Big\rvert \leq eps(\F)
			\]
			\item Kommutativgesetz gilt für $\oplus, \odot$, nicht jedoch Assoziativ- und Distributivgesetz.
		\end{itemize}
	\end{bemerkung}
\end{document}