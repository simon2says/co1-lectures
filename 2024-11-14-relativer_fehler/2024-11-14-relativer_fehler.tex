\documentclass[a4paper,12pt]{article}
\usepackage{amsmath}
\usepackage{amssymb}
\usepackage{amsthm}
\usepackage{marvosym}
\usepackage[utf8]{inputenc}
\usepackage[T1]{fontenc}
\usepackage[ngerman]{babel}
\usepackage[left=2.5cm,right=2cm,top=3cm,bottom=3cm]{geometry}
\usepackage{fancyhdr}
\usepackage{tabularx}
\usepackage{booktabs}
\usepackage{amsfonts}
\usepackage{tcolorbox}
\usepackage{adjustbox}
\usepackage{eurosym}
\usepackage[inline]{enumitem}
\usepackage{lastpage}
\usepackage{mathtools}
\usepackage{microtype}
\usepackage[ruled, linesnumbered, noline]{algorithm2e}
\usepackage{listings}
\usepackage{xcolor}

\setlength{\parindent}{0pt}

\usepackage{tikz}
\usetikzlibrary{decorations.pathmorphing,calc,shadows.blur,shadings,er,positioning}

\definecolor{codegreen}{rgb}{0,0.6,0}
\definecolor{codegray}{rgb}{0.5,0.5,0.5}
\definecolor{codepurple}{rgb}{0.58,0,0.82}
\definecolor{backcolour}{rgb}{0.95,0.95,0.92}

\lstdefinestyle{mystyle}{
	backgroundcolor=\color{backcolour},   
	commentstyle=\color{codegreen},
	keywordstyle=\color{magenta},
	numberstyle=\tiny\color{codegray},
	stringstyle=\color{codepurple},
	basicstyle=\ttfamily\footnotesize,
	breakatwhitespace=false,         
	breaklines=true,                 
	captionpos=b,                    
	keepspaces=true,                 
	numbers=left,                    
	numbersep=5pt,                  
	showspaces=false,                
	showstringspaces=false,
	showtabs=false,                  
	tabsize=2
}

\lstset{style=mystyle}
%\usepackage{pgfplots}
%\usepackage[leqno]{amsmath}

%\newcounter{mathseed}
%\setcounter{mathseed}{1}
%\pgfmathsetseed{\arabic{mathseed}} 
%\pgfdeclarelayer{background}
%\pgfsetlayers{background,main}

\newcommand{\RN}[1]{%
	\textup{\uppercase\expandafter{\romannumeral#1}}%
}

\newcommand{\obda}{o.B.d.A. }

\newenvironment{enum}{\begin{enumerate*}[label=\alph*),itemjoin=\hspace{2em}]}{\end{enumerate*}\\[2ex]}

\DeclareMathOperator{\N}{\mathbb N}
\DeclareMathOperator{\Q}{\mathbb Q}
\DeclareMathOperator{\R}{\mathbb R}
\DeclareMathOperator{\Z}{\mathbb Z}
\DeclareMathOperator{\F}{\mathcal{F}}
\DeclareMathOperator{\E}{\mathcal{E}}
\DeclareMathOperator{\BigO}{\mathcal O}
\DeclareMathOperator{\mL}{\mathcal{L}}
\DeclareMathOperator{\mU}{\mathcal{U}}
\DeclareMathOperator{\ggt}{\text{ggT}}
\DeclareMathOperator{\kgv}{\text{kgV}}


%------ Defining multiple types of theorems ------%
\newtheorem{axiom}{Axiom}[section]
\newtheorem{theorem}[axiom]{Theorem}
\newtheorem{lemma}[axiom]{Lemma}
\newtheorem{satz}[axiom]{Satz}
\theoremstyle{definition}
\newtheorem*{example}{Beispiel}
\newtheorem*{bemerkung}{Bemerkung}
\newtheorem*{frage}{Frage}
\newtheorem*{loesung}{Lösung}
\newtheorem{definition}[axiom]{Definition}
\newtheorem{folg}[axiom]{Folgerung}
\newtheorem{notation}[axiom]{Notation}

%---- Changing forall and exists ----%
\let\oldforall\forall
\renewcommand{\forall}{\:\oldforall \, }
\let\oldexist\exists
\renewcommand{\exists}{\:\oldexist \: }
\newcommand\existu{\oldexist! \: }
\let\oldepsilon\epsilon
\renewcommand{\epsilon}{\varepsilon}
\newcommand{\corresponds}{\;\widehat{=}\;}

%------ Giving circled star ---------%
\makeatletter
\newcommand{\ostar}{\mathbin{\mathpalette\make@circled\star}}
\newcommand{\make@circled}[2]{%
	\ooalign{$\m@th#1\smallbigcirc{#1}$\cr\hidewidth$\m@th#1#2$\hidewidth\cr}%
}
\newcommand{\smallbigcirc}[1]{%
	\vcenter{\hbox{\scalebox{0.77778}{$\m@th#1\bigcirc$}}}%
}
\makeatother
%-----------------------------------%

%-------------- Vars ---------------%
\newcommand{\vldate}{14. November 2024}
\newcommand{\vlname}{Computerorientierte Mathematik I}
\newcommand{\vltopic}{Relativer Fehler}

\pagestyle{fancy}
\setlength{\headheight}{34pt}
\fancyhf{}
\fancyhead[L]{\includegraphics[width=1cm]{C:/Users/chole/Documents/tub/tublogo.png}}
\fancyhead[C]{{\bfseries\vlname}\\%
	\vltopic}
\fancyhead[R]{\vldate}
\fancyfoot[L]{}
\fancyfoot[C]{- \thepage\ of \pageref{LastPage} -}
\fancyfoot[R]{}

\renewcommand{\headrulewidth}{1pt}
\renewcommand{\footrulewidth}{1pt}
\renewcommand{\labelitemii}{-}

\begin{document}
	\setcounter{section}{4}
	\section{Rechnen mit Fehlern}
	\subsection{Fehlerfortpflanzung}
	\begin{lemma}
		Es seien $x, y \in \R$ und $\tilde{x}, \tilde{y}$ Näherungen mit $\epsilon_x = \frac{x - \tilde{x}}{x}$ und $\epsilon_y = \frac{y - \tilde{y}}{y}$. Für $\circ \in \{+, -, \cdot, \div\}$ mit $x \circ y \neq 0$ und $\epsilon_\circ \coloneq \frac{(x \circ y) - (\tilde{x} \circ \tilde{y})}{x \circ y}$ gilt dann:
		\begin{align*}
			\epsilon_+ &= \epsilon_x \cdot \frac{x}{x + y} + \epsilon_y \cdot \frac{y}{x + y}\\
			\epsilon_- &= \epsilon_x \cdot \frac{x}{x - y} + \epsilon_y \cdot \frac{y}{x - y}\\
			\epsilon_\cdot &= \epsilon_x + \epsilon_y - \epsilon_x \cdot \epsilon_y\\
			\epsilon_\div &= \frac{\epsilon_x - \epsilon_y}{1 - \epsilon_y}
		\end{align*}
	\end{lemma}
	\begin{proof}
		Es gilt für $\tilde{x} = x \cdot (1 - \epsilon_x)$ und für $\tilde{y} = y \cdot (1-\epsilon_y)$. Dann ausrechnen.
	\end{proof}
	\begin{bemerkung}
		$\epsilon_+$ und $\epsilon_-$ können sehr groß werden, während $\epsilon_\cdot$ und $\epsilon_\div$ eher klein bleiben.
	\end{bemerkung}
	\begin{example}
		Lösung eines Linearen Gleichungssystems. Betrachte das folgende LGS:
		\begin{align*}
			10^{-20}x + 2y &= 1\\
			10^{-20}x + 10^{-20}y &= 10^{-20}
			\intertext{Durch Subtraktion der beiden Gleichungen folgt:}
			(2 - 10^{-20})y &= (1 - 10^{-20})
			\intertext{Runden auf die nächste darstellbare Zahl:}
			2y &= 1
		\end{align*}
		Also $y = \frac{1}{2}$. Einsetzen in die erste Gleichung liefert $x = 0$. Die korrekte Lösung wäre:
		\[
			x = \frac{1}{2 - 10^{-20}} = 0.5000\ldots, y = \frac{1-10^{-20}}{2-10^{-20}} = 0.4999\ldots
		\]
	\end{example}
	\subsection{Binäre Suche}
	$f: \R \to \R$ monoton steigend, $\mathcal{U}, \mathcal{L} \in \R, x \in \R$. Ziel: Berechne $f^{-1}(x) \in \R$.
	\begin{example}
		Es sei $f: \R_{\leq 0} \to \R_{\leq 0}$ mit $f(x) \coloneq x^2$; gesucht ist $f^{-1}(3) \approx 1.73205$
		Beachte: $f$ ist monoton wachsend.
		\begin{itemize}
			\item Da $f(1) = 1^2 = 1 < 3$ und $f(2) = 2^2 = 4 > 3$, ist $\sqrt{3} \in [1,2]$
			\item Da $f(1.5) = 1.5^2 = 2.25 < 3$, ist $\sqrt{3} \in [1.5,2]$
			\item Da $f(1.75) = 1.75^2 = 3.0625 > 3$, ist $\sqrt{3} \in [1.5, 1.75]$
			\item Da $f(1.625) = 1.625^2 = 2.640625 < 3$, ist $\sqrt{3} \in [1.625, 1.75]$
		\end{itemize}
		Es sei $[\ell_i, u_i]$ das Intervall in der $i$-ten Iteration und $m_i = \frac{\ell_i + u_i}{2}$.
	\end{example}
	\begin{bemerkung}
		\begin{itemize}
			\item Offenbar ist
			\[
				\lim_{i \to \infty} \ell_i = \lim_{i \to \infty} m_i = \lim_{i \to \infty} u_i = \sqrt{3}
			\]
			\item Denn wir wissen a priori, dass
			\[
				\lvert m_i - \sqrt{3} \rvert \leq \frac{1}{2}(u_i - \ell_i) = 2^{-i}(u_1 - \ell_1) = 2^{-i}
			\]
			\item Also ist zum Beispiel $\lvert m_6 - \sqrt{3} \rvert \leq 2^{-6} = \frac{1}{64}$
			\item A posteriori stellen wir fest, dass 
			\[
				\lvert m_i - \sqrt{3} \rvert = \frac{\lvert m_i^2 - 3\rvert}{m_i + \sqrt{3}} \leq \frac{\lvert m_i^2 - 3\rvert}{m_i + \ell_i}
			\]
			\item Für $i = 6$ ergibt sich bspw.
			\[
				\lvert m_6 - \sqrt{3} \rvert \leq \frac{0.008056640625}{3.453125} \approx 0.002333
			\]
		\end{itemize}
	\end{bemerkung}
	\subsection{Diskrete binäre Suche}
	\begin{algorithm}
		\caption{Binäre Suche}
		\KwIn{Orakel für monoton wachsende Funktion $f: \Z \to \R$, Zahlen $\mathcal{L}, \mathcal{U} \in \Z$ mit $\mathcal{L} < \mathcal{U}$ sowie $y \in \R$ mit $y \geq f(\mathcal{L})$}
		\KwOut{Das maximale  $n \in \{\mL, \ldots, \mU\}$ mit $f(n) \leq y$}
		$\ell \gets 2, u \gets \mU + 1$\\
		\While{$u - 1 < \ell$}{
			$m \gets \lfloor \frac{\ell + u}{2}\rfloor$\\
			\If{$f(m) > y$}{
				$u \gets m$
			}
			\Else{
				$\ell \gets m$
			}
		}
		\Return{$\ell$}
	\end{algorithm}
	\begin{satz}
		Der Algorithmus liefert das korrekte Ergebnis nach $\BigO(\log(\mU - \mL + 2))$ Iterationen.
	\end{satz}
	\begin{proof}
		Zeige durch Induktion, dass jederzeit
		\begin{enumerate}[label=(\roman*)]
			\item $\mL \leq \ell \leq u - 1 \leq \mU$
			\item $f(\ell) \leq y$
			\item $u > \mU$ oder $f(u) > y$
		\end{enumerate}
		Also liegt korrektes $n$ stets in $\{\ell, \ldots, u - 1\}$. Terminiert der Algorithmus, ist der Output korrekt.\\[2ex]
		Laufzeit: In jeder Iteration verringert sich $u - \ell - 1$ auf höchstens
		\begin{align*}
			&\max\Big\{ \Big\lfloor \frac{\ell + u}{2} \Big\rfloor - \ell - 1, u - \Big\lfloor \frac{\ell + u}{2} - 1 \Big\rfloor \Big\}\\
			\leq&\max\Big\{\frac{\ell + u}{2} - \ell - 1, u - \frac{\ell + u - 1}{2} - 1\Big\}\\
			=&\frac{u - \ell - 1}{2}
		\end{align*}
	\end{proof}
\end{document}