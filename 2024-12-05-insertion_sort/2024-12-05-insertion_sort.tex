\documentclass[a4paper,12pt]{article}
\usepackage{amsmath}
\usepackage{amssymb}
\usepackage{amsthm}
\usepackage{marvosym}
\usepackage[utf8]{inputenc}
\usepackage[T1]{fontenc}
\usepackage[ngerman]{babel}
\usepackage[left=2.5cm,right=2cm,top=3cm,bottom=3cm]{geometry}
\usepackage[dvipsnames]{xcolor}
\usepackage{fancyhdr}
\usepackage{tabularx}
\usepackage{booktabs}
\usepackage{amsfonts}
\usepackage{tcolorbox}
\usepackage{adjustbox}
\usepackage{eurosym}
\usepackage[inline]{enumitem}
\usepackage{lastpage}
\usepackage{mathtools}
\usepackage{microtype}
\usepackage[ruled, linesnumbered, noline]{algorithm2e}
\usepackage{listings}


\setlength{\parindent}{0pt}

\usepackage{tikz}
\usetikzlibrary{decorations.pathmorphing,calc,shadows.blur,shadings,er,positioning}

\definecolor{codegreen}{rgb}{0,0.6,0}
\definecolor{codegray}{rgb}{0.5,0.5,0.5}
\definecolor{codepurple}{rgb}{0.58,0,0.82}
\definecolor{backcolour}{rgb}{0.95,0.95,0.92}

\lstdefinestyle{mystyle}{
	backgroundcolor=\color{backcolour},   
	commentstyle=\color{magenta},
	keywordstyle=\color{NavyBlue},
	numberstyle=\tiny\color{codegray},
	stringstyle=\color{codepurple},
	basicstyle=\ttfamily\footnotesize,
	breakatwhitespace=false,         
	breaklines=true,                 
	captionpos=b,                    
	keepspaces=true,                 
	numbers=left,                    
	numbersep=5pt,                  
	showspaces=false,                
	showstringspaces=false,
	showtabs=false,                  
	tabsize=2
}

\lstset{style=mystyle}
%\usepackage{pgfplots}
%\usepackage[leqno]{amsmath}

%\newcounter{mathseed}
%\setcounter{mathseed}{1}
%\pgfmathsetseed{\arabic{mathseed}} 
%\pgfdeclarelayer{background}
%\pgfsetlayers{background,main}

\newcommand{\RN}[1]{%
	\textup{\uppercase\expandafter{\romannumeral#1}}%
}

\newcommand{\obda}{o.B.d.A. }

\newenvironment{enum}{\begin{enumerate*}[label=\alph*),itemjoin=\hspace{2em}]}{\end{enumerate*}\\[2ex]}

\DeclareMathOperator{\N}{\mathbb N}
\DeclareMathOperator{\Q}{\mathbb Q}
\DeclareMathOperator{\R}{\mathbb R}
\DeclareMathOperator{\Z}{\mathbb Z}
\DeclareMathOperator{\C}{\mathbb C}
\DeclareMathOperator{\F}{\mathcal{F}}
\DeclareMathOperator{\E}{\mathcal{E}}
\DeclareMathOperator{\BigO}{\mathcal O}
\DeclareMathOperator{\mL}{\mathcal{L}}
\DeclareMathOperator{\mU}{\mathcal{U}}
\DeclareMathOperator{\ggt}{\text{ggT}}
\DeclareMathOperator{\kgv}{\text{kgV}}


%------ Defining multiple types of theorems ------%
\newtheorem{axiom}{Axiom}[section]
\newtheorem{theorem}[axiom]{Theorem}
\newtheorem{lemma}[axiom]{Lemma}
\newtheorem{satz}[axiom]{Satz}
\theoremstyle{definition}
\newtheorem*{example}{Beispiel}
\newtheorem*{bemerkung}{Bemerkung}
\newtheorem*{frage}{Frage}
\newtheorem*{loesung}{Lösung}
\newtheorem*{behauptung}{Behauptung}
\newtheorem*{beobachtung}{Beobachtung}
\newtheorem{definition}[axiom]{Definition}
\newtheorem{folg}[axiom]{Folgerung}
\newtheorem{notation}[axiom]{Notation}

%---- Changing forall and exists ----%
\let\oldforall\forall
\renewcommand{\forall}{\:\oldforall \, }
\let\oldexist\exists
\renewcommand{\exists}{\:\oldexist \: }
\newcommand\existu{\oldexist! \: }
\let\oldepsilon\epsilon
\renewcommand{\epsilon}{\varepsilon}
\newcommand{\corresponds}{\;\widehat{=}\;}

%------ Giving circled star ---------%
\makeatletter
\newcommand{\ostar}{\mathbin{\mathpalette\make@circled\star}}
\newcommand{\make@circled}[2]{%
	\ooalign{$\m@th#1\smallbigcirc{#1}$\cr\hidewidth$\m@th#1#2$\hidewidth\cr}%
}
\newcommand{\smallbigcirc}[1]{%
	\vcenter{\hbox{\scalebox{0.77778}{$\m@th#1\bigcirc$}}}%
}
\makeatother
%-----------------------------------%

%-------------- Vars ---------------%
\newcommand{\vldate}{05. Dezember 2024}
\newcommand{\vlname}{Computerorientierte Mathematik I}
\newcommand{\vltopic}{Sortieralgorithmen}

\pagestyle{fancy}
\setlength{\headheight}{34pt}
\fancyhf{}
\fancyhead[L]{\includegraphics[width=1cm]{C:/Users/chole/Documents/tub/tublogo.png}}
\fancyhead[C]{{\bfseries\vlname}\\%
	\vltopic}
\fancyhead[R]{\vldate}
\fancyfoot[L]{}
\fancyfoot[C]{- \thepage\ of \pageref{LastPage} -}
\fancyfoot[R]{}

\renewcommand{\headrulewidth}{1pt}
\renewcommand{\footrulewidth}{1pt}
\renewcommand{\labelitemii}{-}

\begin{document}
	\setcounter{section}{8}
	\setcounter{subsection}{2}
	\begin{proof}[Laufzeitanalyse]
		Zeilen 3 und 4 benötigen konstante Laufzeit in $\BigO(1)$. In Iteration $i$ der äußeren Schleife hat die innere {\ttfamily for}-Schleife genau $n-1-i$ Iterationen. Daher ist die Anzahl der Aufrufe
		\[
			\sum_{i = 0}^{n - 2}(n - 1 - i) = \sum_{k = 1}^{n - 1}\frac{n(n - 1)}{2}\in \Theta(n^2)
		\]
	\end{proof}
	\begin{proof}[Korrektheitsbeweis]
		Zeige mittels vollständiger Induktion über $k$ folgende Schleifeninvariante:
		
		Für $k = 0, 1, \ldots, n - 2$ gilt nach $k$ Iterationen der äußeren {\ttfamily for}-Schleife
		\[
			A[i] \nsucc A[j] \text{ für } i = 0, \ldots, k - 1 \text{ und } j = i + 1, \ldots, n - 1
		\]
		d. h. die Felder $A[0], \ldots, A[k - 1]$ sind topologisch aufsteigend sortiert und keiner ist größer als eines der übrigen Felder $A[k], \ldots, A[n - 1]$.
		
		Induktionsanfang $(k = 0)$: Trivial (leere Aussage.)
		
		Induktionsschluss $(k \to k + 1)$:\\[2ex]
		Die Behauptung gelte für ein beliebiges, aber fest gewähltes $k$, d. h. 
		\[
			A[i] \nsucc A[j] \text{ für } i = 0, \ldots, k - 1 \text{ und } j = i + 1, \ldots, n - 1
		\]
		Betrachte nun $(k + 1)$-te Iteration der äußeren {\ttfamily for}-Schleife (d. h. $i = k$). Nach Durchlauf der inneren {\ttfamily for}-Schleife $(j = k + 1, \ldots, n - 1)$ gilt
		\[
			A[k] \nsucc A[j] \text{ für } j = k + 1, \ldots, n - 1
		\]
		Gemeinsam implizieren diese Aussagen die Schleifeninvariante für $k + 1$.
	\end{proof}
	\begin{lstlisting}[language=Python]
def selectionSort(A):
# sort the given list A
	n = len(A)
	for i in range(0, n - 1):
		for j in range(i + 1, n):
			if A[i] % A[j] == 0:
				A[i], A[j] = A[j], A[i]\end{lstlisting}
	\begin{bemerkung}
		Für partielle Ordnungen gibt SelectionSort eine topologische Sortierung zurück.
	\end{bemerkung}
	\subsection{Insertion Sort}
	\begin{algorithm}[H]
		\caption{InsertionSort}
		\KwIn{Array $A$ der Länge $n$, totale Ordnung $\preceq$}
		\KwOut{$A$ sortiert}
		\For{$i \gets 1$ \KwTo $n - 1$}{
			$s \gets A[i]$\\
			$k \gets i$
			\While{$k > 0 \land s \prec A[k - 1]$}{
				$A[k] \gets A[k - 1]$\\
				$k \gets k - 1$
			}
			$A[k] \gets s$
		}
	\end{algorithm}
	\begin{lstlisting}[language=Python]
def insertionSort(A):
	n = len(A)
	for i in range(1, n):
		s = A[i]
		k = i
		while k > 0 and s < A[k - 1]:
			A[k] = A[k - 1]
			k -= 1
		A[k] = s\end{lstlisting}
	\begin{theorem}
		Sortieren durch Einfügen ist korrekt und hat Laufzeit $\Theta(n^2)$.
	\end{theorem}
	\begin{bemerkung}
		Es sei $\preceq$ nur eine partielle Ordnung.
		\begin{itemize}
			\item Sortieren durch Einfügen liefert im Allgemeinen keine topologische Sortierung.
			\item Gegenbeispiel: Ist $\preceq$ die Teilbarkeitsrelation, so liefert InsertionSort bei Eingabe $A = [4, 5, 2]$ die Ausgabe $A = [4, 5, 2]$.
			\item Für die Aussgabe ist in diesem Fall lediglich
			\[
				A[0] \nsucceq A[1] \nsucceq \ldots \nsucceq A[n - 1]
			\]
		\end{itemize}
	\end{bemerkung}
	\subsection{Divide and Conquer}
	Idee für einen rekursiven Sortieralgorithmus:
	\begin{enumerate}[label=(\roman*)]
		\item Teile zu sortierende Menge $A$ in zwei etwa gleich große Teilmengen auf.
		\item Sortiere beide Teilmengen (rekursiv).
		\item Füge die beiden Sortierungen zu einer Sortierung von $A$ zusammen.
	\end{enumerate}
	\begin{algorithm}[H]
		\caption{Die Merge-Operation}
		\KwIn{Sortierungen $\pi_A = \{1, 2, \ldots, \lvert A\rvert\} \to A, \pi_B = \{1, 2, \ldots, \lvert B\rvert\} \to B$}
		\KwOut{Sortierung $\pi: \{1, 2, \ldots, \lvert A + B\rvert\} \to A \cup B$}
		$a \gets 1$\\
		$b \gets 1$\\
		$c \gets 1$
		\While{$a \leq \lvert A\rvert \land b \leq \lvert B\rvert$}{
			\If{$\pi_A(a) \preceq \pi_B(b)$}{
				$\pi(c) \gets \pi_A(a)$\\
				$a \gets a + 1$
			}
			\Else{
				$\pi(c) \gets \pi_B(b)$\\
				$b \gets b + 1$
			}
			$c \gets c + 1$
		}
		\While{$a \leq \lvert A\rvert$}{
			$\pi(c) \gets \pi_A(a)$\\
			$a \gets a + 1$\\
			$c \gets c + 1$
		}
		\While{$b \leq \lvert B\rvert$}{
			$\pi(c) \gets \pi_B(b)$\\
			$b \gets b + 1$\\
			$c \gets c + 1$
		}
	\end{algorithm}
	\begin{lemma}
		Die Merge-Operation arbeitet korrekt. Sie benötigt $\leq \lvert A\rvert + \lvert B\rvert - 1$ Vergleiche und hat Laufzeit $\BigO(\lvert A\rvert + \lvert B\rvert)$.
	\end{lemma}
\end{document}