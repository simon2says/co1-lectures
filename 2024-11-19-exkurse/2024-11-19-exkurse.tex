\documentclass[a4paper,12pt]{article}
\usepackage{amsmath}
\usepackage{amssymb}
\usepackage{amsthm}
\usepackage{marvosym}
\usepackage[utf8]{inputenc}
\usepackage[T1]{fontenc}
\usepackage[ngerman]{babel}
\usepackage[left=2.5cm,right=2cm,top=3cm,bottom=3cm]{geometry}
\usepackage{fancyhdr}
\usepackage{tabularx}
\usepackage{booktabs}
\usepackage{amsfonts}
\usepackage{tcolorbox}
\usepackage{adjustbox}
\usepackage{eurosym}
\usepackage[inline]{enumitem}
\usepackage{lastpage}
\usepackage{mathtools}
\usepackage{microtype}
\usepackage[ruled, linesnumbered, noline]{algorithm2e}
\usepackage{listings}
\usepackage{xcolor}

\setlength{\parindent}{0pt}

\usepackage{tikz}
\usetikzlibrary{decorations.pathmorphing,calc,shadows.blur,shadings,er,positioning}

\definecolor{codegreen}{rgb}{0,0.6,0}
\definecolor{codegray}{rgb}{0.5,0.5,0.5}
\definecolor{codepurple}{rgb}{0.58,0,0.82}
\definecolor{backcolour}{rgb}{0.95,0.95,0.92}

\lstdefinestyle{mystyle}{
	backgroundcolor=\color{backcolour},   
	commentstyle=\color{codegreen},
	keywordstyle=\color{magenta},
	numberstyle=\tiny\color{codegray},
	stringstyle=\color{codepurple},
	basicstyle=\ttfamily\footnotesize,
	breakatwhitespace=false,         
	breaklines=true,                 
	captionpos=b,                    
	keepspaces=true,                 
	numbers=left,                    
	numbersep=5pt,                  
	showspaces=false,                
	showstringspaces=false,
	showtabs=false,                  
	tabsize=2
}

\lstset{style=mystyle}
%\usepackage{pgfplots}
%\usepackage[leqno]{amsmath}

%\newcounter{mathseed}
%\setcounter{mathseed}{1}
%\pgfmathsetseed{\arabic{mathseed}} 
%\pgfdeclarelayer{background}
%\pgfsetlayers{background,main}

\newcommand{\RN}[1]{%
	\textup{\uppercase\expandafter{\romannumeral#1}}%
}

\newcommand{\obda}{o.B.d.A. }

\newenvironment{enum}{\begin{enumerate*}[label=\alph*),itemjoin=\hspace{2em}]}{\end{enumerate*}\\[2ex]}

\DeclareMathOperator{\N}{\mathbb N}
\DeclareMathOperator{\Q}{\mathbb Q}
\DeclareMathOperator{\R}{\mathbb R}
\DeclareMathOperator{\Z}{\mathbb Z}
\DeclareMathOperator{\C}{\mathbb C}
\DeclareMathOperator{\F}{\mathcal{F}}
\DeclareMathOperator{\E}{\mathcal{E}}
\DeclareMathOperator{\BigO}{\mathcal O}
\DeclareMathOperator{\mL}{\mathcal{L}}
\DeclareMathOperator{\mU}{\mathcal{U}}
\DeclareMathOperator{\ggt}{\text{ggT}}
\DeclareMathOperator{\kgv}{\text{kgV}}


%------ Defining multiple types of theorems ------%
\newtheorem{axiom}{Axiom}[section]
\newtheorem{theorem}[axiom]{Theorem}
\newtheorem{lemma}[axiom]{Lemma}
\newtheorem{satz}[axiom]{Satz}
\theoremstyle{definition}
\newtheorem*{example}{Beispiel}
\newtheorem*{bemerkung}{Bemerkung}
\newtheorem*{frage}{Frage}
\newtheorem*{loesung}{Lösung}
\newtheorem{definition}[axiom]{Definition}
\newtheorem{folg}[axiom]{Folgerung}
\newtheorem{notation}[axiom]{Notation}

%---- Changing forall and exists ----%
\let\oldforall\forall
\renewcommand{\forall}{\:\oldforall \, }
\let\oldexist\exists
\renewcommand{\exists}{\:\oldexist \: }
\newcommand\existu{\oldexist! \: }
\let\oldepsilon\epsilon
\renewcommand{\epsilon}{\varepsilon}
\newcommand{\corresponds}{\;\widehat{=}\;}

%------ Giving circled star ---------%
\makeatletter
\newcommand{\ostar}{\mathbin{\mathpalette\make@circled\star}}
\newcommand{\make@circled}[2]{%
	\ooalign{$\m@th#1\smallbigcirc{#1}$\cr\hidewidth$\m@th#1#2$\hidewidth\cr}%
}
\newcommand{\smallbigcirc}[1]{%
	\vcenter{\hbox{\scalebox{0.77778}{$\m@th#1\bigcirc$}}}%
}
\makeatother
%-----------------------------------%

%-------------- Vars ---------------%
\newcommand{\vldate}{19. November 2024}
\newcommand{\vlname}{Computerorientierte Mathematik I}
\newcommand{\vltopic}{Exkurse}

\pagestyle{fancy}
\setlength{\headheight}{34pt}
\fancyhf{}
\fancyhead[L]{\includegraphics[width=1cm]{C:/Users/chole/Documents/tub/tublogo.png}}
\fancyhead[C]{{\bfseries\vlname}\\%
	\vltopic}
\fancyhead[R]{\vldate}
\fancyfoot[L]{}
\fancyfoot[C]{- \thepage\ of \pageref{LastPage} -}
\fancyfoot[R]{}

\renewcommand{\headrulewidth}{1pt}
\renewcommand{\footrulewidth}{1pt}
\renewcommand{\labelitemii}{-}

\begin{document}
	\setcounter{section}{5}
	\setcounter{subsection}{5}
	\subsection{Exkurs: Codierung von Zeichen}
	Einzelne Zeichen werden durch Tabellen in natürliche Zahlen übersetzt:
	\subsubsection*{ASCII-Code (8 Bit)}
	\begin{tabular}{ccc}
		'A' $ = 65 = 0x41$ & 'B' $ = 66 = 0x42$ & 'C' $ = 67 = 0x43$\\
		'a' $ = 97 = 0x61$ & 'b' $ = 98 = 0x62$ & 'c' $ = 99 = 0x63$\\
		'0' $ = 48 = 0x30$ & '1' $ = 49 = 0x31$ & '2' $ = 50 = 0x32$\\
		'\textvisiblespace' $ = 32 = 0x20$ & '@' $ = 64 = 0x40$
	\end{tabular}
	\subsubsection*{Unicode (UTF-8, variable Länge mit Blöcken zu je 8 Bit)}
	\begin{itemize}
		\item 1 Byte: 0xxx~xxxx = 7 Bit ASCII
		\item 2 Bytes: 110x~xxxx~10xx~xxxx = 11 Bit Codierung
		\item 3 Bytes: 1110 xxxx 10xx xxxx 10xx xxxx = 16 Bit Codierung
		\item \ldots
	\end{itemize}
	
	\subsection{Exkurs: Rechnen mit Folgen und Reihen}
	\begin{bemerkung}
		Obwohl der Computer endlich ist, lassen sich (gewisse) unendliche Mengen (und Funktionen auf unendliche Mengen etc.) explizit darstellen.
	\end{bemerkung}	
	\begin{example}~
		\begin{lstlisting}[language=Python]
			
			def a(n):
				return 1/n
		\end{lstlisting}
		ist die Codierung einer Folge reeller Zahlen (im Rahmen der Maschinengenauigkeit).
	\end{example}
	\begin{bemerkung}
		Lambda-Funktionen kommen ursprünglich aus der funktionalen Programmierung. Eben genannte Folge kann man in Python auch schreiben als
		\begin{lstlisting}[language=Python]
			
			a = lambda n: 1/n
		\end{lstlisting}
		\begin{itemize}
			\item Die Nutzung ist gleich
			\item Kein return-Statement
		\end{itemize}
		\begin{lstlisting}[language=Python]
			
			print(list(map(a, [1,2,4,8])))
			>>> [1, 0.5, 0.25, 0.125]
		\end{lstlisting}
	\end{bemerkung}
	\begin{itemize}[label=]
		\item Addition:
		\begin{lstlisting}[language=Python]
			
			def add(a, b):
				return lambda n: a(n) + b(n)
		\end{lstlisting}
		
		\item Subtraktion:
		\begin{lstlisting}[language=Python]
			
			def sub(a,b):
				return lambda n: a(n) - b(n)
		\end{lstlisting}
		
		\item Reihen:
		\begin{lstlisting}[language=Python]
			
			def finite_subsequence(a, k):
				return [a(n) for n in range(0, k)]
			
			def series(a):
				return lambda n: sum(finite_subsequence(a, n))
		\end{lstlisting}
	\end{itemize}
	\subsection{Exkurs: Riemannsche Zeta-Funktion}
	Für komplexe Zahlen $s = \sigma \cdot \tau i \in \mathbb{C}$ mit $\sigma > 1$ konvergiert Dirichlet-Reihe
	\[
		\zeta(s) = \sum_{n = 0}^{\infty} \frac{1}{n^s} = 1 + \frac{1}{2^s} + \frac{1}{3^s} + \frac{1}{4^s} + \ldots
	\]
	Die Riemannsche Zeta-Funktion ist die eindeutige analytische Fortsetzung auf ganz $\C$.
	Die Riemannsche Vermutung ist: Alle Nullstellen (mit $\sigma > 0$) erfüllen $\sigma = \frac{1}{2}$.
	\begin{lstlisting}[language=Python]
		
		def zeta(s):
			return series(lambda n: 1/(pow(n+1, z)))
	\end{lstlisting}
	\begin{satz}[Riemannsches Umordnungsgesetz]
		Es seien $a_n \in \R$ für $n \in \N$ und $\sum_{n = 0}^{\infty}a_n$ sei konvergent, aber nicht absolut konvergent. Weiter sei $x \in \R$ beliebig. Dann existiert eine bijektive Abbildung $\phi: \N \to \N$ so, dass
		\[
			\sum_{n = 0}^{\infty} a_\phi(n) = x
		\]
	\end{satz}
	\begin{lstlisting}[language=Python, caption={Riemann-Reordering}]
		
		"""
			a is a series such that the series of partial sums converges, but not absolutely. 
			x is the float to converge to for the reordering.
			returns first k indices of reordering bijection phi
		"""
		def riemann(a, x, k):
			partialsum = 0
			i = 0
			ipos, ineg = (0,0)
			reordering = []
			for j in range(0, k):
				if(partialsum < x):
					i = next_positive_index(a, ipos)
					ipos = i + 1
				else:
					i = next_negative_index(a, ineg)
					ineg = i + 1
				reordering.append(i)
				partialsums += a(i)
			return reordering
	\end{lstlisting}
	\section{Graphen}
	\begin{definition}
		Ein ungerichteter Graph $G$ ist ein Tripel $G = (V, E, \Psi)$, wobei $V$ und $E$ endliche Mengen sind, $V \neq 0$ und
		\[
			\Psi: E \to \{X \mid X \subseteq V, \lvert X \rvert = 2\}
		\]
		Elemente in $V = V(G)$ heißen Knoten. Elemente in $E = E(G)$ heißen Kanten.
	\end{definition}
	\begin{definition}
		Ein gerichteter Graph (oder Digraph) $G$ ist ein Tripel $G = (V, E, \Psi)$, wobei $V$ und $E$ endliche Mengen sind, $V \neq 0$ und
		\[
			\Psi: E \to \{(v,w) \in V \times V \mid v \neq w\}
		\]
	\end{definition}
	\subsection{Terminologie für Graphen}
	\begin{definition}
		Auf einem gerichteten oder ungerichteten Graphen verbindet die Kante $e = \{v, w\}$ bzw. $e = (v, w)$ die Knoten $v$ und $w$. Dann heißen die Knoten $v$ und $w$ benachbart oder {\itshape adjazent}. Sie sind Endknoten der Kante $e$ und mit dieser {\itshape inzident}.
		
		Zwei Kanten $e, e' \in E$ heißen parallel, falls $\Psi(e) = \Psi(e')$.
	\end{definition}
	\begin{definition}
		Auf einem gerichteten oder ungerichteten Graphen $G = (V, E, \Psi), v \in V$
		\begin{itemize}
			\item Ist $G$ ungerichtet, so ist
			\[
				\delta(v) \coloneq \{e \in E \mid v \in \Psi(e)\}
			\]
			die Menge der zu $v$ inzidenten Kanten.
			\item Der Grad eines Knotens $v$ ist $\lvert \delta(v) \rvert$.
			\item Ist $G$ ungerichtet, so ist
			\begin{align*}
				\delta^+(v) &\coloneq \{e \in E \mid \exists w \in V: \Psi(e) = (v, w)\}\\
				\delta^-(v) &\coloneq \{e \in E \mid \exists w \in V: \Psi(e) = (w, v)\}\\
				\delta(v) &\coloneq \delta^+(v) \cup \delta^-(v)
			\end{align*}
			\item Der {\itshape Eingangsgrad} von $v$ ist $\lvert \delta^-(v) \rvert$.
			\item Der {\itshape Ausgangsgrad} von $v$ ist $\lvert \delta^+(v) \rvert$.
		\end{itemize}
	\end{definition}
	\begin{definition}
		Ein gerichteter oder ungerichteter Graph ohne parallele Kanten wird {\itshape einfach} genannt.
		\begin{itemize}
			\item Für einfache Graphen identifiziert man $e \in E$ mit $\Psi(e)$ und schreibt
			\begin{align*}
				G = (V, E) \text{ mit } E &\subseteq \{\{v, w\} \mid v, w \in V, v \neq w\}\\
				\text{ bzw } E &\subseteq \{(v, w) \mid v, w \in V, v \neq w\}
			\end{align*}
			\item Insbesondere ist dann die Kantenmenge $E$ eines einfachen Graphen eine Relation auf $V$ (d. h. auf $V \times V$).
			\item Die Kantenmenge $E$ eines ungerichteten Graphen kann als symmetrische Relation $R_E$ auf $V$ aufgefasst werden:
			\[
				(v, w) \in R_E :\Longleftrightarrow \{v, w\} \in E
			\]
		\end{itemize}
	\end{definition}
	\subsection{Handschlaglemma}
	\begin{lemma}
		Es sei $G = (V, E, \Psi)$ ungerichteter oder gerichteter Graph.
		\begin{enumerate}[label=(\alph*)]
			\item Die Anzahl Knoten ungeraden Grades ist gerade, denn
			\[
				\sum_{v \in V} \lvert \delta(v) \rvert = 2 \lvert E \rvert
			\]
			\item Ist $G$ gerichtet, dann gilt:
			\[
				\sum_{v \in V} \lvert \delta^+(v) \rvert = \lvert E \rvert = \sum_{v \in V} \lvert \delta^-(v) \rvert
			\]
		\end{enumerate}
	\end{lemma}
	\begin{proof}
		\begin{enumerate}[label=(\alph*)]
			\item 
			\begin{align*}
				\sum_{v \in V} \lvert \delta(v) \rvert &= \sum_{v \in V} \big\lvert \{e \in E \mid \Psi(e) = \{v, w\} \text{ für ein } w \in V\} \big\rvert\\
				&= \sum_{e \in E}2  = 2 \lvert E \rvert 
			\end{align*}
			\item 
			\begin{align*}
				\sum_{v \in V}\lvert \delta^+(v) \rvert &= \sum_{v \in V} \lvert \{e \in E \mid \Psi(e) = (v, w) \text{ für ein } w \in V\}\rvert\\
				&= \sum_{e \in E} 1 = \lvert E \rvert = \sum_{v \in V}\lvert \delta^-(v) \rvert
			\end{align*}
		\end{enumerate}
	\end{proof}
	\begin{definition}
		Ein ungerichteter Graph heißt regulär (oder $d$-regulär), falls jeder Knoten Grad $d$ besitzt.
	\end{definition}
	\subsection{Teilgraphen, Wege, Kreise}
	\begin{definition}
		Sei $G = (V(G), E(G), \Psi_G)$ ein ungerichteter oder gerichteter Graph.
		\begin{enumerate}[label=(\alph*)]
			\item Graph $H = (V(H), E(H), \Psi_H)$ ist {\itshape Teilgraph} von $G$, falls
			\[
				V(H) \subseteq V(G), E(H) \subseteq E(G), \Psi_H = \Psi_{G \mid E(H)}
			\]
			\item Ist zusätzlich $V(H) = V(G)$, so ist $H$ {\itshape aufgespannter Teilgraph}.
			\item $H$ ist induzierter Teilgraph, falls
			\begin{align*}
				E(H) &= \{e \in E(G) \mid \Psi_G(e) = \{v, w\}: v, w \in V(H)\}\\
				\text{bzw. } E(H) &= \{e \in E(G) \mid \Psi_G(e) = (v, w): v, w \in V(H)\}
			\end{align*}
			In diesem Falle schreiben wir auch $H = G[V(H)]$.
		\end{enumerate}
	\end{definition}
\end{document}