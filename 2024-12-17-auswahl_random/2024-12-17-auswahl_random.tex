\documentclass[a4paper,12pt]{article}
\usepackage{amsmath}
\usepackage{amssymb}
\usepackage{amsthm}
\usepackage{marvosym}
\usepackage[utf8]{inputenc}
\usepackage[T1]{fontenc}
\usepackage[ngerman]{babel}
\usepackage[left=2.5cm,right=2cm,top=3cm,bottom=3cm]{geometry}
\usepackage{fancyhdr}
\usepackage{tabularx}
\usepackage{booktabs}
\usepackage{amsfonts}
\usepackage{tcolorbox}
\usepackage{adjustbox}
\usepackage{eurosym}
\usepackage[inline]{enumitem}
\usepackage{lastpage}
\usepackage{mathtools}
\usepackage{microtype}
\usepackage[ruled, linesnumbered, noline]{algorithm2e}
\usepackage{listings}
\usepackage{xcolor}

\setlength{\parindent}{0pt}

\usepackage{tikz}
\usetikzlibrary{decorations.pathmorphing,calc,shadows.blur,shadings,er,positioning}

\definecolor{codegreen}{rgb}{0,0.6,0}
\definecolor{codegray}{rgb}{0.5,0.5,0.5}
\definecolor{codepurple}{rgb}{0.58,0,0.82}
\definecolor{backcolour}{rgb}{0.95,0.95,0.92}

\lstdefinestyle{mystyle}{
	backgroundcolor=\color{backcolour},   
	commentstyle=\color{codegreen},
	keywordstyle=\color{magenta},
	numberstyle=\tiny\color{codegray},
	stringstyle=\color{codepurple},
	basicstyle=\ttfamily\footnotesize,
	breakatwhitespace=false,         
	breaklines=true,                 
	captionpos=b,                    
	keepspaces=true,                 
	numbers=left,                    
	numbersep=5pt,                  
	showspaces=false,                
	showstringspaces=false,
	showtabs=false,                  
	tabsize=2
}

\lstset{style=mystyle}
%\usepackage{pgfplots}
%\usepackage[leqno]{amsmath}

%\newcounter{mathseed}
%\setcounter{mathseed}{1}
%\pgfmathsetseed{\arabic{mathseed}} 
%\pgfdeclarelayer{background}
%\pgfsetlayers{background,main}

\newcommand{\RN}[1]{%
	\textup{\uppercase\expandafter{\romannumeral#1}}%
}

\newcommand{\obda}{o.B.d.A. }

\newenvironment{enum}{\begin{enumerate*}[label=\alph*),itemjoin=\hspace{2em}]}{\end{enumerate*}\\[2ex]}

\DeclareMathOperator{\N}{\mathbb N}
\DeclareMathOperator{\Q}{\mathbb Q}
\DeclareMathOperator{\R}{\mathbb R}
\DeclareMathOperator{\Z}{\mathbb Z}
\DeclareMathOperator{\C}{\mathbb C}
\DeclareMathOperator{\F}{\mathcal{F}}
\DeclareMathOperator{\E}{\mathcal{E}}
\DeclareMathOperator{\BigO}{\mathcal O}
\DeclareMathOperator{\mL}{\mathcal{L}}
\DeclareMathOperator{\mU}{\mathcal{U}}
\DeclareMathOperator{\ggt}{\text{ggT}}
\DeclareMathOperator{\kgv}{\text{kgV}}


%------ Defining multiple types of theorems ------%
\newtheorem{axiom}{Axiom}[section]
\newtheorem{theorem}[axiom]{Theorem}
\newtheorem{lemma}[axiom]{Lemma}
\newtheorem{satz}[axiom]{Satz}
\theoremstyle{definition}
\newtheorem*{example}{Beispiel}
\newtheorem*{bemerkung}{Bemerkung}
\newtheorem*{frage}{Frage}
\newtheorem*{loesung}{Lösung}
\newtheorem*{behauptung}{Behauptung}
\newtheorem*{beobachtung}{Beobachtung}
\newtheorem{definition}[axiom]{Definition}
\newtheorem{folg}[axiom]{Folgerung}
\newtheorem{notation}[axiom]{Notation}

%---- Changing forall and exists ----%
\let\oldforall\forall
\renewcommand{\forall}{\:\oldforall \, }
\let\oldexist\exists
\renewcommand{\exists}{\:\oldexist \: }
\newcommand\existu{\oldexist! \: }
\let\oldepsilon\epsilon
\renewcommand{\epsilon}{\varepsilon}
\newcommand{\corresponds}{\;\widehat{=}\;}

%------ Giving circled star ---------%
\makeatletter
\newcommand{\ostar}{\mathbin{\mathpalette\make@circled\star}}
\newcommand{\make@circled}[2]{%
	\ooalign{$\m@th#1\smallbigcirc{#1}$\cr\hidewidth$\m@th#1#2$\hidewidth\cr}%
}
\newcommand{\smallbigcirc}[1]{%
	\vcenter{\hbox{\scalebox{0.77778}{$\m@th#1\bigcirc$}}}%
}
\makeatother
%-----------------------------------%

%-------------- Vars ---------------%
\newcommand{\vldate}{17. Dezember 2024}
\newcommand{\vlname}{Computerorientierte Mathematik I}
\newcommand{\vltopic}{Das Auswahlproblem}

\pagestyle{fancy}
\setlength{\headheight}{34pt}
\fancyhf{}
\fancyhead[L]{\includegraphics[width=1cm]{C:/Users/chole/Documents/tub/tublogo.png}}
\fancyhead[C]{{\bfseries\vlname}\\%
	\vltopic}
\fancyhead[R]{\vldate}
\fancyfoot[L]{}
\fancyfoot[C]{- \thepage\ of \pageref{LastPage} -}
\fancyfoot[R]{}

\renewcommand{\headrulewidth}{1pt}
\renewcommand{\footrulewidth}{1pt}
\renewcommand{\labelitemii}{-}

\begin{document}
	\setcounter{section}{8}
	\setcounter{subsection}{14}
	\begin{bemerkung}
		Wählt man Pivor-Element $x$ in Linearzeit so, dass 
		\[
			\max\{\lvert A_{<x}\rvert, \lvert A_{>x}\rvert\} \leq \lfloor d \cdot \lvert A\rvert\rfloor
		\]
		für eine Konstante $d < 1$, so hat Auswahl$(a, i)$ Laufzeit $\BigO(n)$.
	\end{bemerkung}
	
	Denn für Laufzeitfunktion $f: \N \to \N$ gilt dann $f(n) \leq f(\lfloor d \cdot n\rfloor) + \tilde{c} \cdot n$ und damit $f(\lceil \frac{1}{d} \cdot n\rceil) \leq 1 \cdot f(n) + \frac{\tilde{c}}{d} \cdot n$. Aus dem Aufteilungs-Beschleunigungs-Satz folgt, dass $f(n) \in \BigO(n)$.
	
	\subsection{Randomisierter Linearzeit-Algorithmus für das Auswahlproblem.}
	\begin{satz}
		Wählt man in dem Auswahl-Algorthmus das Pivor-Element zufällig und gleichverteilt aus, so hat der Algorithmus Laufzeitfunktion in $\BigO(n)$.
	\end{satz}
	\begin{proof}
		Betrachte monoton wachsende Funktion $f: \N_{>0} \to \R$, die erwartete Laufzeit nach oben beschränkt. Nach Konstruktion des Algorithmus gibt es eine Konstante $c > 0$ mit $f(1) = 4c$ und
		\[
			f(n) \leq \frac{1}{n} \cdot \sum_{j = 1}^{n}f(\max\{j - 1, n - j\}) + c \cdot n \text{ für alle } n \in \N_{>1}.
 		\]
 		\begin{behauptung}
 			$f(n) \leq 4c \cdot n$ für alle $n \in \N_{>0}$.
 		\end{behauptung}
 		Induktionsanfang $(n = 1)$: $f(1) \leq 4c$.
 		
 		Induktionsschluss $(n \to n + 1)$:\\
 		Für $n > 1$ gilt:
 		\begin{align*}
 			f(n) &\leq \frac{1}{n} \cdot \sum_{j = 1}^{n}f(\max\{j - 1, n + j\}) + c \cdot n\\
 			&\leq \frac{2}{n} \cdot \sum_{j = 1}^{\lceil \frac{n}{2} \rceil}f(n - j) + c\cdot n\\
 			&\leq \frac{8c}{n} \cdot \sum_{j = 1}^{\lceil \frac{n}{2} \rceil}(n - j) + c\cdot n\\
 			&= \frac{8c}{n}\Big(\frac{n(n - 1)}{2} - \frac{\lfloor \frac{n}{2} \rfloor (\lfloor \frac{n}{2} \rfloor - 1)}{2}\Big) + c \cdot n\\
 			&\leq \frac{8c}{n}\Big(\frac{n(n - 1)}{2} - \frac{\frac{n - 1}{2}(\frac{n - 1}{2} - 1)}{2}\Big) + c\cdot n\\
 			&= \frac{8c}{n}\Big(\frac{n^2}{2} - \frac{n}{2} - \big(\frac{n^2}{8} - \frac{n}{2} + \frac{3}{8}\big)\Big) + c \cdot n\\
 			&\leq \frac{8c}{n}\Big(\frac{3}{8}n^2\Big) + c\cdot n\\
 			&= 4c \cdot n
 		\end{align*}
	\end{proof}
	\subsection{Deterministischer Linearzeit-Algorithmus für Auswahlproblem}
	Ersetze nun Schritt $1$ des Algorithmus durch folgende Routine:
	\begin{enumerate}[label=(\roman*)]
		\item Partitioniere $A$ in $\lfloor \frac{\lvert A\rvert}{5}\rfloor$ Teilmengen mit $5$ Elementen und eine Teilmenge mit den restlichen $n \mod 5$ Elementen $\longrightarrow$ Laufzeit $\BigO(\lvert A\rvert)$.
		\item Bestimme für jede dieser Teilmengen deren Median und bilde Menge $A_M$ dieser Mediane $\longrightarrow$ Laufzeit $\BigO(\lvert A\rvert)$.
		\item Berechne mittels rekursiven Aufrufs von Auswahl den Median $x$ von $A_M$ und verwende $x$ als Pivot-Element.
	\end{enumerate}
	\begin{bemerkung}
		~
		\begin{itemize}
			\item $\lvert A_M = \lceil \frac{n}{5} \rceil \rvert$
			\item $\lvert A_{>x}\rvert \geq 3\big(\big\lceil \frac{1}{2} \lceil \frac{n}{5}\rceil\big\rceil - 2\big) \geq \frac{3n}{10} - 6 \Longrightarrow \lvert A_{< x}\rvert \leq \lfloor \frac{7n}{10} + 6\rfloor$
			\item analog: $\lvert A_{< x}\rvert \geq \frac{3n}{10} - 6 \Longrightarrow \lvert A_{>x}\rvert \leq \lfloor \frac{7n}{10} + 6\rfloor$
		\end{itemize}
	\end{bemerkung}
	Folglich gilt für die monotone Laufzeitfunktion des rekursiven Algorithmus:
	\[
		f(n) \leq f\Big(\Big\lceil \frac{n}{5} \Big\rceil\Big) + f\Big(\Big\lfloor \frac{7n}{10} + 6\Big\rfloor\Big) + c\cdot n 
	\]
	für ein $c > 0$ und alle $n \in \N_{> 0}$.
	\begin{satz}
		Wählt man im Algorithmus \glqq Auswahl\grqq\; das Pivot-Element mit der oben beschriebenen rekursiven Routine, dann ist die Laufzeitfunktion in $\BigO(n)$.
	\end{satz}
	\begin{proof}
		Es gibt ein $c > 0$, sodass für die monotone Laufzeit gilt:
		\[
			f(n) \leq \begin{dcases*}
				20c & \text{ für alle } n < 140\\
				f\Big(\Big\lceil\frac{n}{5} \Big\rceil\Big) + f\Big(\Big\lfloor \frac{7n}{10}\Big\rfloor + 6\Big) + c\cdot n &\text{ sonst}
			\end{dcases*}
		\]
		Zeige mit vollständiger Induktion zweiter Art: $f(n) \leq 2ac \cdot n$ für alle $n \in \N$.
		
		Induktionsanfang $(n < 140)$: Klar.
		
		Induktionsschritt: $(n \geq 140)$: 
		\begin{align*}
			f(n) &\leq f\Big(\underbrace{\Big\lceil \frac{n}{5}\Big\rceil}_{< n}\Big) + f\Big(\underbrace{\Big\lfloor \frac{7n}{10}\Big\rfloor + 6}_{< n}\Big) + c\cdot n\\
			&\stackrel{\text{Ind.}}{\leq} 20c \cdot \Big(\frac{n}{5} + 1\Big) + 20c \cdot \Big(\frac{7n}{10} + 6\Big) + c\cdot n\\
			&= 19 c\cdot n + 140c\\
			&\leq 20c \cdot n
		\end{align*}
		Es gilt also: $f(n) \leq 20c \cdot n$ für alle $n \in \N$ und damit $f(n) \in \BigO(n)$.
	\end{proof}
\end{document}