\documentclass[a4paper,12pt]{article}
\usepackage{amsmath}
\usepackage{amssymb}
\usepackage{amsthm}
\usepackage{marvosym}
\usepackage[utf8]{inputenc}
\usepackage[T1]{fontenc}
\usepackage[ngerman]{babel}
\usepackage[left=2.5cm,right=2cm,top=3cm,bottom=3cm]{geometry}
\usepackage[dvipsnames]{xcolor}
\usepackage{fancyhdr}
\usepackage{tabularx}
\usepackage{booktabs}
\usepackage{amsfonts}
\usepackage{tcolorbox}
\usepackage{adjustbox}
\usepackage{eurosym}
\usepackage[inline]{enumitem}
\usepackage{lastpage}
\usepackage{mathtools}
\usepackage{microtype}
\usepackage[ruled, linesnumbered, noline]{algorithm2e}
\usepackage{listings}


\setlength{\parindent}{0pt}

\usepackage{tikz}
\usetikzlibrary{decorations.pathmorphing,calc,shadows.blur,shadings,er,positioning}

\definecolor{codegreen}{rgb}{0,0.6,0}
\definecolor{codegray}{rgb}{0.5,0.5,0.5}
\definecolor{codepurple}{rgb}{0.58,0,0.82}
\definecolor{backcolour}{rgb}{0.95,0.95,0.92}

\lstdefinestyle{mystyle}{
	backgroundcolor=\color{backcolour},   
	commentstyle=\color{magenta},
	keywordstyle=\color{NavyBlue},
	numberstyle=\tiny\color{codegray},
	stringstyle=\color{codepurple},
	basicstyle=\ttfamily\footnotesize,
	breakatwhitespace=false,         
	breaklines=true,                 
	captionpos=b,                    
	keepspaces=true,                 
	numbers=left,                    
	numbersep=5pt,                  
	showspaces=false,                
	showstringspaces=false,
	showtabs=false,                  
	tabsize=2
}

\lstset{style=mystyle}
%\usepackage{pgfplots}
%\usepackage[leqno]{amsmath}

%\newcounter{mathseed}
%\setcounter{mathseed}{1}
%\pgfmathsetseed{\arabic{mathseed}} 
%\pgfdeclarelayer{background}
%\pgfsetlayers{background,main}

\newcommand{\RN}[1]{%
	\textup{\uppercase\expandafter{\romannumeral#1}}%
}

\newcommand{\obda}{o.B.d.A. }

\newenvironment{enum}{\begin{enumerate*}[label=\alph*),itemjoin=\hspace{2em}]}{\end{enumerate*}\\[2ex]}

\DeclareMathOperator{\N}{\mathbb N}
\DeclareMathOperator{\Q}{\mathbb Q}
\DeclareMathOperator{\R}{\mathbb R}
\DeclareMathOperator{\Z}{\mathbb Z}
\DeclareMathOperator{\C}{\mathbb C}
\DeclareMathOperator{\F}{\mathcal{F}}
\DeclareMathOperator{\E}{\mathcal{E}}
\DeclareMathOperator{\BigO}{\mathcal O}
\DeclareMathOperator{\mL}{\mathcal{L}}
\DeclareMathOperator{\mU}{\mathcal{U}}
\DeclareMathOperator{\ggt}{\text{ggT}}
\DeclareMathOperator{\kgv}{\text{kgV}}


%------ Defining multiple types of theorems ------%
\newtheorem{axiom}{Axiom}[section]
\newtheorem{theorem}[axiom]{Theorem}
\newtheorem{lemma}[axiom]{Lemma}
\newtheorem{satz}[axiom]{Satz}
\theoremstyle{definition}
\newtheorem*{example}{Beispiel}
\newtheorem*{bemerkung}{Bemerkung}
\newtheorem*{frage}{Frage}
\newtheorem*{loesung}{Lösung}
\newtheorem*{behauptung}{Behauptung}
\newtheorem*{beobachtung}{Beobachtung}
\newtheorem{definition}[axiom]{Definition}
\newtheorem{folg}[axiom]{Folgerung}
\newtheorem{notation}[axiom]{Notation}

%---- Changing forall and exists ----%
\let\oldforall\forall
\renewcommand{\forall}{\:\oldforall \, }
\let\oldexist\exists
\renewcommand{\exists}{\:\oldexist \: }
\newcommand\existu{\oldexist! \: }
\let\oldepsilon\epsilon
\renewcommand{\epsilon}{\varepsilon}
\newcommand{\corresponds}{\;\widehat{=}\;}

%------ Giving circled star ---------%
\makeatletter
\newcommand{\ostar}{\mathbin{\mathpalette\make@circled\star}}
\newcommand{\make@circled}[2]{%
	\ooalign{$\m@th#1\smallbigcirc{#1}$\cr\hidewidth$\m@th#1#2$\hidewidth\cr}%
}
\newcommand{\smallbigcirc}[1]{%
	\vcenter{\hbox{\scalebox{0.77778}{$\m@th#1\bigcirc$}}}%
}
\makeatother
%-----------------------------------%

%-------------- Vars ---------------%
\newcommand{\vldate}{12. Dezember 2024}
\newcommand{\vlname}{Computerorientierte Mathematik I}
\newcommand{\vltopic}{Laufzeitbeschleunigungssatz}

\pagestyle{fancy}
\setlength{\headheight}{34pt}
\fancyhf{}
\fancyhead[L]{\includegraphics[width=1cm]{C:/Users/chole/Documents/tub/tublogo.png}}
\fancyhead[C]{{\bfseries\vlname}\\%
	\vltopic}
\fancyhead[R]{\vldate}
\fancyfoot[L]{}
\fancyfoot[C]{- \thepage\ of \pageref{LastPage} -}
\fancyfoot[R]{}

\renewcommand{\headrulewidth}{1pt}
\renewcommand{\footrulewidth}{1pt}
\renewcommand{\labelitemii}{-}

\begin{document}
	\setcounter{section}{8}
	\setcounter{subsection}{5}
	\subsection{Der Laufzeitbeschleunigungssatz}
	Oft hilft bei der Laufzeitanalyse rekursiver Funktionen folgender Satz:
	\begin{satz}
		Es sei $f: \N \to \N$ eine monoton wachsende Funktion, $a \in \N_{>1}, b, c \in \R_{> 0}$, so dass $f(1) \leq \frac{c}{2}$ und die folgende Ungleichung gilt:
		\[
			f(a \cdot n) \leq b \cdot f(n) + c\cdot n \text{ für alle } n \geq 1.
		\]
		Dann gilt:
		\[
			f(n) \in
			\begin{dcases*}
				\BigO(n) &\text{ falls } a > b\\
				\BigO(n \log n) &\text{ falls } a = b\\
				\BigO(n^{\log_2b}) &\text{ falls } a < b
			\end{dcases*}
		\]
	\end{satz}
	\begin{example}
		\begin{itemize}
			\item Karazubas Multiplikation: $a = 2, b = 3 \implies \in \BigO(n^{\log_23})$
			\item MergeSort: $a = 2, b = 2 \implies \in \BigO(n \log n)$ 
		\end{itemize}
	\end{example}
	\begin{proof}
		Zeige zunächst: \[
			f(a^q) \leq c \cdot a^{q - 1} \cdot \sum_{i = 0}^{q}\big(\frac{b}{a}\big)^i
		\]
		Induktionsanfang $(q = 0)$: 
		\begin{align*}
			f(a^0) = f(1) &\stackrel{\tiny !}{\leq} \frac{c}{a} \leq c \cdot a^{0 - 1}\cdot \sum_{i = 0}^{0}\big(\frac{b}{a}\big)^i\\
			&\leq \frac{c}{a}
		\end{align*}
		Induktionsschluss $(q \to q + 1)$:
		\begin{align*}
			f(a^{q + 1}) &= f(a \cdot a^q)\\
			&\leq b \cdot f(a^q) + c\cdot a^q\\
			&\leq b \cdot c \cdot a^{q - 1} \cdot \sum_{i = 0}^{q}\big(\frac{b}{a}\big)^i + c\cdot a^q\\
			&= c \cdot a^q \cdot \sum_{i = 0}^{q} \big(\frac{b}{a}\big)^{i + 1} + c \cdot a^q\\
			&= c \cdot a^q\Big(\sum_{i = 1}^{q + 1}\big(\frac{b}{a}\big)^i + 1\Big)\\
			&= c \cdot a^q\sum_{i = 0}^{q + 1}\big(\frac{b}{a}\big)^i
		\end{align*}
		Wir haben gezeigt:
		\[
			f(a^q) \leq c \cdot a^{q - 1}\cdot \sum_{i = 0}^{q}\big(\frac{b}{a}\big)^i \text{ für alle } q \in \N
		\]
		Folglich gilt:
		\[
			f(n) \leq f(a^{\lceil \log_a n\rceil}) \leq c \cdot n \cdot \sum_{i = 0}^{\lceil \log_a n\rceil}\big(\frac{b}{a}\big)^i \text{ für alle } n \in \N_{>0}
		\]
		\subsubsection*{1. Fall, $a > b$}
		\begin{align*}
			f(n) &\leq c \cdot n \cdot \sum_{i = 0}^{\lceil \log_a n\rceil}\big(\frac{b}{a}\big)^i\\
			&\leq c \cdot n \cdot \sum_{i = 0}^{\infty}\big(\frac{b}{a}\big)^i\\
			&= \frac{c\cdot n}{1 - \frac{b}{a}} \in \BigO(n)
		\end{align*}
		\subsubsection*{2. Fall, $a = b$}
		\begin{align*}
			f(n) & \leq c \cdot n \cdot \sum_{i = 0}^{\lceil \log_a n\rceil}1\\
			&\leq c \cdot n \cdot (\log_a n + 2) \in \BigO(n \log n)
		\end{align*}
		\subsubsection*{3. Fall, $a < b$}
		\begin{align*}
			f(n) &\leq c \cdot n \cdot \sum_{i = 0}^{\lceil \log_a n\rceil}\big(\frac{b}{a}\big)^i\\
			&= c \cdot n \cdot \frac{(\frac{b}{a})^{\lceil \log_a n\rceil + 1} - 1}{\frac{b}{a} - 1}\\
			&\leq \frac{c}{\frac{b}{a} - 1} \cdot n \cdot \big(\frac{b}{a}\big)^{\log_a n + 2}\\
			&= \frac{c}{\frac{b}{a} - 1} \cdot \big(\frac{b}{a}\big)^2 \cdot n \cdot \frac{b^{\log_a n}}{a^{\log_a n}}\\
			&= \frac{c}{\frac{b}{a} - 1} \cdot \big(\frac{b}{a}\big)^2 \cdot b^{\log_a n} \in \BigO(n^{\log_a b})
		\end{align*}
	\end{proof}
	\subsection{Untere Schranke für Sortieren}
	\begin{satz}
		Jeder auf paarweisen Vergleichen basierende Algorithmus benötigt zum Sortieren einer $n$-elementigen Menge im Worst-Case $\Omega(n \log n)$ Vergleiche.
	\end{satz}
	\begin{bemerkung}
		\begin{itemize}
			\item Der Satz gilt zunächst nur für deterministische (im Vergleich zu randomisierten) Algorithmen.
			\item Man kann sogar zeigen, dass die untere Schranke nicht nur im Worst-Case, sondern sogar im Mittel gilt.
			\item MergeSort ist also mit Blick auf die asymptotische Laufzeit ein bestmöglicher vergleichsbasierter Sortieralgorithmus.
			\item Es gibt Sortieralgorithmen, die nicht vergleichsbasiert sind und bessere Laufzeit erzielen (bspw. BucketSort, RadixSort).
		\end{itemize}
	\end{bemerkung}
\end{document}