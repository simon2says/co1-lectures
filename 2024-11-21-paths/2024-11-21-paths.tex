\documentclass[a4paper,12pt]{article}
\usepackage{amsmath}
\usepackage{amssymb}
\usepackage{amsthm}
\usepackage{marvosym}
\usepackage[utf8]{inputenc}
\usepackage[T1]{fontenc}
\usepackage[ngerman]{babel}
\usepackage[left=2.5cm,right=2cm,top=3cm,bottom=3cm]{geometry}
\usepackage{fancyhdr}
\usepackage{tabularx}
\usepackage{booktabs}
\usepackage{amsfonts}
\usepackage{tcolorbox}
\usepackage{adjustbox}
\usepackage{eurosym}
\usepackage[inline]{enumitem}
\usepackage{lastpage}
\usepackage{mathtools}
\usepackage{microtype}
\usepackage[ruled, linesnumbered, noline]{algorithm2e}
\usepackage{listings}
\usepackage{xcolor}

\setlength{\parindent}{0pt}

\usepackage{tikz}
\usetikzlibrary{decorations.pathmorphing,calc,shadows.blur,shadings,er,positioning}

\definecolor{codegreen}{rgb}{0,0.6,0}
\definecolor{codegray}{rgb}{0.5,0.5,0.5}
\definecolor{codepurple}{rgb}{0.58,0,0.82}
\definecolor{backcolour}{rgb}{0.95,0.95,0.92}

\lstdefinestyle{mystyle}{
	backgroundcolor=\color{backcolour},   
	commentstyle=\color{codegreen},
	keywordstyle=\color{magenta},
	numberstyle=\tiny\color{codegray},
	stringstyle=\color{codepurple},
	basicstyle=\ttfamily\footnotesize,
	breakatwhitespace=false,         
	breaklines=true,                 
	captionpos=b,                    
	keepspaces=true,                 
	numbers=left,                    
	numbersep=5pt,                  
	showspaces=false,                
	showstringspaces=false,
	showtabs=false,                  
	tabsize=2
}

\lstset{style=mystyle}
%\usepackage{pgfplots}
%\usepackage[leqno]{amsmath}

%\newcounter{mathseed}
%\setcounter{mathseed}{1}
%\pgfmathsetseed{\arabic{mathseed}} 
%\pgfdeclarelayer{background}
%\pgfsetlayers{background,main}

\newcommand{\RN}[1]{%
	\textup{\uppercase\expandafter{\romannumeral#1}}%
}

\newcommand{\obda}{o.B.d.A. }

\newenvironment{enum}{\begin{enumerate*}[label=\alph*),itemjoin=\hspace{2em}]}{\end{enumerate*}\\[2ex]}

\DeclareMathOperator{\N}{\mathbb N}
\DeclareMathOperator{\Q}{\mathbb Q}
\DeclareMathOperator{\R}{\mathbb R}
\DeclareMathOperator{\Z}{\mathbb Z}
\DeclareMathOperator{\C}{\mathbb C}
\DeclareMathOperator{\F}{\mathcal{F}}
\DeclareMathOperator{\E}{\mathcal{E}}
\DeclareMathOperator{\BigO}{\mathcal O}
\DeclareMathOperator{\mL}{\mathcal{L}}
\DeclareMathOperator{\mU}{\mathcal{U}}
\DeclareMathOperator{\ggt}{\text{ggT}}
\DeclareMathOperator{\kgv}{\text{kgV}}


%------ Defining multiple types of theorems ------%
\newtheorem{axiom}{Axiom}[section]
\newtheorem{theorem}[axiom]{Theorem}
\newtheorem{lemma}[axiom]{Lemma}
\newtheorem{satz}[axiom]{Satz}
\theoremstyle{definition}
\newtheorem*{example}{Beispiel}
\newtheorem*{bemerkung}{Bemerkung}
\newtheorem*{frage}{Frage}
\newtheorem*{loesung}{Lösung}
\newtheorem*{behauptung}{Behauptung}
\newtheorem*{beobachtung}{Beobachtung}
\newtheorem{definition}[axiom]{Definition}
\newtheorem{folg}[axiom]{Folgerung}
\newtheorem{notation}[axiom]{Notation}

%---- Changing forall and exists ----%
\let\oldforall\forall
\renewcommand{\forall}{\:\oldforall \, }
\let\oldexist\exists
\renewcommand{\exists}{\:\oldexist \: }
\newcommand\existu{\oldexist! \: }
\let\oldepsilon\epsilon
\renewcommand{\epsilon}{\varepsilon}
\newcommand{\corresponds}{\;\widehat{=}\;}

%------ Giving circled star ---------%
\makeatletter
\newcommand{\ostar}{\mathbin{\mathpalette\make@circled\star}}
\newcommand{\make@circled}[2]{%
	\ooalign{$\m@th#1\smallbigcirc{#1}$\cr\hidewidth$\m@th#1#2$\hidewidth\cr}%
}
\newcommand{\smallbigcirc}[1]{%
	\vcenter{\hbox{\scalebox{0.77778}{$\m@th#1\bigcirc$}}}%
}
\makeatother
%-----------------------------------%

%-------------- Vars ---------------%
\newcommand{\vldate}{21. November 2024}
\newcommand{\vlname}{Computerorientierte Mathematik I}
\newcommand{\vltopic}{Kantenzüge und Wege}

\pagestyle{fancy}
\setlength{\headheight}{34pt}
\fancyhf{}
\fancyhead[L]{\includegraphics[width=1cm]{C:/Users/chole/Documents/tub/tublogo.png}}
\fancyhead[C]{{\bfseries\vlname}\\%
	\vltopic}
\fancyhead[R]{\vldate}
\fancyfoot[L]{}
\fancyfoot[C]{- \thepage\ of \pageref{LastPage} -}
\fancyfoot[R]{}

\renewcommand{\headrulewidth}{1pt}
\renewcommand{\footrulewidth}{1pt}
\renewcommand{\labelitemii}{-}

\begin{document}
	\setcounter{section}{6}
	\setcounter{subsection}{3}
	\begin{definition}
		Es sei $G = (V, E, \Psi)$ ein Graph (ungerichtet oder gerichtet). Ein Kantenzug in $G$ ist eine Folge
		\[
			v_1,e_1,v_2,e_2,\ldots ,v_k,e_k,v_{k+1}, k \in \N
		\]
		mit $v_1\ldots v_{k+1} \in V, e_i \in E$ und $\Psi(e_i) = \{v_i, v_{i+1}\}$ bzw. $\Psi(e_i) = (v_i, v_{i+1})$ für $i = 1, \ldots, k$.
	\end{definition}
	Die Länge eines Kantenzugs ist die Anzahl $k$ der enthaltenen Kanten.
	\begin{definition}
		Ein Weg ist ein Kantenzug $v_1,e_1,v_2,e_2,\ldots ,v_k,e_k,v_{k+1}$, so dass $v_i \neq v_j$ für $1 \leq i < j \leq k+1$.
	\end{definition}
	Um Anfangs- und Endknoten des Weges hervorzuheben, sprechen wir auch von einem $v_1,v_{k+1}$-Weg. Ein Kantenzug der Länge $0$ besteht aus eienm Knoten ist ist ein Weg. Ein Kreis ist ein geschlossener Kantenzug, sodass $v_i \neq v_j$ für $1\leq i < j \leq k$.
	\subsection{Zusammenhang und Zusammenhangskomponenten}
	\begin{definition}
		Ein ungerichteter Graph $G = (V, E, \Psi)$ heißt zusammenhängend, falls es für je zwei Knoten $v, w \in V$ einen $v, w$-Weg in $G$ gibt. Sonst heißt $G$ unzusammenhängend.
	\end{definition}
	\begin{lemma}
		Es gibt genau dann einen $v, w$-Weg in $G$, wenn es einen Kantenzug von $v$ nach $w$ gibt.
	\end{lemma}
	\begin{proof}
		$\Longrightarrow:$ Jeder $v$-$w$-Weg ist ein Kantenzug von $v$ nach $w$.
		
		$\Longleftarrow:$ Ein Kantenzug von $v$ nach $w$ wird durch Abkürzen zu einem $v$-$w$-Weg.
	\end{proof}
	Für einen ungerichteten Graphen $G = (V, E, \Psi)$ betrachten wir die Relation $R \coloneq \{(u, v) \in V \times V \mid \exists u, v\text{-Weg in }G\}$.
	\begin{lemma}
		Die Relation $R \subseteq V \times V$ ist
		\begin{enumerate}[label=(\roman*)]
			\item reflexiv, d. h. $(u, u) \in R$ für alle $u \in V$
			\item symmetrisch, d. h. für alle $u, v$ gilt: $(u, v) \in R \implies (v, u) \in R$
			\item transitiv, d. h. für alle $u, v, w \in V$ gilt: $(u, v) \in R \land (v, w) \in R \implies (u, w) \in R$ 
		\end{enumerate}
		Folglich ist $R$ eine Äquivalenzrelation. Die zugehörigen Äquivalenzklassen bilden also eine Partition der Knotenmenge $V$.
	\end{lemma}
	\begin{definition}
		Ein von einer Äquivalenzklasse $U \subseteq V$ induzierter Teilgraph $G[U]$ heißt {\itshape Zusammenhangskomponente} von $G$.
	\end{definition}
	\begin{definition}
		Es sei $G = (V, E, \Psi)$ ein gerichteter Graph.
		\begin{enumerate}[label=(\roman*)]
			\item Der zu Grunde liegende ungerichtete Graph ist $G' = (V, E, \Psi')$, wobei
			\[
				\Psi'(e) = \{u, v\} \text{ mit } \Psi(e) = (u, v) \text{ für } e \in E
			\]
			\item $G$ heißt zusammenhängend, falls $G'$ zusammenhängend ist.
			\item $G$ heißt stark zusammenhängend, falls es für jedes Knotenpaar $u, v \in V$ sowohl einen $u$-$v$-Weg, als auch einen $v$-$u$-Weg in $G$ gibt.
		\end{enumerate}
	\end{definition}
	\subsection{Eulersche Graphen}
	\begin{definition}
		Es sei $G = (V, E, \Psi)$ ein ungerichteter Graph.
		\begin{enumerate}[label=(\alph*)]
			\item Eine Eulertour in $G$ ist ein geschlossener Kantenzug, der jede Kante in $E$ genau ein Mal enthält.
			\item $G$ heißt Eulersch, falls jeder Knoten in $V$ geraden Grad besitzt.
		\end{enumerate}
	\end{definition}
	\begin{beobachtung}
		Besitzt G eine Eulertour, so ist $G$ eulersch.
	\end{beobachtung}
	\begin{lemma}
		Die Kantenmenge eines eulerschen Graphen zerfällt in kantendisjunkte Kreise.
	\end{lemma}
	\begin{proof}[Beweis durch vollständige Induktion zweiter Art.]
		Induktionsanfang $(\lvert E\rvert = 0)$: klar.
		
		Induktionsschluss: Für ein beliebiges, fest gewähltes $m$ gelte das Lemma für alle $\lvert E\rvert \leq m$. Betrachte Graph $G = (V, E, \Psi)$ mit $\lvert E\rvert = m + 1$.
		
		Behauptung: $G$ enthält einen Kreis $C$. Beginne mit Kantenzug $v_1e_1v_2$ für eine Kante $e_1 \in E(G)$. Da $\lvert \delta(v_2)\rvert$ gerade ist $\implies\exists e_2 \in \delta(v_2)\setminus\{e_1\} \implies$ Kantenzug $v_1e_1v_2e_2v_3$. Ist $v_3 = v_1$, so haben wir $C$ gefunden. Sonst:
		
		Da $\lvert \delta(v_3)\rvert$ gerade ist $\implies \exists e_3 \in \delta(v_3) \setminus\{e_2\} \implies$ Kantenzug $v_1e_1v_2e_2v_3e_3v_4$. Ist $v_4 \in \{v_1, v_2\}$, haben wir $C$ gefunden. Sonst \dots
		
		Da $E(G)$ endlich ist, terminiert der beschriebene Prozess mit einem Kreis.  
	\end{proof}
\end{document}