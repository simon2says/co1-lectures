\documentclass[a4paper,12pt]{article}
\usepackage{amsmath}
\usepackage{amssymb}
\usepackage{amsthm}
\usepackage{marvosym}
\usepackage[utf8]{inputenc}
\usepackage[T1]{fontenc}
\usepackage[ngerman]{babel}
\usepackage[left=2.5cm,right=2cm,top=3cm,bottom=3cm]{geometry}
\usepackage{fancyhdr}
\usepackage{tabularx}
\usepackage{booktabs}
\usepackage{amsfonts}
\usepackage{tcolorbox}
\usepackage{adjustbox}
\usepackage{eurosym}
\usepackage[inline]{enumitem}
\usepackage{lastpage}
\usepackage{mathtools}
\usepackage{microtype}
\usepackage[ruled, linesnumbered, noline]{algorithm2e}
\usepackage{listings}
\usepackage{xcolor}

\setlength{\parindent}{0pt}

\usepackage{tikz}
\usetikzlibrary{decorations.pathmorphing,calc,shadows.blur,shadings,er,positioning}

\definecolor{codegreen}{rgb}{0,0.6,0}
\definecolor{codegray}{rgb}{0.5,0.5,0.5}
\definecolor{codepurple}{rgb}{0.58,0,0.82}
\definecolor{backcolour}{rgb}{0.95,0.95,0.92}

\lstdefinestyle{mystyle}{
	backgroundcolor=\color{backcolour},   
	commentstyle=\color{codegreen},
	keywordstyle=\color{magenta},
	numberstyle=\tiny\color{codegray},
	stringstyle=\color{codepurple},
	basicstyle=\ttfamily\footnotesize,
	breakatwhitespace=false,         
	breaklines=true,                 
	captionpos=b,                    
	keepspaces=true,                 
	numbers=left,                    
	numbersep=5pt,                  
	showspaces=false,                
	showstringspaces=false,
	showtabs=false,                  
	tabsize=2
}

\lstset{style=mystyle}
%\usepackage{pgfplots}
%\usepackage[leqno]{amsmath}

%\newcounter{mathseed}
%\setcounter{mathseed}{1}
%\pgfmathsetseed{\arabic{mathseed}} 
%\pgfdeclarelayer{background}
%\pgfsetlayers{background,main}

\newcommand{\RN}[1]{%
	\textup{\uppercase\expandafter{\romannumeral#1}}%
}

\newcommand{\obda}{o.B.d.A. }

\newenvironment{enum}{\begin{enumerate*}[label=\alph*),itemjoin=\hspace{2em}]}{\end{enumerate*}\\[2ex]}

\DeclareMathOperator{\N}{\mathbb N}
\DeclareMathOperator{\Q}{\mathbb Q}
\DeclareMathOperator{\R}{\mathbb R}
\DeclareMathOperator{\Z}{\mathbb Z}
\DeclareMathOperator{\C}{\mathbb C}
\DeclareMathOperator{\F}{\mathcal{F}}
\DeclareMathOperator{\E}{\mathcal{E}}
\DeclareMathOperator{\BigO}{\mathcal O}
\DeclareMathOperator{\mL}{\mathcal{L}}
\DeclareMathOperator{\mU}{\mathcal{U}}
\DeclareMathOperator{\ggt}{\text{ggT}}
\DeclareMathOperator{\kgv}{\text{kgV}}


%------ Defining multiple types of theorems ------%
\newtheorem{axiom}{Axiom}[section]
\newtheorem{theorem}[axiom]{Theorem}
\newtheorem{lemma}[axiom]{Lemma}
\newtheorem{satz}[axiom]{Satz}
\theoremstyle{definition}
\newtheorem*{example}{Beispiel}
\newtheorem*{bemerkung}{Bemerkung}
\newtheorem*{frage}{Frage}
\newtheorem*{loesung}{Lösung}
\newtheorem*{behauptung}{Behauptung}
\newtheorem*{beobachtung}{Beobachtung}
\newtheorem{definition}[axiom]{Definition}
\newtheorem{folg}[axiom]{Folgerung}
\newtheorem{notation}[axiom]{Notation}

%---- Changing forall and exists ----%
\let\oldforall\forall
\renewcommand{\forall}{\:\oldforall \, }
\let\oldexist\exists
\renewcommand{\exists}{\:\oldexist \: }
\newcommand\existu{\oldexist! \: }
\let\oldepsilon\epsilon
\renewcommand{\epsilon}{\varepsilon}
\newcommand{\corresponds}{\;\widehat{=}\;}

%------ Giving circled star ---------%
\makeatletter
\newcommand{\ostar}{\mathbin{\mathpalette\make@circled\star}}
\newcommand{\make@circled}[2]{%
	\ooalign{$\m@th#1\smallbigcirc{#1}$\cr\hidewidth$\m@th#1#2$\hidewidth\cr}%
}
\newcommand{\smallbigcirc}[1]{%
	\vcenter{\hbox{\scalebox{0.77778}{$\m@th#1\bigcirc$}}}%
}
\makeatother
%-----------------------------------%

%-------------- Vars ---------------%
\newcommand{\vldate}{02. Dezember 2024}
\newcommand{\vlname}{Computerorientierte Mathematik I}
\newcommand{\vltopic}{Breitensuche und Tiefensuche}

\pagestyle{fancy}
\setlength{\headheight}{34pt}
\fancyhf{}
\fancyhead[L]{\includegraphics[width=1cm]{C:/Users/chole/Documents/tub/tublogo.png}}
\fancyhead[C]{{\bfseries\vlname}\\%
	\vltopic}
\fancyhead[R]{\vldate}
\fancyfoot[L]{}
\fancyfoot[C]{- \thepage\ of \pageref{LastPage} -}
\fancyfoot[R]{}

\renewcommand{\headrulewidth}{1pt}
\renewcommand{\footrulewidth}{1pt}
\renewcommand{\labelitemii}{-}

\begin{document}
	\setcounter{section}{7}
	\setcounter{subsection}{1}
	\subsection{Breitensuche und Tiefensuche}
	\subsubsection*{Breitensuche (BFS = breadth-first-search)}
	\begin{itemize}
		\item Berechneter Baum heißt BFS-Baum
		\item Für $v \in R$ ist $v$-$w$-Weg in $(R, F, \Psi\mid_F)$ kürzester $v$-$w$-Weg in $G$
		\item Mit Entfernungslabels für Knoten während der Breitensuche kann die Länge des kürzesten $v$-$w$-Weges für alle Knoten $v \in R$ bestimmt werden. 
	\end{itemize}
	\subsubsection*{Tiefensuche (DFS = depth-first-search)}
	\begin{itemize}
		\item Berechneter Baum heißt DFS-Baum
		\item Es gibt genau dann einen von $r$ aus erreichbaren Kreis in $G$, wenn der Algorithmus eine Kante von $v \in Q$ zu einem Knoten $w \in Q$ findet. Entfernt man alle Knoten aus $G$, so gibt es einen von $r$ aus erreichbaren Kreis mehr.
	\end{itemize}
	\section{Sortieren}
	\subsection{Partielle und totale Ordnungen}
	\begin{definition}
		Eine Relation $R \subseteq S \times S$ heißt partielle Ordnung auf der Menge $S$, falls
		\begin{enumerate}[label=(\roman*)]
			\item $(a, a) \in R$\hfill (Reflexivität)
			\item $(a, b) \in R \land (b, a) \in R \implies a = b$ \hfill (Antisymmetrie)
			\item $(a, b) \in R \land (b, c) \in R \implies (a, c) \in R$ \hfill (Transitivität)
		\end{enumerate}
		für alle $a, b, c \in S$ gilt: $R$ heißt totale (oder lineare) Ordnung, falls zusätzlich für alle $a, b \in S$ gilt:
		\begin{enumerate}[label=(\roman*)]
			\setcounter{enumi}{3}
			\item $(a, b) \in R \lor (b, a) \in R$
		\end{enumerate}
	\end{definition}
	Notation: Statt $(a, b) \in R$ schreiben wir auch $aRb$
	\begin{example}
		\begin{enumerate}[label=(\arabic*)]
			\item $R_{\leq} \coloneq \{(a, b) \in \N \times \N \mid a \leq b\}$ ist totale Ordnung auf $\N$.
			\item $R_{\geq} \coloneq \{(a, b) \in \R \times \R \mid a \geq b\}$ ist totale Ordnung auf $\R$.
			\item $R_{\mid} \coloneq \{(a, b) \in \N \times \N \mid a \mid b\}$ ist partielle Ordnung auf $\N$.
			\item $R \coloneq \big\{\big(\binom{a_1}{a_2}, \binom{b_1}{b_2}\big) \in \R^2 \times \R^2 \mid a_1 \leq b_1 \land a_2 \leq b_2\big\}$ ist partielle Ordnung auf $\R^2$. 
			\item $R \coloneq \big\{\big(\binom{a_1}{a_2}, \binom{b_1}{b_2}\big) \in \R^2 \times \R^2 \mid a_1 < b_1 \lor (a_1 = b_1 \land a_2 \leq b_2)\big\}$ ist totale Ordnung auf $\R^2$.
			\item Sei $G = (V, E, \Psi)$ azyklischer gerichteter Graph.
			
			$R \coloneq \{(v, w) \in V \times V \mid \exists v$-$w$-Weg$\}$ ist eine partielle Ordnung auf $V$.
		\end{enumerate}
	\end{example}
	\begin{definition}
		Es sei $\preceq$ eine partielle Ordnung auf einer endlichen Menge $S$ mit $\lvert S\rvert = n$. Eine Bijektion $\pi \coloneq \{1, 2, \ldots, n\} \to S$ heißt topologische Sortierung, falls $\pi(j) \npreceq \pi(i)$ für $i < j$. Ist $\preceq$ eine totale Ordnung, so ist die topologische Sortierung $\pi$ eindeutig. Es gilt
		\[
			\pi(1) \preceq \pi(2) \preceq \ldots \preceq \pi(n).
		\]
		In diesem Fall heißt $\pi$ Sortierung von $S$.
	\end{definition}
	\begin{example}
		Es sei $S = \{12, 7, 2, 3, 9, 6, 5\}$ und $\preceq$ die Teilbarkeitsrelation. Die topologische Sortierung ist hier:
		\[
			5, 3, 7, 2, 6, 12, 9
		\]
	\end{example}
	Es sei $\preceq$ partielle Ordnung auf einer endlichen Menge $S$ mit $\lvert S\rvert = n$.
	
	\begin{algorithm}[H]
		\caption{Das Sortierproblem}
		\KwIn{Liste/Array A der Länge $n$ mit Einträgen in $S$, also $S = \{A[0], A[1], \ldots, A[n - 1]\}$
		
		$\preceq$ gegeben als Orakel}
		\KwOut{Topologische Sortierung $\pi$ kodiert als Liste/Array}
	\end{algorithm}
	\subsection{Selection Sort}
	\begin{lstlisting}[language=Python]
for i in range(len(arr)):
	for j in range(i, len(arr)):
		if(ar[j] <= arr[i]):
			swap(arr[i], arr[j])
return arr\end{lstlisting}
	\begin{satz}
		Sortieren durch sukzessive Auswahl ist korrekt und hat Laufzeitfunktion in $\Theta(n^2)$.
	\end{satz}
\end{document}